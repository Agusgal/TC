\documentclass[a4paper]{article}
\usepackage[spanish]{babel}
%Para uso de palabras acentuadas
\usepackage[utf8]{inputenc}

\documentclass[a4paper]{article}
\usepackage[utf8]{inputenc}
\usepackage[spanish, es-tabla]{babel}

\usepackage{geometry}
 \geometry{includehead, footskip=7mm, headsep=6mm, headheight=4.8mm, top=25mm, bottom=5mm, left=10mm, right=10mm}

%\usepackage[a4paper, 					% Page Layout
%                     %showframe,				% This shows the frame
%                     includehead,
%                     footskip=7mm, headsep=6mm, headheight=4.8mm,
%                     top=25mm, bottom=5mm, left=5mm, right=5mm]{geometry}

\usepackage{amsmath}
\usepackage{amsfonts}
\usepackage{amssymb}

\usepackage{float}
\usepackage{graphicx}
\usepackage{caption}
\usepackage{subcaption}

\usepackage{multirow}
\setlength{\doublerulesep}{\arrayrulewidth}

\newcommand{\quotes}[1]{``#1''}

\usepackage{array}
\newcolumntype{C}[1]{>{\centering\let\newline\\\arraybackslash\hspace{0pt}}m{#1}}

\usepackage[american]{circuitikz}

\usepackage{fancyhdr}

\usepackage{units} 

\pagestyle{fancy}
\fancyhf{}
\lhead{22.01 Teoría de Circuitos}
\rfoot{Página \thepage}

\usepackage{float}
\usepackage{graphicx}

\usepackage[american voltage]{circuitikz}

\usepackage{amsmath}

\usepackage{xcolor}

\usepackage{caption}
\usepackage{subcaption}

\begin{document}

\section{Circuitos integradores y derivadores}
\subsection{Introducción}
Los circuitos implementados con amplificadores operacionales permiten la implementación de diferentes configuraciones que resuelven problemas matematicos. Con el diseño adecuado pueden usarse para resolver sistemas de ecuaciones diferenciales.
En este caso estudiaremos 2 bloques fundamentales. El circuito integrador y el circuito derivador. Se analizaran sus caracteristicas más relevantes así como tambien sus límites de uso y como extenderlos para aprovecharlos al máximo.
 
\subsection{Analisis teorico bajo diferentes condiciones de $A_{vol}$}
\subsubsection{Circuito Integrador}


\end{document}