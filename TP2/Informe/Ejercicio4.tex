\documentclass[a4paper]{article}
\usepackage[spanish]{babel}
%Para uso de palabras acentuadas
\usepackage[utf8]{inputenc}

\documentclass[a4paper]{article}
\usepackage[utf8]{inputenc}
\usepackage[spanish, es-tabla, es-noshorthands]{babel}
\usepackage[table,xcdraw]{xcolor}
\usepackage[a4paper, footnotesep = 1cm, width=20cm, top=2.5cm, height=25cm, textwidth=18cm, textheight=25cm]{geometry}
%\geometry{showframe}

\usepackage{tikz}
\usepackage{amsmath}
\usepackage{amsfonts}
\usepackage{amssymb}
\usepackage{float}
\usepackage{graphicx}
\usepackage{caption}
\usepackage{subcaption}
\usepackage{multicol}
\usepackage{multirow}
\setlength{\doublerulesep}{\arrayrulewidth}
\usepackage{booktabs}

\usepackage{hyperref}
\hypersetup{
    colorlinks=true,
    linkcolor=blue,
    filecolor=magenta,      
    urlcolor=blue,
    citecolor=blue,    
}

\newcommand{\quotes}[1]{``#1''}
\usepackage{array}
\newcolumntype{C}[1]{>{\centering\let\newline\\\arraybackslash\hspace{0pt}}m{#1}}
\usepackage[american]{circuitikz}
\usetikzlibrary{calc}
\usepackage{fancyhdr}
\usepackage{units} 

\graphicspath{{../Ejercicio-1/}{../Ejercicio-2/}{../Ejercicio-3/}{../Ejercicio-4/}}

\pagestyle{fancy}
\fancyhf{}
\lhead{22.01 Teoría de Circuitos}
\rhead{Mechoulam, Lambertucci, Rodriguez Turco, Londero, Galdeman}
\rfoot{\centering \thepage}

\usepackage{float}
\usepackage{graphicx}

\usepackage[american voltage]{circuitikz}

\usepackage{amsmath}

\usepackage{xcolor}

\usepackage{caption}
\usepackage{subcaption}

\begin{document}

\section{Circuitos integradores y derivadores}
\subsection{Introducción}
Los circuitos implementados con amplificadores operacionales permiten la implementación de diferentes configuraciones que resuelven problemas matemáticos. Con el diseño adecuado pueden usarse para resolver sistemas de ecuaciones diferenciales.
En este caso estudiaremos 2 bloques fundamentales. El circuito integrador y el circuito derivador. Se analizaran sus características más relevantes así como también sus límites de uso y como extenderlos para aprovecharlos al máximo.

\subsection{Análisis previo del \textbf{LM833N} }
El amplificador a utilizar es el \textbf{LM833N} de STMicroelectronics. El mismo es un operacional de bajo ruido diseñado para aplicaciones relacionadas al manejo de audio.
En su datasheet podemos ver que posee un alto nivel Slew Rate de hasta 7 $\frac{V}{s}$, un GBP de 15MHz y una ganancia de tensión 
a lazo abierto típica de unos 110 dB. Notemos que también se indica que la ganancia mínima es de unos 90 dB. 
 \subsubsection{Cálculo de $A_{vol}$}
Para poder realizar los cálculos de transferencia y establecer sus correspondientes transferencias de teóricas primero debemos conocer $A_{vol}$ medido en veces. Para esto basta con tomar el valor de ganancia de tensión típica a veces.
$$A_{vol} = 10^{\frac{110}{20}}$$
$$A_{vol} \approx 316227.77$$

\subsubsection{Cálculo de  la $f_p$, polo dominante}

Dado el \textbf{GBP} de 15MHz podemos fácilmente calcular la frecuencia de corte a lazo abierto

$$ f_p = \frac{15MHz}{A_{vol}} $$
$$ f_p \approx 47.44Hz$$





 
\subsection{Circuito Integrador}
\subsubsection{Integrador Ideal}


\end{document}