\documentclass[a4paper]{article}

\documentclass[a4paper]{article}
\usepackage[utf8]{inputenc}
\usepackage[spanish, es-tabla]{babel}

\usepackage{geometry}
 \geometry{includehead, footskip=7mm, headsep=6mm, headheight=4.8mm, top=25mm, bottom=5mm, left=10mm, right=10mm}

%\usepackage[a4paper, 					% Page Layout
%                     %showframe,				% This shows the frame
%                     includehead,
%                     footskip=7mm, headsep=6mm, headheight=4.8mm,
%                     top=25mm, bottom=5mm, left=5mm, right=5mm]{geometry}

\usepackage{amsmath}
\usepackage{amsfonts}
\usepackage{amssymb}

\usepackage{float}
\usepackage{graphicx}
\usepackage{caption}
\usepackage{subcaption}

\usepackage{multirow}
\setlength{\doublerulesep}{\arrayrulewidth}

\newcommand{\quotes}[1]{``#1''}

\usepackage{array}
\newcolumntype{C}[1]{>{\centering\let\newline\\\arraybackslash\hspace{0pt}}m{#1}}

\usepackage[american]{circuitikz}

\usepackage{fancyhdr}

\usepackage{units} 

\pagestyle{fancy}
\fancyhf{}
\lhead{22.01 Teoría de Circuitos}
\rfoot{Página \thepage}

\usepackage{float}
\usepackage{graphicx}

\usepackage{xcolor}

\usepackage{caption}
\usepackage{subcaption}

\begin{document}

\subsection{Diseño del circuito}

En el siguiente punto se propuso como objetivo realizar un sensor de temperatura, valiéndose del uso del transistor LM35. Este dispositivo se caracteriza por entregar a la salida una tensión proporcional a la temperatura de la forma
\[
	V_{LM35} = 10 \frac{mV}{^{\circ}C} \ T
\]

Como se requiere que el circuito diseñado posea una tensión de salida de $ 0 \ V $ a $ 35 ^{\circ}C $ y $ 5 \ V $ a $ 45 ^{\circ}C $, se busca la ecuación de la recta que atraviese los puntos especificados. Es así que se obtiene que la tensión de salida viene dada por la siguiente ecuación:

\begin{equation}
	V_{out} = 50V_{LM35} - 17.5 \ V =  0.5 \frac{V}{^{\circ}C} \ T - 17.5 \ V
	\label{equ:sistema}
\end{equation}

Para lograr su funcionamiento, el fabricante especifica que el transistor debe ser alimentado con una tensión de entre $4 \ V$ y $20 \ V$. Es por ello que se decidió suministrarle a este dispositivo $10 \ V$ para utilizarlo. Además, por cuestiones de seguridad, se desea que el circuito no brinde una tensión de salida menor a $-1 \ V$ y mayor a $6 \ V$. Esto se ahondará con mayor profundidad en los de diseño del modelo.

Luego se requiere que la tensión de salida del circuito responda de la siguiente forma frente a la tensión de salida del transistor.

\begin{figure}[H]
	\centering
	\includegraphics[width=0.8\textwidth]{Ejercicio6/Imagenes/SalidaVsVLM35.png}
\caption{Tensión de salida Vs. Tensión del LM35.}
	\label{fig:vout}
\end{figure}



\subsubsection{Primer modelo: inversor y restador}

En un principio, se decidió armar un circuito compuesto por tres diferentes etapas, las dos primeras compuestas cada una por un amplificador operacional TL072 y la final por un diodo zener \textcolor{red}{\textbf{PONER NOMBRE DEL ZENER}}. Cabe destacar que por cuestiones de especificación del fabricante, se decidió alimentar a ambos operacionales con $10 \ V$ y $-10 \ V$.

En la primer etapa se dispuso un amplificador inversor, tomando como tensión de entrada la de salida del transistor.

\begin{figure}[H]
	\centering
	\includegraphics[width=0.8\textwidth]{Ejercicio6/Imagenes/CircuitoEtapa1-M1.png}
\caption{Primera etapa del circuito.}
	\label{fig:cir1-M1}
\end{figure}

De esta forma, y siendo $V_1$ la tensión de salida de esta etapa, se logra
\[
	V_1 = -\frac{R_6}{R_5} \ V_{LM35}
\]

En la segunda etapa, a diferencia de la primera, se utilizó un amplificador restador. Se decidió conectar al borne negativo del operacional la tensión $V_1$ y al  una positivo tensión de $- 10 \ V$, reutilizando la alimentación del operacional.

\begin{figure}[H]
	\centering
	\includegraphics[width=0.8\textwidth]{Ejercicio6/Imagenes/CircuitoEtapa2-M1.png}
	\caption{Segunda etapa del circuito.}
	\label{fig:cir2-M1}
\end{figure}

Aplicando el teorema de superposición, se puede demostrar que la transferencia de este circuito es
\[
	V_2 = \frac{R_3}{R_2} \cdot \frac{R_2 // R_4}{R_1 // R_3} \left( -V_{CC} \right) \ - \ \frac{R_3}{R_1} V_1
\]

Finalmente, la tercer etapa, consiste simplemente en una resistencia con un diodo zener en serie, cumpliendo la función de limitar la salida entre $-0,7 \ V$ y $5.6 \ V$. El objetivo de esta sección es simplemente limitar la tensión de salida, evitando que se encuentre por debajo de $-1 \ V$ y por encima de $6 \ V$.

\begin{figure}[H]
	\centering
	\includegraphics[width=0.6\textwidth]{Ejercicio6/Imagenes/CircuitoEtapa3-M1.png}
	\caption{Tercer etapa del circuito.}
	\label{fig:cir3}
\end{figure}

Es así como se observa al circuito final en la Figura (\ref{fig:cirfin-M1}).

\begin{figure}[H]
	\centering
	\includegraphics[width=0.99\textwidth]{Ejercicio6/Imagenes/CircuitoFinal-M1.png}
	\caption{Modelo final del circuito.}
	\label{fig:cirfin-M1}
\end{figure}

Se calcula la transferencia utilizando las ecuaciones parciales mostradas anteriormente. Es así que se llega a
\begin{equation}
	V_{out} = \frac{R_3}{R_2} \cdot \frac{R_2 // R_4}{R_1 // R_3} \left( -V_{CC} \right) \ + \
	\frac{R_3}{R_1} \frac{R_6}{R_5} \ V_{LM35}
	\label{equ:transfm1}
\end{equation}

A la hora de seleccionar los valores de las resistencias, se pide:
\begin{equation}
	\frac{R_6}{R_5} = 5
	\label{equ:condm1-1}
\end{equation}

\begin{equation}
	\frac{R_3}{R_1} = 10
	\label{equ:condm1-2}
\end{equation}

\begin{equation}
	\frac{R_3}{R_2} = 1.75
	\label{equ:condm1-3}
\end{equation}

\begin{equation}
	\frac{R_2 // R_4}{R_1 // R_3} = 1
	\label{equ:condm1-4}
\end{equation}

Cumpliendo con esto y eligiendo arbitrariamente $R_6 = 5 \ k\Omega$ y $R_2 = 10 \ k\Omega$ se llega a los siguientes valores: $R_5 = 1 \ k\Omega$, $R_1 = 1.75 \ k\Omega$, $R_3 = 17.5 \ k\Omega$ y $R_4 = \frac{37}{7} \ k\Omega \approx 1.89 \ k\Omega$. De esta forma se consigue llegar a la Ecuación (\ref{equ:sistema}).

Por otro lado, para $R_7$ y debido a la función que cumple, para seleccionar un valor adecuado, se considera la máxima tensión que se puede obtener a la salida el operacional, $13.5 \ V$ en el peor caso, según el fabricante, y se asume una corriente de $0.5 \ mA$ para evitar que se sobre exija al operacional:

\begin{equation}
	R_7 = \frac{13.5 \ V \ - \ 5.6 \ V}{0.5 \ mA} = 15.8 \ k\Omega \approx 16 \ k\Omega
	\label{equ:condm1-5}
\end{equation}

Una vez establecido tanto la forma del circuito como sus componentes, se procedió a analizar simulaciones del mismo mediante el uso del programa LTSpice. En este, se valió del comando ``.step'' para analizar cada resistencia con una tolerancia del 5\%. Es por ello que se realiza un detenimiento en las resistencias $R_1$ y $R_4$.

\begin{figure}[H]
	\centering
	\includegraphics[width=0.99\textwidth]{Ejercicio6/Imagenes/StepR1-M1.png}
	\caption{Salida del circuito en función de la tensión del transistor variando $R_1$.}
	\label{fig:r1-M1}
\end{figure}

\begin{figure}[H]
	\centering
	\includegraphics[width=0.99\textwidth]{Ejercicio6/Imagenes/StepR4-M1.png}
	\caption{Salida del circuito en función de la tensión del transistor variando $R_4$.}
	\label{fig:r4-M1}
\end{figure}

Ante pequeñas variaciones en $R_1$, la pendiente de la recta que describe la salida en función de la tensión del LM35 varía. De igual forma ocurre con $R_4$ y la ordenada al origen de dicha recta. Es por eso que se considera que una solución posible para esto es reemplazando tanto $R_1$ como $R_4$ por presets. De esta forma, se podrá seleccionar una valor más adecuado para obtener una salida más estable.

\subsubsection{Segundo modelo: restador e inversor}

Al igual que el circuito anterior, este está compuesto por las mismas tres etapas, con la única diferencia que se utiliza primero el amplificador restador. No se han variado ni el diodo zener utilizado, ni los modelos de operacionales utilizados, únicamente la tensión de alimentación de ambos operacionales a $12 \ V$ y $-12 \ V$.

En la primer etapa de este nuevo circuito, se suministra la tensión brindada por el transistor al borne negativo del operacional, mientras que para el positivo, se busca reducir la tensión de $12 \ V$ de la fuente de alimentación mediante un divisor resistivo. Con esto último, se busca tener una tensión menor en el nodo $V_2$, para luego poder amplificarla con mayor grado de libertad.

\begin{figure}[H]
	\centering
	\includegraphics[width=0.8\textwidth]{Ejercicio6/Imagenes/CircuitoEtapa1-M2.png}
\caption{Primera etapa del circuito.}
	\label{fig:cir1-M2}
\end{figure}

De la misma forma que con el primer modelo, se puede demostrar que la transferencia de esta etapa se encuentra dada por
\[
	V_1 = \frac{R_3}{R_2} \cdot \frac{R_2 // R_4}{R_1 // R_3} \left( V_{CC} \cdot \frac{R_8}{R_7 + R_8} \right) \ - \ \frac{R_3}{R_1} V_{LM35}
\]

Luego, para la parte del amplificador inversor, simplemente se alimentó con la tensión de salida de la etapa previa al borne negativo, mientras que se conectó el positivo a tierra.

\begin{figure}[H]
	\centering
	\includegraphics[width=0.8\textwidth]{Ejercicio6/Imagenes/CircuitoEtapa2-M2.png}
	\caption{Segunda etapa del circuito.}
	\label{fig:cir2-M2}
\end{figure}

siendo la transferencia de esta última
\[
	V_3 = - \frac{R_6}{R_5} \ V_1
\]

La etapa final, es decir, la protección, no cambia para este circuito. Tanto el diodo zener, como la resistencia seleccionada, son las mismas. De esta forma, la transferencia final queda de la forma
\begin{equation}
	V_{out} = - \frac{R_6}{R_5} \left[ \frac{R_3}{R_2} \cdot \frac{R_2 // R_4}{R_1 // R_3} \left( V_{CC} \cdot \frac{R_8}{R_7 + R_8} \right) \ - \ \frac{R_3}{R_1} V_{LM35} \right]
	\label{equ:transfm2}
\end{equation}

\begin{figure}[H]
	\centering
	\includegraphics[width=0.99\textwidth]{Ejercicio6/Imagenes/CircuitoFinal-M2.png}
	\caption{Modelo final del circuito.}
	\label{fig:cirfin-M2}
\end{figure}

Nuevamente, se imponen condiciones para obtener los valores de las resistencias:
\begin{equation}
	V_2 = 0.35 \ V
	\label{equ:condm2-1}
\end{equation}

\begin{equation}
	\frac{R_3}{R_1} = 10
	\label{equ:condm2-2}
\end{equation}

\begin{equation}
	\frac{R_2 // R_4}{R_1 // R_3} = 1
	\label{equ:condm3-3}
\end{equation}

\begin{equation}
	\frac{R_3}{R_2} = 10
	\label{equ:condm2-4}
\end{equation}

\begin{equation}
	\frac{R_6}{R_5} = 5
	\label{equ:condm2-5}
\end{equation}

Es así que, eligiendo arbitrariamente $R_6 = 5 \ k\Omega$, $R_1 = 10 \ k\Omega$ y $R_7 = 100 \ k\Omega$ y utilizando las condiciones impuestas, se llega a: $R_2 = 10 \ k\Omega$, $R_3 = 100 \ k\Omega$, $R_4 = 100 \ k\Omega$, $R_5 = 1 \ k\Omega$ y $R_8 = \frac{700}{233} \ k\Omega \approx 3 \ k\Omega$. Nuevamente, se consigue con estos valores llegar a la Ecuación (\ref{equ:sistema}). Cabe recordar que como la tercer etapa no ha sido alterada, el valor de la resistencia, en este caso llamada $R_9$, sigue siendo $16 \ k\Omega$.

Se repite el análisis empleado para el primer circuito en LTSpice. En este caso, son las resistencias $R_5$ y $R_8$ las que resaltan.

\begin{figure}[H]
	\centering
	\includegraphics[width=0.99\textwidth]{Ejercicio6/Imagenes/StepR5-M2.png}
	\caption{Salida del circuito en función de la tensión del transistor variando $R_5$.}
	\label{fig:r5-M2}
\end{figure}

\begin{figure}[H]
	\centering
	\includegraphics[width=0.99\textwidth]{Ejercicio6/Imagenes/StepR8-M2.png}
	\caption{Salida del circuito en función de la tensión del transistor variando $R_8$.}
	\label{fig:r8-M2}
\end{figure}

La pendiente de la recta que describe la salida varía con los cambios de $R_5$. Por otro lado, ordenada al origen de esta cambia con $R_4$. Nuevamente se propone como solución para esto reemplazar ambas resistencias presets.

\subsubsection{Selección del modelo más óptimo}

Se han presentado dos modelos válidos que cumplen con las especificaciones del circuito. Uno de los dos debe ser descartado. Para tomar dicha decisión se valió de la ayuda del análisis de Montecarlo, que permite realizar el programa LTSpice. Para este, se considero la tolerancia de todas las resistencias a excepción de las 4 que fueron reemplazadas por presets. Debido a que los gráficos de dicho análisis que brinda LTSpice pueden llegar a ser confusos, se decidió tomar para cada caso las curvas más alejadas de la ideal y mostrar el área encerrada entre estas primeras. Es así que se muestra dicho resultado en la Figura (\ref{fig:mccomp}). 

\begin{figure}[H]
\centering
\begin{subfigure}{.8\textwidth}
  \centering
  \includegraphics[width=.85\linewidth]{Ejercicio6/Imagenes/MC-1M.png}
  \caption{Primer modelo propuesto.}
  \label{fig:mcm1}
\end{subfigure}
\begin{subfigure}{\textwidth}
  \centering
  \includegraphics[width=.7\linewidth]{Ejercicio6/Imagenes/MC-2M.png}
  \caption{Segundo modelo propuesto.}
  \label{fig:mcm2}
\end{subfigure}
\caption{Comparación del los análisis de Montecarlo.}
\label{fig:mccomp}
\end{figure}

Como se observa en la Figura (\ref{fig:mccomp}), el área encerrada entre ambos extremos del primer circuito diseñado es mucho mayor que la del segundo, es decir, el primero es mucho más sensible a pequeños cambios de los componentes existentes. Fue por ello que el primer modelo propuesto fue descartado, mientras que el segundo fue aceptado. 

\begin{center}
	\textcolor{red}{\textbf{PODRÍA BUSCAR ALGUNA JUSTIFICACIÓN DE PORQUÉ SON ASÍ LOS CIRCUITOS}}\\
	\textcolor{red}{\textbf{ESTARÍA BUENO PONER MÁS RAZONES DE PORQUÉ SE ELIGIÓ EL 2 Y NO EL 1}}
\end{center}

\subsection{Modo de uso}
%alimentacion, lectura, calibracion


\subsection{Detalles técnicos}
%Mediciones, corrientes/tensiones maximas, rango de temperatura, 

%El LM35 es un circuito integrado cuya tensi´on de salida var´ıa linealmente con la temperatura. Se desea que la
%se~nal pueda ser adquirida por un sistema con (por ejemplo, un conversor anal´ogico/digital) con tensi´on de entrada
%variable entre 0V y 5V .
%a. Dise~nar un circuito utilizando el LM35 que adapte la se~nal para que pueda ser adquirida con m´axima excursi´on
%para temperaturas que var´ıen entre 35◦C y 45◦C (35◦C ! 0V − 45◦C ! 5V ).
%b. Implementar el circuito en placa multiperforada o PCB.
%c. Dise~nar un m´etodo de calibraci´on del circuito para que se cumpla la especificaci´on.
%d. El circuito debe contar con una protecci´on de forma tal de que la tensi´on de salida no se encuentre por debajo
%de −1V ni por encima de 6V .
%e. Incluir en el informe un datasheet de la implementaci´on final, incluyendo toda la informaci´on relevante.


\end{document}