\documentclass[border={0.5cm 0.5cm 0.5cm 0.5cm}]{standalone}
\usepackage[utf8]{inputenc}
\usepackage[spanish, es-tabla, es-noshorthands]{babel}

\usepackage[a4paper, footnotesep = 1cm, width=18cm, left=2cm, top=2.5cm, height=25cm, textwidth=18cm, textheight=25cm]{geometry}
%\geometry{showframe}

\usepackage{tikz}
\usepackage{textcomp}
\usetikzlibrary{shapes,arrows}

\usepackage{amsmath}
\usepackage{amsfonts}
\usepackage{amssymb}
\usepackage{float}
\usepackage{graphicx}
\usepackage{caption}
\usepackage{subcaption}
\usepackage{multicol}
\usepackage{multirow}
\setlength{\doublerulesep}{\arrayrulewidth}
\usepackage{booktabs}

\usepackage{hyperref}
\hypersetup{
    colorlinks=true,
    linkcolor=blue,
    filecolor=magenta,      
    urlcolor=blue,
    citecolor=blue,    
}

\newcommand{\quotes}[1]{``#1''}
\usepackage{array}
\newcolumntype{C}[1]{>{\centering\let\newline\\\arraybackslash\hspace{0pt}}m{#1}}
\usepackage[american]{circuitikz}
\usepackage{fancyhdr}
\usepackage{units}

% Definition of blocks:
\tikzset{%
  block/.style    = {draw, thick, rectangle, minimum height = 3em,
    minimum width = 3em},
  sum/.style      = {draw, circle, node distance = 2cm}, % Adder
  input/.style    = {coordinate}, % Input
  output/.style   = {coordinate}, % Output
  >=Stealth
}

% Defining string as labels of certain blocks.
\newcommand{\suma}{\Large $\Sigma$}
\newcommand{\inte}{$\displaystyle \int$}
\newcommand{\derv}{\huge $\frac{d}{dt}$}

\begin{document}

%Oscilador de Wien
\begin{page}
\begin{circuitikz}
	\node [op amp, yscale=-1](oa){};
	\draw (oa.out) -- ++(2,0) node[](vo){} -- ++(0.5,0) node[label=right:$V_o$, ocirc](){};
	\draw (oa.-) -- ++(0,-1.5) to[R, l=$R_2$] ++(4,0) -| (vo.center);
	\draw (oa.-) ++(0,-1.5) to[R, l=$R_1$] ++(0,-2) node[ground](){};
	\draw (oa.+) -- ++(0,1.5) node[circ, label=above:$V_p$](aux1){} to[C, l=$C$] ++(2,0) to[R, l=$R_1$] ++(2,0) -| (vo.center);
	\draw (aux1) -- ++(-2.5,0) to[C, l_=$C$] ++(0,-2) node[ground](gnd){}; 
	\draw (aux1) ++(-1,0) to[R, l=$R$] ++(0,-2) |- (gnd);	
\end{circuitikz}
\end{page}

%Puente de Wien con ACG
\begin{page}
\begin{circuitikz}
	\node [op amp, yscale=-1](oa){};
	\draw (oa.out) -- ++(2,0) node[](vo){} -- ++(0.5,0) node[label=right:$V_o$, ocirc](){};
	\draw (oa.-) -- ++(0,-1.5) to[R, l=$R_2$] ++(4,0) -| (vo.center);
	\draw (oa.-) ++(0,-1.5) to[R, l=$R_1$] ++(0,-2) node[](aux2){};
	\draw (oa.+) -- ++(0,1.5) node[circ, label=above:$V_p$](aux1){} to[C, l=$C$] ++(2,0) to[R, l=$R_1$] ++(2,0) -| (vo.center);
	\draw (aux1) -- ++(-2.5,0) to[C, l_=$C$] ++(0,-2) node[ground](gnd){}; 
	\draw (aux1) ++(-1,0) to[R, l=$R$] ++(0,-2) |- (gnd);
		
	\draw (aux2.center) node[njfet, anchor=D, xscale=-1](jfet){};
	\draw (jfet.S) -- ++(0,-1.5) node[ground](){};
	\draw (jfet.G) ++(1,0) to[C, l=$C_3$] ++(0,-2) node[ground](){};
	\draw (jfet.G) ++(3.4,0) to[R, l=$R_X$] ++(0,-2) node[ground](){};
	
	\draw (vo.center) ++(0,-2) node[](aux3){} ++(0,-1.5) node[](aux4){} to[zzDo, mirror, l=$1N5230$] ++(0,-1.5) |- (jfet.G);
	\draw (aux4) to[Do, l_=$1N4148$] (aux3);	
\end{circuitikz}
\end{page}

\end{document}