\documentclass[a4paper]{article}
\usepackage[utf8]{inputenc}
\usepackage[spanish, es-tabla, es-noshorthands]{babel}
\usepackage[table,xcdraw]{xcolor}
\usepackage[a4paper, footnotesep = 1cm, width=20cm, top=2.5cm, height=25cm, textwidth=18cm, textheight=25cm]{geometry}
%\geometry{showframe}

\usepackage{tikz}
\usepackage{amsmath}
\usepackage{amsfonts}
\usepackage{amssymb}
\usepackage{float}
\usepackage{graphicx}
\usepackage{caption}
\usepackage{subcaption}
\usepackage{multicol}
\usepackage{multirow}
\setlength{\doublerulesep}{\arrayrulewidth}
\usepackage{booktabs}

\usepackage{hyperref}
\hypersetup{
    colorlinks=true,
    linkcolor=blue,
    filecolor=magenta,      
    urlcolor=blue,
    citecolor=blue,    
}

\newcommand{\quotes}[1]{``#1''}
\usepackage{array}
\newcolumntype{C}[1]{>{\centering\let\newline\\\arraybackslash\hspace{0pt}}m{#1}}
\usepackage[american]{circuitikz}
\usetikzlibrary{calc}
\usepackage{fancyhdr}
\usepackage{units} 

\graphicspath{{../Ejercicio-1/}{../Ejercicio-2/}{../Ejercicio-3/}{../Ejercicio-4/}}

\pagestyle{fancy}
\fancyhf{}
\lhead{22.01 Teoría de Circuitos}
\rhead{Mechoulam, Lambertucci, Rodriguez Turco, Londero, Galdeman}
\rfoot{\centering \thepage}
\begin{document}

\subsection{Introducción}

Un oscilador es un circuito electrónico capaz de producir una señal oscilante periódica, a menudo senoidal o cuadrada. Existen dos tipos de oscialdors: los oscialdores armónicos y los osciladores de relajación. Los primeros producen una señal sinusoidal a la salida y se rijen por el criterio de BarkHausen. En cambio, los osciladores de relajación u osciladores no lienales, producen una señal no sinusoidal a su salida, como por ejemplo una triangular, un diente de sierra o un tren de pulsos. Basicamente concisten en dos partes: un elemento que almacena energía (por lo general un copacitor) y un dispositivo de switching no lineal, ambos conectados por algún tipo de realimentación. El dispositivo switching se encarga de cargar y descargar el capacitor periodicamente, ocasionando cambios abruptos en la salida del circuito. 


\subsection{Oscilador controlado por tension}

Un tipo especial de osciladores de relajación son los VCO u osciladores controlados por voltaje por sus siglas en inglés. La función principal de los mismos es la de convertir una señal DC de entrada a una frecuencia de señal a la salida, frecuentemente una señal sinusoidal o triangular. Al ser un tipo de osciladores los VCOs poseen dos partes, una activa que actúa como amplificador y una red de retroalimentación que provee retroalimentación positiva al sistema. Esta red de retroalimentación contiene por lo general un elemento de reactancia variable, el cuál controla la frecuencia de salida del circuito. 
Los VCOs poseen infinidad de aplicaciones, entre ellas son componentes fundamentales de los circuitos amarradores de fase (PLL), sirven de sintetizadores de frecuencia controlables, son usados también como entrada de frecuencia de la portadora para moduladores, etc.


%%Agregar teroia sobre distorsion armonica y jitter aca
En la siguiente figura puede verse un diagram del circuito utilizado:

\begin{figure}[H]
\begin{center}
\begin{circuitikz}
	
	\ctikzset{tripoles/mos style/arrows}
	\node [op amp, label = center:$AO1$](U1){};
	\draw (U1.-) to[short] ++(-1, 0) node[](v-){};
	\draw (v-) to[short, *-*] ++(0, 1) node[](cizq){};
	\draw (cizq) to[R, l = $2R$, *-*] ++(-2.5, 0) node[](2rizq){}	;
	\draw (2rizq) to[short] ++(-1.5, 0) to[american voltage source, l = $V_{in}$] ++(0, -3) node[ground]{};
	\draw (2rizq) coordinate(leftR3) to[R, *-*, l = $10k\Omega$] (leftR3 |- U1.+) to[short] (U1.+);
	\draw (leftR3 |- U1.+) to[R, *-, l = $10k\Omega$] ++(0, -2) node[ground]{};
	
	\draw (v-) to[R, l = $R$] ++(0, -6.5) node[nmos,xscale=-1](mos){};
	\draw (mos.drain) node[anchor = north]{} (mos.source) node[anchor = south]{} (mos.gate) node[anchor = east]{};
	\draw (mos.source) node[ground]{};
	
	\draw (cizq) to[short] ++(0, 0.5) coordinate(leftC) to[C, l = $C$] (leftC -| U1.out) to[short] (U1.out);
	
	\draw (U1.out) to[open] ++(3, -0.5) node[op amp, label=center:$AO2$](CMP){};
	
	\draw (CMP.-) to[short] (U1.out);
	
	\draw (CMP.out) to[short] ++(1, 0) node[](out){};
	
	
	\draw (mos.gate) to[R, l = $1k\Omega$] (mos.gate -|out) to[short] (out);	
	
	\draw (CMP.+) to[short] ++(-0.5, 0) to[short, -*] ++(0, -1) node[](feedback){};
	 
	\draw (feedback) to[R, l = $1k\Omega$] ++(0, -2) node[ground]{};
	
	\draw (feedback) to[vR, l = $5k\Omega$, *-*] (feedback -| out);
	
	\draw (U1.out) to[open] ++(1, 0) to[short,label = $V_{TR}$, -*] ++(0, 1);
	
	\draw (out) to[short, -*] ++(1, 0) node[label = right:$V_{Sq}$](){};
	
\end{circuitikz}
	\caption{VCO implementado}
	\label{fig:VCO}
\end{center}
\end{figure}

Como se mencionó anteriormente el circuito consta de dos partes que logran tener una señal triangular a la salida con frecuencia dependiente de la tension $V_{in}$. Más precisamente la frecuencia final de la señal de salida se comportará de la siguiente manera:

\begin{equation}
f_0 = kV_{in}  \qquad  V_{in} > 0
\end{equation}

donde $k$ es la sensibilidad del VCO en Hertz por Volt. \newline


El primer amplificador operacional se comporta como convertidor $V-I$ y obliga al capacitor a conducir una corriente linelamente proporcional a $V_{in}$. Pero para conseguir una onda triangular a la salida, el capacitor debe cargarse y descargarse y, por ende, alternar entre polaridades opuestas. Dicha polaridad se controla mediante el transsitor MOSFET tipo n que esta actuando de interruptor en este circuito. Si el transistor conduce por al resistencia R circula una corriente igual a $\frac{V_{in}}{2R}$, pero solo la mitad de esta corriente es suministrada por la resistencia $2R$, por ende, la mitad restante la suministra el capacitor. En el caso que el transistor este apagado y no conduzca, toda la corriente que llega al terminal no inversor debe fluir por el capacitor pero con sentido contrario al anterior. 


A partir de la ecuación fundamental de carga y descarga del capacitor:

\begin{equation}
\Delta t = \frac{C}{I}\Delta v 
\end{equation}

y teniendo en cuenta la corriente que circula por el mismo se llega a la siguiente expresión que caracteriza la salida del primer amplificador operacional:

\begin{equation}\label{eq:rampa}
V_{TR} = \frac{V_{in}}{2} + V_C = \frac{V_{in}}{2}(1 \pm \frac{\Delta t}{2RC})
\end{equation}

Entonces los cambios en la tension de salida $V_{TR}$ vienen dados por:

\begin{equation}
\Delta V_{TR	} = V_{in} \frac{\Delta t}{4RC}
\end{equation}

 

Por otro lado, el segundo op-amp forma un Schmidtt trigger el cual controla la tension en la base del transistor y por consiguiente, la pendiente de la rampa. La salida del Schmidtt trigger cumple $V_{out} = A_{vol}(V^{+} - V^{-})$, limitada por la saturación del operacional. Dada esta ecuación el comparador tiene dos estados posibles:

\begin{itemize}
	\item Con $V^{+} > V_{TR}$ $V_{SQ}$ adopta el valor $+V_{SAT}$ y luego $V^{+}$ será un valor menor a $V_{SAT}$ que dependerá del valor de la resistencias de realimentación, este estado se define como $V_{TH}$ (trigger high).
	\item Con  $V^{+} < V_{TR}$ $V_{SQ}$ adopta el valor $-V_{SAT}$ y ocurre lo mismo que en el caso anterior con $V^{+}$, este estado será definido como $V_{TL}$ (trigger low).
\end{itemize}


En consecuencia, la rampa cambiará de pendiente positiva  negativa cuando $V_{TR} = V_{TH}$ y de pendiente negtaiva a positiva cuando $V_{TR} = V{TL}$. Si se reemplaza $\Delta V_{TR}$ por $V_{TH} - V_{TL}$ y $\Delta t $ por $\frac{1}{2f_0}$ en la ecuación \ref{eq:rampa}:

\begin{equation}\label{eq:frecuencia}
f_0 = \frac{V_{in}}{8RC(V_{TH} - V_{TL})}
\end{equation}

consiguiendose así una frecuecnia variable dependiente de la tensión de entrada. \newline

El diseño anterior funciona como VCO pero posee dos complicaciones: \newline

En primer lugar, cuando la señal de entrada es nula la salida también. Por ende, debe agregarse una etapa que logre una tensión de 1 Volt cuando la entrada sea nula, y de 10 Volt cuando sea igual a 5, ya que así se obtiene una relación 1:1 tensión-frecuencia. \newline

En segundo lugar, el VCO diseñado es un oscilador de relajación, es decir, con salida no sinusoidal, por ende debe efectuarse una conversión triangular-sinusoidal a la salida del primer opamp. 

\subsection{Eleccion de componentes}

Los primeros Componentes a tener en cuenta son el capacitor y las resistencias del primer opamp. Deben elegirse dos valores respetandose la tension máxima de saturación del Schmidtt trigger $V_{TH} - V_{TL}$, la cual rondará los 13 Volt. Se hizo la resistencia de realimentación del comparador variable ya que así puede calibrarse la tensión de disparo para los valores de resistencias y capacitor elegidos. Con un capacitor de $10nF$ y resistencias de $1.5k\Omega$ para una tensión de entrada de 1 Volt debe ser igual a $1kHz$:

\begin{equation}
1000 = \frac{1}{8\cdot 1500 \cdot 10^{-8} \cdot (V_{TH} - V_{TL}) } \Rightarrow (V_{TH} - V_{TL}) = 8.33 \; Volt
\end{equation}

Por ende se esta dentro del rango de tensiones adecuado. \newline

El operacional usado es el TL084, contiene los 4 opamps necesarios para el diseño y posee características deseadas como alto slew-rate ($13 \frac{V}{\ s}$), baja corriente de bias ($30pA$) y baja distorsión armónica (menor al $0.0003\%$).

%%Agregar transistor usado.
\subsection{Cirucito Sumador}

El circuito deseado debe cumplir la siguiente ecuación de la recta:

\begin{equation}\label{eq:lineal}
V_{out} = \frac{9}{5}V_{in} + 1
\end{equation}

Lo natural es utilizar con este propósito un circuito sumador no inversor:

\begin{figure}[H]
\begin{center}
\begin{circuitikz}
	
	\ctikzset{tripoles/mos style/arrows}
	
	\node [op amp, label = center:$AO1$, yscale = -1](U1){};
	
	\draw (U1.+) to[short] ++(-0.5, 0) node[](vmas){} to[short] ++(0, 1.5) to[R, l = $R_1$, -*] ++(-2, 0) node[label = left:$15V$](15v){};

	\draw (vmas) to[R, label = $R_2$, -*] ++(-2, 0) node[label = left:$V_{in}$](vin){};
	
	\draw (U1.-) to[short] ++(-0.5, 0) to[short] ++(0, -1.3) node[](feedback){};
	\draw (feedback) to[R, l = $R_4$] ++(0, -2) node[ground](){};
	
	\draw (feedback) to[R, l = $R_3$, *-] (feedback -| U1.out) to[short] (U1.out);
	
	\draw (U1.out) to[short, -*] ++(0.5, 0) node[label = right:$V_{out}$](out){};
	
	
	
\end{circuitikz}
	\caption{Sumador no Inversor}
	\label{fig:sum}
\end{center}
\end{figure}

Si se usa el teorema de superposición se llega a la siguiente ecuación para la tension de salida:

\begin{equation}
V_{out} = (\frac{1 + \frac{R_3}{R_4}}{R_1 + R_2})\cdot (R_1\cdot V_{in} + R_2\cdot 15)
\end{equation}

Si se iguala coeficiente a coeficiente la expresión anterior con \ref{eq:lineal} se llega a la siguiente relación entre resistencias:

\begin{equation}
1 + \frac{R_3}{R_4} = \frac{9}{5} \cdot (1 + \frac{R_2}{R_1}) =\frac{1}{15} \cdot (1 + \frac{R_1}{R_2})
\end{equation}

Despejando de la segunda igualdad se llega a la siguiente relación entre $R_1$ y $R_2$:

\begin{equation}
R_1 = 27R_2
\end{equation}

si luego se despeja la primera igualdad:

\begin{equation}
13R_4 = 15R_3
\end{equation}

Los siguientes valores comerciales cumplen a la perfección ambas igualdades:

\begin{itemize}
\item $R_1 = 1k\Omega$
\item $R_2 = 27k\Omega$
\item $R_3 = 1.3k\Omega$
\item $R_4 = 1.5k\Omega$
\end{itemize}

\newpage

\subsection{Conversion onda triangular a senoidal}
Existen diversor métodos para transformar una señal triangular a senoidal, en este caso se optó por la utilización de un par diferencial, ya que si se hubiera utilizado algún tipo de filtro la respuesta en frecuencia podría afectar la trasnformación.


\begin{figure}[H]
\begin{center}
\begin{circuitikz}
	
	\draw node[label = left:$V_{in}$](beginning){};
	\draw (beginning) to[short, o-] ++(1, 0) node[](leftp1){} to[american potentiometer,n=mypot,l = $100k\Omega$] ++(2,0) to[short, -*] ++(1, 0) node[](leftQ1){} to[R, l = $2.2k\Omega$] ++(0, -2.5) node[ground]{};
	\draw (mypot.wiper) to[short] ++(-1,0) to[short] (leftp1);
	
	\draw (leftQ1) to[open] ++(2, 0) node[npn](Q1){};
	\draw (leftQ1) to[short] (Q1.B);
	\draw (Q1.E) to[short] ++(0, -1) node[](botQ1){};
	\draw (botQ1) to[R, l = $680\Omega$] ++(2.5, 0) node[](botQ2){};
	\draw (botQ1) to[short] ++(0, -1.5) node[](leftp2){} to[american potentiometer,mirror, n=pot2, l = $50k\Omega$] ++(2.5, 0) node[](rightp2){};
	\draw (pot2.wiper) to[R, l = $18k\Omega$] ++(0, -1.9) node[ground]{}; 
		
	\draw (botQ2) to[open] (botQ2 |- Q1.B) node[npn, xscale = -1](Q2){};
	\draw (Q2.E) to[short] (rightp2);
	\draw (Q2.B)to[short] ++(0.20, 0) to[R, label = $2.2k\Omega$ ] ++(0, -2) node[ground]{};
	
	\draw (Q1.C) to[short] ++(0, 1) node[](out1){} to[R, l = $10k\Omega$] ++(0, 1.8) to[short] ++(1.25, 0) node[](middle){} to[short] ++(1.25, 0);
	\draw (Q2.C) to[short] ++(0, 0.5) node[](out2){} to[short] ++(0, 0.5) to[R, l = $10k\Omega$] ++(0, 1.8);
	\draw (middle) to[short, -*, l = {$V_{cc}$}] ++(0, 0.5);
	
	\draw (out1) to[open] ++(7, -0.5) node[op amp, label = center:$AO1$](U1){};
	\draw (out1) to[short]  (U1.-);
	
	\draw (U1.+) to[short, -*] (U1.+ -| Q2.C);
	
	\draw (U1.+) to[short] ++(0, -1.5) node[ground]{};
	\draw (U1.-) to[short] ++(0, 1) node[](leftR9){};
	\draw (leftR9) to[vR, l = $5k\Omega$] (leftR9 -| U1.out) to[short] (U1.out);
	\draw (U1.out) to[short, -*] ++(0.5, 0) node[label = right:$V_{out}$](out){};
	
	
\end{circuitikz}
	\caption{Conversor triangular-sinusoidal}
	\label{fig:CTS}
\end{center}
\end{figure}

%%Agregar pequeña explicación
 

\subsection{Calibración}

En total, el circuito posee 4 potenciómetros que deben ajustarse para calibrar correctamente el VCO:


\begin{itemize}
\item La resistencia de realimentación del Schmitt trigger debe ajustarse para lograr el nivel de tension deseado, dados los componentes elegidos. Para eso se ponen 0 Volts a la entrada y se ajusta el preset hasta lograr una señal de $1kHz$.
\item El preset entre los emisores de los trasnsitores en el par diferencial debe ajustarse para lograr mejor simetría en la onda senoidal.
\item El preset a la entrada del conversor se ajusta para suavizar la triangular y lograr una onda senoidal.
\item El preset a la salida del sistema se ajusta para lograr una señal de 1 $Volt$ de amplitud. 
\end{itemize}


\subsection{Simulaciones}

\subsection{Mediciones}

\subsection{Conclusiones}









\end{document}