\documentclass[border={0.5cm 0.5cm 0.5cm 0.5cm}]{standalone}
\usepackage[utf8]{inputenc}
\usepackage[spanish, es-tabla, es-noshorthands]{babel}

\usepackage[a4paper, footnotesep = 1cm, width=18cm, left=2cm, top=2.5cm, height=25cm, textwidth=18cm, textheight=25cm]{geometry}
%\geometry{showframe}

\usepackage{tikz}
\usepackage{textcomp}
\usetikzlibrary{shapes,arrows}

\usepackage{amsmath}
\usepackage{amsfonts}
\usepackage{amssymb}
\usepackage{float}
\usepackage{graphicx}
\usepackage{caption}
\usepackage{subcaption}
\usepackage{multicol}
\usepackage{multirow}
\setlength{\doublerulesep}{\arrayrulewidth}
\usepackage{booktabs}

\usepackage{hyperref}
\hypersetup{
    colorlinks=true,
    linkcolor=blue,
    filecolor=magenta,      
    urlcolor=blue,
    citecolor=blue,    
}

\newcommand{\quotes}[1]{``#1''}
\usepackage{array}
\newcolumntype{C}[1]{>{\centering\let\newline\\\arraybackslash\hspace{0pt}}m{#1}}
\usepackage[american]{circuitikz}
\usepackage{fancyhdr}
\usepackage{units}

% Definition of blocks:
\tikzset{%
  block/.style    = {draw, thick, rectangle, minimum height = 3em,
    minimum width = 3em},
  sum/.style      = {draw, circle, node distance = 2cm}, % Adder
  input/.style    = {coordinate}, % Input
  output/.style   = {coordinate}, % Output
  >=Stealth
}

% Defining string as labels of certain blocks.
\newcommand{\suma}{\Large $\Sigma$}
\newcommand{\inte}{$\displaystyle \int$}
\newcommand{\derv}{\huge $\frac{d}{dt}$}

\begin{document}

% Definition of blocks:
\tikzset{%
  block/.style    = {draw, thick, rectangle, minimum height = 3em,
    minimum width = 3em},
  sum/.style      = {draw, circle, node distance = 2cm}, % Adder
  input/.style    = {coordinate}, % Input
  output/.style   = {coordinate} % Output
}

% Defining string as labels of certain blocks.
\newcommand{\suma}{\Large $\Sigma$}
\newcommand{\inte}{$\displaystyle \int$}
\newcommand{\derv}{\huge $\frac{d}{dt}$}

\begin{tikzpicture}
	\draw node at (0,0)[]{}
	node [input, name=input1, label=above:$V_1$] {$V_i$}
	node at (input1) [sum, right=15mm] (suma1) {\suma}
	node at (suma1) [block, right=15mm] (int1) {$-\int$}
	node at (int1) [right = 15mm] (aux1) {}
	node at (aux1) [block, right=15mm] (wo1) {$\omega_o$}
	node at (aux1) [above = 15mm] (aux2) {}
	node at (aux1) [block, right=15mm] (wo2) {$\omega_o$}
	node at (wo2) [block, below=15mm] (int2) {$-\int$}
	node at (int2) [block, left=15mm] (b-aux) {$-1$}
	node at (suma1) [right=3mm, above=3mm] () {\tiny +}
	node at (suma1) [left=3mm, below=3mm] () {\tiny +}
	node at (suma1) [left=5mm, above=1mm] () {\tiny +};	
	
	\draw[o->] (input1) -- (suma1);
	\draw[->] (suma1) -- (int1);
	\draw[->] (int1) -- (aux1.center);
	\draw[->] (aux1.center) -- (aux2.center);
	\draw[->] (aux2.center) -| (suma1);
	\draw[->] (aux1.center) -- (wo2);
	\draw[->] (wo2) -- (int2);
	\draw[->] (int2) -- (b-aux);
	\draw[->] (b-aux) -| (suma1);
	
\end{tikzpicture}

\end{document}