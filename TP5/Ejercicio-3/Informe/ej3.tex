\documentclass[a4paper]{article}
\usepackage[utf8]{inputenc}
\usepackage[spanish, es-tabla]{babel}

\usepackage{geometry}
 \geometry{includehead, footskip=7mm, headsep=6mm, headheight=4.8mm, top=25mm, bottom=5mm, left=10mm, right=10mm}

%\usepackage[a4paper, 					% Page Layout
%                     %showframe,				% This shows the frame
%                     includehead,
%                     footskip=7mm, headsep=6mm, headheight=4.8mm,
%                     top=25mm, bottom=5mm, left=5mm, right=5mm]{geometry}

\usepackage{amsmath}
\usepackage{amsfonts}
\usepackage{amssymb}

\usepackage{float}
\usepackage{graphicx}
\usepackage{caption}
\usepackage{subcaption}

\usepackage{multirow}
\setlength{\doublerulesep}{\arrayrulewidth}

\newcommand{\quotes}[1]{``#1''}

\usepackage{array}
\newcolumntype{C}[1]{>{\centering\let\newline\\\arraybackslash\hspace{0pt}}m{#1}}

\usepackage[american]{circuitikz}

\usepackage{fancyhdr}

\usepackage{units} 

\pagestyle{fancy}
\fancyhf{}
\lhead{22.01 Teoría de Circuitos}
\rfoot{Página \thepage}
\begin{document}

\subsection{Introducción}

En esta sección se implementó un filtro High-Pass utilizando una aproximación \textbf{Cauer} e implementandola con celdas \textbf{Sedra}, el filtro a diseñar deberá cumplir con la siguiente plantilla.
\begin{table}[H]
\centering
\begin{tabular}{|c|c|}
\hline
$f_s$      & 11.65kHz          \\ \hline
$f_p$      & 23.3kHz           \\ \hline
$A_p$      & 2dB               \\ \hline
$A_s$      & 40dB              \\ \hline
$|Z_{in}|$ & $\geq 50k \Omega$ \\ \hline
\end{tabular}
\end{table}
\subsection{Aproximación de Cauer.}
\subsubsection{Cálculo Analítico}
\subsubsection{Elecciones de diseño}
\subsection{Celda Sedra-Ghorab-Martin.}
\subsubsection{Cálculo Analítico}
\subsubsection{Elecciones de diseño}
\subsection{Respuesta en Frecuencia.}
\subsubsection{Etapas.}
\subsubsection{Filtro definitivo.}
%
%\begin{figure}[H]
%	\centering
%	\includegraphics[width=0.4\textwidth]{/ImagenesEjercicio3/Graph.png}
%	\label{fig:graph}
%\end{figure}
\end{document}