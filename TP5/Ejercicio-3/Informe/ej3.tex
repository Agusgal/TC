\documentclass[a4paper]{article}
\usepackage[utf8]{inputenc}
\usepackage[spanish, es-tabla, es-noshorthands]{babel}
\usepackage[table,xcdraw]{xcolor}
\usepackage[a4paper, footnotesep = 1cm, width=20cm, top=2.5cm, height=25cm, textwidth=18cm, textheight=25cm]{geometry}
%\geometry{showframe}

\usepackage{tikz}
\usepackage{amsmath}
\usepackage{amsfonts}
\usepackage{amssymb}
\usepackage{float}
\usepackage{graphicx}
\usepackage{caption}
\usepackage{subcaption}
\usepackage{multicol}
\usepackage{multirow}
\setlength{\doublerulesep}{\arrayrulewidth}
\usepackage{booktabs}

\usepackage{hyperref}
\hypersetup{
    colorlinks=true,
    linkcolor=blue,
    filecolor=magenta,      
    urlcolor=blue,
    citecolor=blue,    
}

\newcommand{\quotes}[1]{``#1''}
\usepackage{array}
\newcolumntype{C}[1]{>{\centering\let\newline\\\arraybackslash\hspace{0pt}}m{#1}}
\usepackage[american]{circuitikz}
\usetikzlibrary{calc}
\usepackage{fancyhdr}
\usepackage{units} 

\graphicspath{{../Ejercicio-1/}{../Ejercicio-2/}{../Ejercicio-3/}{../Ejercicio-4/}}

\pagestyle{fancy}
\fancyhf{}
\lhead{22.01 Teoría de Circuitos}
\rhead{Mechoulam, Lambertucci, Rodriguez Turco, Londero, Galdeman}
\rfoot{\centering \thepage}
\begin{document}

\subsection{Introducción}

En esta sección se implementó un filtro High-Pass utilizando una aproximación \textbf{Cauer} e implementandola con celdas \textbf{Sedra}, el filtro a diseñar deberá cumplir con la siguiente plantilla.
\begin{table}[H]
\centering
\begin{tabular}{|c|c|}
\hline
$f_s$      & 11.65kHz          \\ \hline
$f_p$      & 23.3kHz           \\ \hline
$A_p$      & 2dB               \\ \hline
$A_s$      & 40dB              \\ \hline
$|Z_{in}|$ & $\geq 50k \Omega$ \\ \hline
\end{tabular}
\end{table}
\subsection{Aproximación de Cauer.}
Para esta sección se utlizó la aproximación elíptica de \textbf{Cauer}.
\subsubsection{Cálculo Analítico}

Para la realización de este filtro lo primero es hacer una transformación de frecuencia High-Pass a Low-Pass y normalizar la plantilla, así obteniendo la siguiente plantilla normalizada.
\begin{center}
	\huge{\textcolor{red}{\textbf{Imagen Plantilla normalizada}}}
\end{center}
A partir de esta plantilla se procedió a calcular el n del filtro bajo la siguiente fórmula:
\begin{center}
	\huge{\textcolor{red}{\textbf{Formula N CAUER}}}
\end{center}
Luego se calcula la función transferencia normalizada, para luego desnormalizar:
\begin{center}
	\huge{\textcolor{red}{\textbf{Transferencia Normalizada}}}
\end{center}
\begin{center}
	\huge{\textcolor{red}{\textbf{Transferencia Desnormalizada}}}
\end{center}
Obteniendo el siguiente diagrama de polos y ceros:
\begin{center}
	\huge{\textcolor{red}{\textbf{Diagrama de polos y ceros}}}
\end{center}
teniendo los pares de polos con un Q de...
\begin{center}
	\huge{\textcolor{red}{\textbf{Hablar del Q de los polos}}}
\end{center}
\subsubsection{Elecciones de diseño}
Se decidió armar etapas con celdas segundo orden en cascada dado a que el orden es 4.
Para la asociación de polos se tomo como criterio agrupar los polos por su cercanía, agrupandolos de las siguiente forma, así mismo la etapa de menor Q será la primera y la de mayor la última.
\subsection{Celda Sedra-Ghorab-Martin.}
La celda Sedra fue propuesta en el paper :....
como una mejora de la ...
Tiene la particularidad de que...
Finalmente en el paper discutido se tomó la configuración de High-Pass-Notch dado a que es lo único que utlizaremos.
\subsubsection{Cálculo Analítico}
---
\subsubsection{Elecciones de diseño}
Realizando el análisis de sensibilidades para los componentes del circuito se puede llegar a la siguiente tabla: 
\begin{center}
	\huge{\textcolor{red}{\textbf{Tabla sensibilidades}}}
\end{center}

Se tomaron como valor de los componentes...
Se eligió utilizar presets para las resistencias... Para ajustar el Q y el $\omega_0$ del filtro.
\subsection{Respuesta en Frecuencia.}
Se realizó un análisis de Montecarlo a la respuesta en frecuencia del circuito, utilizando una tolerancia de las resistencias al 1$\%$ y capacitores al 10$\%$ obteniendo la siguiente disperción,
\subsubsection{Etapas.}
Se realizaron 2 etapas, ambas siendo el mismo tipo de celda, pero con distintos parámetros.
\subsubsection{Filtro definitivo.}
%
%\begin{figure}[H]
%	\centering
%	\includegraphics[width=0.4\textwidth]{/ImagenesEjercicio3/Graph.png}
%	\label{fig:graph}
%\end{figure}
\end{document}