\documentclass[a4paper]{article}
\usepackage[utf8]{inputenc}
\usepackage[spanish, es-tabla]{babel}

\usepackage{geometry}
 \geometry{includehead, footskip=7mm, headsep=6mm, headheight=4.8mm, top=25mm, bottom=5mm, left=10mm, right=10mm}

%\usepackage[a4paper, 					% Page Layout
%                     %showframe,				% This shows the frame
%                     includehead,
%                     footskip=7mm, headsep=6mm, headheight=4.8mm,
%                     top=25mm, bottom=5mm, left=5mm, right=5mm]{geometry}

\usepackage{amsmath}
\usepackage{amsfonts}
\usepackage{amssymb}

\usepackage{float}
\usepackage{graphicx}
\usepackage{caption}
\usepackage{subcaption}

\usepackage{multirow}
\setlength{\doublerulesep}{\arrayrulewidth}

\newcommand{\quotes}[1]{``#1''}

\usepackage{array}
\newcolumntype{C}[1]{>{\centering\let\newline\\\arraybackslash\hspace{0pt}}m{#1}}

\usepackage[american]{circuitikz}

\usepackage{fancyhdr}

\usepackage{units} 

\pagestyle{fancy}
\fancyhf{}
\lhead{22.01 Teoría de Circuitos}
\rfoot{Página \thepage}
\begin{document}

\subsection{Introducción}

En esta sección se implementó un filtro Band-Pass utilizando una aproximación \textbf{Chebycheff} e implementandola con celdas \textbf{Rauch}, el filtro a diseñar deberá cumplir con la siguiente plantilla.
\begin{table}[H]
\centering
\begin{tabular}{|c|c|}
\hline
$Pendiente$      & -40$\frac{dB}{dec}$           \\ \hline
$f_p$      & 28kHz          \\ \hline
$B$      & $\frac{1}{10}$           \\ \hline
$A_p$      & 3dB               \\ \hline
$Filtro$      & BP              \\ \hline
$|Z_{in}|$ & $\geq 50k \Omega$ \\ \hline
\end{tabular}
\end{table}
\subsection{Aproximación de Chebycheff.}
Para esta sección se utlizó la aproximación de \textbf{Chebyfeff}, además se propuso una plantilla mas restrictiva, con el fin de asegurar el cumplimiento de la original. 

Se despejó el valor de $f_p^+$ y $f_p^-$ 
\begin{align}
f_0^2 = f_p^+ \cdot f_p^- \\
B = \frac{\Delta f_p}{f_0}\\
f_p^+ =29.435 kHz  \ \ \ f_p^- = 26.635 kHz
\end{align}
Luego teniendo en cuenta que la pendiente originalmente es de 40dB por decada se tomo la frecuencia de atenuación acorde  talque mantenga las condiciones de simetría, siendo estas: $f_a^+= 294.35kHz y f_a^- = 2.635kHz$.


Siendo esta la plantilla final.
\begin{table}[H]
\centering
\begin{tabular}{|c|c|}
\hline
$f_s^-$      & 2.6635 kHz          \\ \hline
$f_p^-$      & 26.635 kHz         \\ \hline
$f_p^+$      & 29.435 kHz           \\ \hline
$f_s^+$      & 294.35 kHz          \\ \hline
$A_s$      & 40dB           \\ \hline
$A_p$      & 1dB               \\ \hline
\end{tabular}
\end{table}
Obteniendo la siguiente función transferencia:
\begin{align}
	H(s)=\frac{s}{\left( \frac{s}{23728.54}\right) ^2+s\cdot \frac{23728.54}{3.23}+1}\cdot \frac{s}{\left( \frac{s}{33052.25} \right)^2+s\cdot \frac{33052.25}{3.23}+1}
\end{align}
al cual le corresponde la siguiente respues en frecuencia:

Y el siguiente diagrama de polos y ceros:
\begin{figure}[H]
	\centering
	\includegraphics[width=0.5\textwidth]{Imagenes-Ej2/DiagramaPolosYCeros.png}
	\label{fig:stepresponse}
	\caption{Diagrama Polos y Ceros}
\end{figure}

Teniendo los pares de polos conjugados un Q de 3.23	


\subsubsection{Elecciones de diseño}
Se decidió armar etapas con celdas segundo orden en cascada dado a que el orden es 4.
Para la asociación de polos se tomo criterio agrupar cada par de polos con 1 cero , agrupandolos de las siguiente forma.
\begin{figure}[H]
	\centering
	\includegraphics[width=\textwidth]{Imagenes-Ej2/UnionCeros.png}
	\label{fig:CeroPoleUnion}
	\caption{Diagrama Polos y Ceros para cada etapa}
\end{figure}

\subsection{Celda Rauch.}
\subsubsection{Cálculo Analítico.}
El circuito clásico para una celda 	Rauch pasa-banda (Deliyannis-Friend) es el siguiente:
\begin{figure}[H]
	\centering
	\includegraphics[width=0.5\textwidth]{Imagenes-Ej2/rauchoriginal.PNG}
	\label{fig:graph}
	\caption{Circuito clásico celda Rauch Band-Pass(Deliyannis-Friend) .}
\end{figure}
De aquí, planteando los nodos y considerando el amplificador operacional ideal se pueden hallar las siguientes ecuaciones:
\begin{align}
\frac{V_{out}}{R_2}=-V_a\cdot sC_1\\ 
\frac{V_{in}-V_a}{R_1}+(V_{out}-V_a)\cdot sC_2 -V_a \cdot sC_1 = \frac{V_a}{R_3}
\end{align}
De aqui se puede despejar la función transferencia como:
\begin{align}
H(s)=\frac{s \cdot C_1R_2R_3}{s^2\cdot R_1R_2C_2C_1+s\cdot (C_3R_1R_3+R_1C_2R_3)+R_3+R_1}=\frac{H_0 \cdot \frac{s}{\omega_0 Q}}{\frac{s^2}{\omega_0^2}+\frac{s}{\omega_0Q}+1}
\end{align}

De aquí se pueden extraer los parámetros típicos del diseño de filtros siendo los siguientes:
\begin{align}
H_0 =\frac{R_2C_2}{R_1C_2+C_1R_1}\\
\omega_0^2= \frac{R_1+R_3}{R_1^2R_2C_2C_1}\\
Q= \frac{\sqrt[]{1+\frac{R_3}{R_1}}}{{\sqrt[]{\frac{R_3C_1}{R_2C_2}}+\sqrt[]{\frac{R_3C_2}{R_2C_1}}}}
\end{align}
Luego para hacer facil la elección de componentes se suele tomar $C_1 \ = \ C_2$, por culpa de esto para obtener un Q relativamente alto se necesitan valore s de resistencia altos, lo cual es un problema.
Para solucionar dicho problema se hizo una modificación a la celda, incluyendo una realimentación positiva, permitiendo obtener un Q de mayor valor.
La celda Rauch con Q mejorado le corresponde el siguiente circuito.
\begin{figure}[H]
	\centering
	\includegraphics[width=0.5\textwidth]{Imagenes-Ej2/Circuit.PNG}
	\label{fig:graph}
	\caption{Circuito celda Rauch Band-Pass con Q enhancmente.}
\end{figure}
para resolver el circuito basta con plantear los siguientes nodos:
\begin{align}
\frac{V_{out}-K \cdot V_{out}}{R_2}=(K \cdot V_{out}-V_a)\cdot sC_1\\ 
\frac{V_{in}-V_a}{R_1}+(V_{out}-V_a)\cdot sC_2 +(K \cdot V_{out}-V_a) \cdot sC_1 = \frac{V_a}{R_3}
\end{align}
Resolviendo la transferencia se llega a la siguiente expresión:
\begin{align}
H(s)=\frac{1}{1-K} \cdot \frac{H_0 \cdot \frac{s}{\omega_0 Q_0}}{\frac{s^2}{\omega_0^2}+\frac{s}{\omega_0Q}+1}
\end{align}
Donde $Q_0$ es una constante de valor 1.5\footnote{[1]R. Schaumann, H. Xiao, M. Van Valkenburg, M. Van Valkenburg and M. Van Valkenburg, Analog filter design. New York: Oxford University Press, 2011, pp. 140-144.}, y el K es utilizado para ajustar el valor de Q del circuito, dado que quedarán así definidos los parámetros del filtro:
\begin{align}
\omega_0^2= \frac{R_1+R_3}{R_1R_3R_2C_2C_1}\\
Q=\frac{Q_0}{1-2Q_0^2\cdot \frac{K}{1-K}}\\
H_0=\frac{R_3R_2}{2(K-1)R_1R_3+K(R_1+R_3)R_2}
\end{align}
\subsubsection{Elecciones de diseño}
\begin{center}
	\huge{\textcolor{red}{\textbf{Tabla sensibilidades}}}
\end{center}
En base a esta tabla se tomo especial cuidado en la elección de componentes y en el matcheo de impedancias.
Los componentes utilizados fueron los siguientes:
\begin{table}[H]
\centering
\begin{tabular}{lllll}
\multicolumn{1}{c}{Componente} & \multicolumn{1}{c}{1er Etapa} & \multicolumn{1}{c}{Composición} & 2da Etapa      & Composición           \\ \hline
$R_1$                          & $7.3 k\Omega$                 & $10k // 27k  \Omega$            & $5.24 k\Omega$ & $5.6k // 82k  \Omega$ \\
$R_2$                          & $5.56 k\Omega$                & $5.6k // 680k  \Omega$          & $3.99 k\Omega$ & $82 + 3.9k  \Omega$   \\
$R_3$                          & $1.43 k\Omega$                & $1.5 k // 33k  \Omega$          & $1.03k\Omega$  & $27 + 1k  \Omega$     \\
$R_4$                          & $3.49 k\Omega$                & $3.9k // 33k  \Omega$           & $3.49 k\Omega$ & $3.9k // 33k  \Omega$ \\
$R_5$                          & $1 k\Omega$                   & $1 k  \Omega$                   & $1 k\Omega$    & $1 k\Omega$           \\
$C_1$                          & 2.2 nF                        & 2.2 nF                          & 2.35 nF         & (4.7+4.7) nF                \\
$C_2$                          & 2.2 nF                        & 2.2 nF                          & 2.35 nF         & (4.7+4.7) nF               
\end{tabular}
\end{table}

Se calculó el error porcentual asociado a la aproximación de la resistencias viendose en la siguiente tabla.
\begin{table}[H]
\centering
\begin{tabular}{lll}
\multicolumn{1}{c}{Error Porcentual} & \multicolumn{1}{c}{1er Etapa} & \multicolumn{1}{c}{2da Etapa} \\ \hline
$R_1$                                & 0.1 $\%$                      & $0.038  \%$                   \\
$R_2$                                & 0.1 $\%$                      & 0.2 $\%$                      \\
$R_3$                                & 0.4 $\%$                      & 0.1 $\%$                      \\
$R_4$                                & 0.1 $\%$                      & 0.1 $\%$                      \\
$R_5$                                & $\approx 0 \%$                & $\approx 0 \%$                \\
$C_1$                                & $\approx 0 \%$                & $\approx 0 \%$                \\
$C_1$                                & $\approx 0 \%$                & $\approx 0 \%$               
\end{tabular}
\end{table}

Cabe destacar que todas las imepdancias que fueron colocadas en el circuito fueron elegidas entre varias de su mismo tipo, con la finalidad de poner impedancias que sean realmente de los valores deseados.

\subsubsection{Acoplamiento de Impedancias.}

Para que ambas etapas no se carguen entre si la impedancia de entrada de la segunda etapa debe ser mucho mayor a la de salida de la primera, para lo siguiente se obtuvieron las impedancias de entrada de ambas celdas, incluyendo la de salida de la primera.
\begin{figure}[H]
	\centering
	\includegraphics[width=\textwidth]{Imagenes-Ej2/ZinE1.png}
	\label{fig:graph}
	\caption{Impedancia de entrada 1er etapa.}
\end{figure}

\begin{figure}[H]
	\centering
	\includegraphics[width=\textwidth]{Imagenes-Ej2/ZoutE1.png}
	\label{fig:graph}
	\caption{Impedancia de salida 1er etapa.}
\end{figure}


\begin{figure}[H]
	\centering
	\includegraphics[width=\textwidth]{Imagenes-Ej2/ZinE2.png}
	\label{fig:graph}
	\caption{Impedancia de entrada 2da etapa.}
\end{figure}

\subsection{Respuesta en Frecuencia.}
Se realizó un análisis de Montecarlo a la respuesta en frecuencia del circuito, utilizando una tolerancia de las resistencias al 1$\%$ y capacitores al 10$\%$ obteniendo la siguiente disperción.
%\begin{figure}[H]
%	\centering
%	\includegraphics[width=0.4\textwidth]{/ImagenesEjercicio3/Graph.png}
%	\label{fig:graph}
%\end{figure}
También se midió la respuesta en frecuencia del filtro y se cotejó con la simulación.
\begin{figure}[H]
	\centering
	\includegraphics[width=\textwidth]{Imagenes-Ej2/BodeRauch.png}
	\label{fig:graph}
\end{figure}
\begin{figure}[H]
	\centering
	\includegraphics[width=0.5\textwidth]{Imagenes-Ej2/BodeRauchFase.png}
	\label{fig:graph}
\end{figure}

\subsubsection{Etapas.}
Se realizaron 2 etapas, ambas siendo el mismo tipo de celda, pero con distintos parámetros.
\subsubsection{Filtro definitivo.}
%
%\begin{figure}[H]
%	\centering
%	\includegraphics[width=0.4\textwidth]{/ImagenesEjercicio3/Graph.png}
%	\label{fig:graph}
%\end{figure}

\end{document}