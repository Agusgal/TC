\documentclass[a4paper]{article}
\usepackage[utf8]{inputenc}
\usepackage[spanish, es-tabla]{babel}

\usepackage{geometry}
 \geometry{includehead, footskip=7mm, headsep=6mm, headheight=4.8mm, top=25mm, bottom=5mm, left=10mm, right=10mm}

%\usepackage[a4paper, 					% Page Layout
%                     %showframe,				% This shows the frame
%                     includehead,
%                     footskip=7mm, headsep=6mm, headheight=4.8mm,
%                     top=25mm, bottom=5mm, left=5mm, right=5mm]{geometry}

\usepackage{amsmath}
\usepackage{amsfonts}
\usepackage{amssymb}

\usepackage{float}
\usepackage{graphicx}
\usepackage{caption}
\usepackage{subcaption}

\usepackage{multirow}
\setlength{\doublerulesep}{\arrayrulewidth}

\newcommand{\quotes}[1]{``#1''}

\usepackage{array}
\newcolumntype{C}[1]{>{\centering\let\newline\\\arraybackslash\hspace{0pt}}m{#1}}

\usepackage[american]{circuitikz}

\usepackage{fancyhdr}

\usepackage{units} 

\pagestyle{fancy}
\fancyhf{}
\lhead{22.01 Teoría de Circuitos}
\rfoot{Página \thepage}

\begin{document}

\subsection{Datasheet}

\begin{center}
\rule{\textwidth}{1pt}
\textsc{Control de Tonos y Ecualizador TCG3 \textsuperscript{\textregistered}}
\rule{\textwidth}{1pt}
\end{center}

\begin{multicols}{2}

\begin{enumerate}
	\item[1] \textbf{Características}
	\begin{itemize}
		\item Sistema de control de tonos con 3 grados de libertad.
		\item Sistema de amplificación y atenuación sobre la totalidad del espectro audible.
		\item Entrada y salida de audio compatible con conectores de $3.5 \ mm$ mono.
		\item Amplificación y atenuación de hasta $\pm 15 \ db$.
		\item Bajo ruido de entrada.
		\item Cobertura total del espectro audible.
	\end{itemize}
	
	\item[2] \textbf{Descripción}\\
		El \textsc{Control de Tonos y Ecualizador TCG3~\textsuperscript{\textregistered}} es un circuito que permite amplificar y atenuar frecuencias bajas, medias y altas con total independencia entre sí. Posee 3 frecuencias centrales que permiten dicho control. Además cuenta con un regulador de audio, que permite amplificar y atenuar en su totalidad todo el conjunto de frecuencias audibles. El dispositivo requiere una fuente de alimentación para los amplificadores operacionales ($\pm 18 \ V$). Posee entrada y salida audio mono. A su vez cuenta con entradas que permiten de cualquier tipo de señal, ya sea para análisis de esta o calibración del mismo dispositivo.
	
	\item[3] \textbf{Alimentación}
	\begin{table}[H]
		\begin{tabular}{ccccc}
			\hline	
			Tensión & Min & Sugerido & Max & Unidad \\
			\hline
			$V_{in}$\footnote{Valores de tensión dados en amplitud pico-pico.}    & 0 	& 1.5		   & 2	 	& V \\
			$Vcc$       & 10  	& 15       & 18 	& V \\
			$-Vcc$      & -10 	& -15      & -18 	& V	\\
			\hline
		\end{tabular}
	\end{table}
		
	\item[4] \textbf{Valores de interés}
	\begin{table}[H]
		\begin{tabular}{ccccc}
			\hline			
			\textbf{Banda} & $\mathbf{f_o}$ & $\mathbf{|Z_{in}|}$ & $\mathbf{Arg\left(Z_{in}\right)}$ & \textbf{Ganancia} \\
			\hline
			Baja           & 81.1 Hz                     & 30.38 $k\Omega$     & $137.45^o$                           & $\pm 12.7 \ dB$          \\
			Media          & 1.2 kHz                     & 10.17 $k\Omega$      & $169.26^o$                           & $\pm 13.7 \ dB$          \\
			Alta           & 8.9 kHz                       & 9.92 $k\Omega$      & $178.57^o$                            & $\pm 15 \ dB$         	\\
			\hline
		\end{tabular}
	\end{table}

\end{enumerate}
\end{multicols}

\begin{enumerate}
	\item[5] \textbf{Esquemático}\\
\end{enumerate}

\begin{multicols}{2}
\centering
	\begin{figure}[H]
		\includegraphics[width=0.5\textwidth]{Imagenes/Schematic-1.png}
	\end{figure}
	\begin{figure}[H]
			\includegraphics[width=0.5\textwidth]{Imagenes/Schematic-2.png}
	\end{figure}
\end{multicols}

\begin{enumerate}
	\item[6] \textbf{Conexiones de entrada y salida}\\
\end{enumerate}
\begin{multicols}{2}
	\begin{figure}[H]
		\includegraphics[width=0.5\textwidth]{Imagenes/Input.png}
	\end{figure}
	\begin{figure}[H]
			\includegraphics[width=0.3\textwidth]{Imagenes/Output.png}
	\end{figure}
\end{multicols}
 

\end{document}