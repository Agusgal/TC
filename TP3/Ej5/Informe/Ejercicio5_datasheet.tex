\documentclass[a4paper]{article}

\documentclass[a4paper]{article}
\usepackage[utf8]{inputenc}
\usepackage[spanish, es-tabla, es-noshorthands]{babel}
\usepackage[table,xcdraw]{xcolor}
\usepackage[a4paper, footnotesep = 1cm, width=20cm, top=2.5cm, height=25cm, textwidth=18cm, textheight=25cm]{geometry}
%\geometry{showframe}

\usepackage{tikz}
\usepackage{amsmath}
\usepackage{amsfonts}
\usepackage{amssymb}
\usepackage{float}
\usepackage{graphicx}
\usepackage{caption}
\usepackage{subcaption}
\usepackage{multicol}
\usepackage{multirow}
\setlength{\doublerulesep}{\arrayrulewidth}
\usepackage{booktabs}

\usepackage{hyperref}
\hypersetup{
    colorlinks=true,
    linkcolor=blue,
    filecolor=magenta,      
    urlcolor=blue,
    citecolor=blue,    
}

\newcommand{\quotes}[1]{``#1''}
\usepackage{array}
\newcolumntype{C}[1]{>{\centering\let\newline\\\arraybackslash\hspace{0pt}}m{#1}}
\usepackage[american]{circuitikz}
\usetikzlibrary{calc}
\usepackage{fancyhdr}
\usepackage{units} 

\graphicspath{{../Ejercicio-1/}{../Ejercicio-2/}{../Ejercicio-3/}{../Ejercicio-4/}}

\pagestyle{fancy}
\fancyhf{}
\lhead{22.01 Teoría de Circuitos}
\rhead{Mechoulam, Lambertucci, Rodriguez Turco, Londero, Galdeman}
\rfoot{\centering \thepage}

\usepackage{float}
\usepackage{graphicx}

\usepackage[american voltage]{circuitikz}

\usepackage{amsmath}

\usepackage{xcolor}

\usepackage{caption}
\usepackage{subcaption}

\usepackage{multicol}

\begin{document}


\subsection{Datasheet}
\begin{center}
\rule{\textwidth}{1pt}
\textsc{Control de Tonos y Ecualizador TCG3 \textsuperscript{\textregistered}}
\rule{\textwidth}{1pt}
\end{center}

\begin{multicols}{2}

\begin{enumerate}
	\item[1] \textbf{Características}
	\begin{itemize}
		\item Sistema de control de tonos con 3 grados de libertad.
		\item Sistema de amplificación y atenuación sobre la totalidad del espectro audible.
		\item Entrada y salida de audio compatible con conectores de $3.5 \ mm$ mono.
		\item Amplificación y atenuación de hasta $\pm 15 \ db$.
	\end{itemize}
	
	\item[2] \textbf{Descripción}\\
		El \textsc{Control de Tonos y Ecualizador TCG3~\textsuperscript{\textregistered}} es un circuito que permite amplificar y atenuar frecuencias bajas, medias y altas con total independencia entre sí. Posee 3 frecuencias centrales que permiten dicho control. Además cuenta con un regulador de audio, que permite amplificar y atenuar en su totalidad todo el conjunto de frecuencias audibles. El dispositivo requiere una fuente de alimentación para los amplificadores operacionales ($\pm 18 \ V$). Posee entrada y salida audio mono. A su vez cuenta con entradas que permiten de cualquier tipo de señal, ya sea para análisis de esta o calibración del mismo dispositivo.
	
	\item[3] \textbf{Alimentación}\\
	\begin{table}[H]
		\begin{tabular}{ccccc}
			\hline	
			Tensión & Min & Sugerido & Max & Unidad \\
			\hline
			$V_{in}$    & 0 	& 2		   & 2.75	 	& Vpp \\
			$Vcc$       & 10  	& 15       & 18 	& V \\
			$-Vcc$      & -10 	& -15      & -18 	& V	\\
			\hline
		\end{tabular}
	\end{table}
		
	\item[4] \textbf{Valores de interés}\\
		

\end{enumerate}
\end{multicols}

\begin{enumerate}
		\item[5] \textbf{Modelo equivalente}\\
	
		\begin{figure}[H]
		\begin{center}
		\begin{circuitikz}
		\draw	(0,0) to[short] ++(1,0) to[short, -o] ++ (0,0.5)node[label=north:$Vcc$]{};
		\draw	(0,0) to[open] ++(1,0) to[short] ++ (1,0) to[short] ++(0,-0.5) to[short, -o] ++(0.5,0)node[label=south:$V_{out}$]{};
		\draw	(0,0) to[open] ++(2,0) to[open] ++(0,-0.5) to[short] ++(0,-0.75) to[short, -o] ++(0.5,0);
		\draw	(0,0) to[open] ++(2,0) to[open] ++(0,-0.5) to[open] ++(0,-0.75) to[short] ++(0,-0.5) to[short] ++(-1,0) to[short, -o] ++(0,-0.5)node[label=south:$-Vcc$]{};
		\draw	(0,0) to[short] ++(0,-0.5) to[short, -o] ++(-0.5,0)node[label=south:$V_{s}$]{};
		\draw	(0,0) to[open] ++(0,-0.5) to[short] ++(0,-0.75) to[short, -o] ++(-0.5,0);
		\draw	(0,0) to[open] ++(0,-1.25) to[short] ++(0,-0.5) to[short] ++(1,0);
		\draw	(0,0) to[open] ++(1,-1.25)node[label=$TCG3$]{};	
		\end{circuitikz}
		\end{center}
		\end{figure}
\end{enumerate}

\end{document}