\documentclass[a4paper]{article}

\documentclass[a4paper]{article}
\usepackage[utf8]{inputenc}
\usepackage[spanish, es-tabla, es-noshorthands]{babel}
\usepackage[table,xcdraw]{xcolor}
\usepackage[a4paper, footnotesep = 1cm, width=20cm, top=2.5cm, height=25cm, textwidth=18cm, textheight=25cm]{geometry}
%\geometry{showframe}

\usepackage{tikz}
\usepackage{amsmath}
\usepackage{amsfonts}
\usepackage{amssymb}
\usepackage{float}
\usepackage{graphicx}
\usepackage{caption}
\usepackage{subcaption}
\usepackage{multicol}
\usepackage{multirow}
\setlength{\doublerulesep}{\arrayrulewidth}
\usepackage{booktabs}

\usepackage{hyperref}
\hypersetup{
    colorlinks=true,
    linkcolor=blue,
    filecolor=magenta,      
    urlcolor=blue,
    citecolor=blue,    
}

\newcommand{\quotes}[1]{``#1''}
\usepackage{array}
\newcolumntype{C}[1]{>{\centering\let\newline\\\arraybackslash\hspace{0pt}}m{#1}}
\usepackage[american]{circuitikz}
\usetikzlibrary{calc}
\usepackage{fancyhdr}
\usepackage{units} 

\graphicspath{{../Ejercicio-1/}{../Ejercicio-2/}{../Ejercicio-3/}{../Ejercicio-4/}}

\pagestyle{fancy}
\fancyhf{}
\lhead{22.01 Teoría de Circuitos}
\rhead{Mechoulam, Lambertucci, Rodriguez Turco, Londero, Galdeman}
\rfoot{\centering \thepage}

\usepackage{float}
\usepackage{graphicx}

\usepackage[american voltage]{circuitikz}

\usepackage{amsmath}

\usepackage{xcolor}

\usepackage{caption}
\usepackage{subcaption}

\begin{document}

\begin{figure}[H]
\begin{center}
\begin{circuitikz}
	\node [op amp](A){};
	\draw (A.+) to[short] ++(-0.5,0) to[short] ++(0,-1) node[ground]{};
	\draw (A.-) to[short] ++(-0.5,0) to[short] ++(0,1.5) node[](v1){} to[R, l = $R_3$] ++(-3.5,0) to[short] ++(0,3);
	\draw (v1) to[R, l_= $R_3$] ++(3.5,0) to[short] ++(0,3);
	\draw (v1) to[open] ++(0,3) node[potentiometershape, rotate=180, label=south:$R_2$](P){};
	\draw (v1) to[C, l = $C_2$] (P.wiper);
	\draw (P.right) to[R, l= $R_1$] ++(-4,0) node[ocirc,label=left:$V_{in}$]{};
	\draw (P.left) to[R, l_= $R_1$] ++(4,0) node[ocirc,label=right:$V_{out}$]{};
	\draw (P.right) to[open] ++(-0.5,0) to[short] ++(0,1.5) to[C, l = $C_1$] ++(2.2,0) to[short] ++(0,-1.5);
	\draw (A.out) to[short] ++(0.62,0) to[short] ++(0,2);
\end{circuitikz}
	\caption{Circuito dado.}
	%\label{}
\end{center}
\end{figure}

$R_2 = R_{2}^{'} + R_{2}^{''} $, con
\begin{equation}
\begin{split}
	R_{2}^{'} &= \psi R_2\\ R_{2}^{''} &= \left( 1 - \psi \right) R_2
\end{split}
\label{equ:r2p}
\end{equation}


\begin{figure}[H]
\begin{center}
\begin{circuitikz}
	\node [op amp](A){};
	\draw (A.+) to[short] ++(-0.5,0) to[short] ++(0,-1) node[ground]{};
	\draw (A.-) to[short] ++(-0.5,0) to[short] ++(0,1.5) node[](v1){} to[R, l = $R_3$] ++(-3.5,0) to[short] ++(0,3);
	\draw (v1) to[R, l_= $R_3$] ++(3.5,0) to[short] ++(0,3);
	\draw (v1) to[C, l = $C_2$] ++(0,1.5) node[](v2){};
	\draw[color=red] (v2) to[short] ++(-1,0) to[R, l= $R_{2}^{'}$, color = red] ++(0,1.5) node[](v3){};
	\draw[color=red] (v2) to[short] ++(1,0) to[R, l_= $R_{2}^{''}$, color = red] ++(0,1.5) node[](v4){};
	\draw[color=red] (v3) to[C, l = $C_1$, color = red] (v4);
	\draw (v3) to[R, l= $R_1$] ++(-3,0) node[ocirc,label=left:$V_{in}$]{};
	\draw (v4) to[R, l_= $R_1$] ++(3,0) node[ocirc,label=right:$V_{out}$]{};
	\draw (A.out) to[short] ++(0.62,0) to[short] ++(0,2);
\end{circuitikz}
	\caption{Reemplazando $R_2$ por resistencias fijas.}
	%\label{}
\end{center}
\end{figure}

Aplicando Kennelly para transformar de configuración Pi a Estrella:

\begin{figure}[H]
\begin{center}
\begin{circuitikz}
	\node [op amp](A){};
	\draw (A.+) to[short] ++(-0.5,0) to[short] ++(0,-1) node[ground]{};
	\draw (A.-) to[short] ++(-0.5,0) to[short] ++(0,1.5) node[](v1){} to[R, l = $R_3$] ++(-3.5,0) to[short] ++(0,3);
	\draw (v1) to[R, l_= $R_3$] ++(3.5,0) to[short] ++(0,3);
	\draw[color=red] (v1) to[C, l = $C_2$] ++(0,1.5) to[generic, l = $Z_C$] ++(0,1.5) node[](v2){};
	\draw[color=red] (v2) to[generic, l_= $Z_A$] ++(-2,0) to[R, l= $R_1$] ++(-1.5,0) node[](aux2){};
	\draw (aux2) to[short] ++(-1,0) node[ocirc,label=left:$V_{in}$]{};
	\draw[color=red] (v2) to[generic, l = $Z_B$] ++(2,0) to[R, l_= $R_1$] ++(1.5,0) node[](aux1){};
	\draw (aux1) to[short] ++(1,0) node[ocirc,label=right:$V_{out}$]{};
	\draw (A.out) to[short] ++(0.62,0) to[short] ++(0,2);
\end{circuitikz}
	\caption{Usando Kennelly por primera vez.}
	%\label{}
\end{center}
\end{figure}

Siendo:

\begin{equation*}
	Z_{A} = \frac{R_{2}^{'}}{C_{1} R_{2} S + 1}
\end{equation*}	
\begin{equation*}
	Z_{B} = \frac{R_{2}^{''}}{C_{1} R_{2} S + 1}
\end{equation*}	
\begin{equation*}
	Z_{C} = \frac{C_{1} R_{2}^{''} R_{2}^{'} S}{C_{1} R_{2} S + 1}
\end{equation*}

Tomando $Z_{A} + R_1$, $Z_{B} + R_1$ y $Z_{C} + \frac{1}{S C_2}$ y aplicando Kennelly nuevamente, se obtiene: 

\begin{figure}[H]
\begin{center}
\begin{circuitikz}
	\node [op amp](A){};
	\draw (A.+) to[short] ++(-0.5,0) to[short] ++(0,-1) node[ground]{};
	\draw (A.-) to[short] ++(-0.5,0) to[short] ++(0,1.5) node[](v1){} to[short] ++(-1.5,0) node[](aux1){};
	\draw[color=red] (aux1) to[short] ++(-2,0) to[R, l = $R_3$] ++(0,3);
	\draw[color=red] (aux1) to[generic, l = $Z_{AC}$] ++(0,3) node[](v2){};

	\draw (v1) to[short] ++(1.5,0) node[](aux2){};
	\draw[color=red] (aux2) to[short] ++(2,0) to[R, l_= $R_3$] ++(0,3);
	\draw[color=red] (aux2) to[generic, l_= $Z_{BC}$] ++(0,3) node[](v3){};
	
	
	\draw (v2) to[generic, l = $Z_{AB}$] (v3);
	\draw[color=red] (v3) to[short] ++(2,0) node[](aux3){};
	\draw (aux3) to[short] ++(1,0) node[ocirc,label=right:$V_{out}$]{};
	
	\draw[color=red] (v2) to[short] ++(-2,0) node[](aux4){};
	\draw (aux4) to[short] ++(-1,0)node[ocirc,label=left:$V_{in}$]{};
	\draw (A.out) to[short] ++(0.62,0) to[short] ++(0,2);
\end{circuitikz}
	\caption{Usando Kennelly por segunda vez.}
	%\label{}
\end{center}
\end{figure}

\begin{equation*}
	Z_{AC} =  \frac{\alpha_{AC} {C_{1}}^{2} C_2 R_{1} R_{2} S^{3} + \beta_{AC} C_1 S^{2}
		+ \gamma_{AC} S + 2 R_1 + R_{2}^{'} + R_{2}^{''}}{
		 C_2 S \left( C_1 R_{2} S + 1 \right)
		\left(C_1 R_1 R_{2} S + R_1 + R_{2}^{''}\right)}
\end{equation*}

con
$$
	\alpha_{AC} = R_{1} R_{2} + 2 {C_{1}}^{2} C_2 R_{2}^{'} R_{2}^{''}
$$

\begin{equation*}
\begin{split}
	\beta_{AC} =\ & 2 C_{1} R_1 R_{2}^{2} + 2 C_2 R_{1}^{2} R_{2} + C_2 R_1 R_{2} R_{2}^{'} +
		C_2 R_1 R_{2} R_{2}^{''} +\\
		&2 C_2 R_1 R_{2}^{'} R_{2}^{''} +
		C_2 R_{2}^{'2} R_{2}^{''} + C_2 R_{2}^{'} R_{2}^{''2}
\end{split}
\end{equation*}

\begin{equation*}
\begin{split}
	\gamma_{AC} =\ & 4 C_1 R_1 R_{2} + C_1 R_{2} R_{2}^{'} + C_1 R_{2} R_{2}^{''} +
		C_2 {R_{1}}^{2} +\\
		&C_2 R_1 R_{2}^{'} + C_2 R_1 R_{2}^{''} + C_2 R_{2}^{'} R_{2}^{''}
\end{split}
\end{equation*}

\begin{equation*}
	Z_{AB} = \frac{3 C_1 R_1 R_{2} S + 3 R_1 + 2 R_{2}^{'} + R_{2}^{''}}{C_1 R_{2} S + 1}
\end{equation*}

\begin{equation*}
	Z_{BC} =
	\frac{ \left( R_1 R_{2} + 2 R_{2}^{'} R_{2}^{''} \right ){C_{1}}^{2} C_2 R_1 R_{2} S^{3} +
	\beta_{BC} C_1 S^{2} + \gamma_{BC} S + 2 R_1 + R_{2}^{'} + R_{2}^{''}}
	{C_2 S \left(C_1 R_{2} S + 1\right) \left(C_1 R_1 R_{2} S + R_1 + R_{2}^{'} \right)}
\end{equation*}
con
\begin{equation*}
\begin{split}
	\beta_{BC} =\ & 2 C_1 R_1 {R_{2}}^{2} + 2C_2 {R_{1}}^{2} R_{2} + C_2 R_1 R_{2} R_{2}^{'} +\\
	&C_2 R_1 R_{2} R_{2}^{''} + 2 C_2 R_1 R_{2}^{'} R_{2}^{''} + 
	C_2 {R_{2}^{'}}^2 R_{2}^{''} + C_2 R_{2}^{'} R_{2}^{''2}
\end{split}
\end{equation*}

\begin{equation*}
\begin{split}
	\gamma_{BC} =  4 C_1 R_1 R_{2} + C_1 R_{2} R_{2}^{'} + C_1 R_{2} R_{2}^{''} +
	C_2 {R_{1}}^{2} + C_2 R_1 R_{2}^{'} + C_2 R_1 R_{2}^{''} +
	C_2 R_{2}^{'} R_{2}^{''} 
\end{split}
\end{equation*}

Definiendo: $Z_{AC}^{'} = Z_{AC} // R_3 $ y $Z_{BC}^{'} = Z_{AC} // R_3 $

\begin{figure}[H]
\begin{center}
\begin{circuitikz}
	\node [op amp](A){};
	\draw (A.-) to[short] ++(-1,0) to[generic, l = $Z_{AC}^{'}$] ++(-2,0) node[](v1){} to[short, -o] node[ocirc,label=left:$V_{in}$]{} ++(-1,0);
	\draw (v1) to[short] ++(0,2.5) to[generic, l = $Z_{AB}$] ++(6.5,0) to[short] ++(0,-3);
	\draw (A.+) to[short] ++(-0.5,0) to[short] ++(0,-1) node[ground]{};
	\draw (A.-) to[short] ++(-0.5,0) to[short] ++(0,1.5) to[generic, l_= $Z_{BC}^{'}$] ++(3,0) to[short] ++(0,-2);
	\draw (A.out) to[short, -o] ++(2,0) node[ocirc,label=right:$V_{out}$]{};
\end{circuitikz}
	\caption{Usando impedancias en paralelo.}
	%\label{}
\end{center}
\end{figure}

\begin{figure}[H]
\begin{center}
\begin{circuitikz}
	\node [op amp](A){};
	\draw (A.-) to[short] ++(-1,0) to[generic, l_= $Z_{AC}^{'}$] ++(-2,0) node[](v1){} to[short, -o] node[ocirc,label=left:$V_{in}$]{} ++(-1,0);
	\draw (v1) to[generic, l = $\frac{Z_{AB}}{1 - k}$] ++(0,-2) node[ground]{};
	\draw (A.out) ++(0.5,0) to[generic, l = $\frac{Z_{AB}}{1 - \frac{1}{k}}$] ++(0,-1.5) node[ground]{};
	\draw (A.+) to[short] ++(-0.5,0) to[short] ++(0,-1) node[ground]{};
	\draw (A.-) to[short] ++(-0.5,0) to[short] ++(0,1.5) to[generic, l= $Z_{BC}^{'}$] ++(3,0) to[short] ++(0,-2);
	\draw (A.out) to[short, -o] ++(2,0) node[ocirc,label=right:$V_{out}$]{};
\end{circuitikz}
	\caption{Modelo final, aplicando Miller.}
	%\label{}
\end{center}
\end{figure}

Observando, y utilizando (\ref{equ:r2p}), se puede resolver el circuito con las condiciones de un amplificador operacional inversor:
\begin{equation}
	H(s) = - \frac{\alpha_H S^{2} + \beta_H S + 2 R_{1} + R_{2}}
	{\gamma_H S^{2} + \delta_H S + 2 R_{1} + R_{2}}
\end{equation}

Siendo: 

\begin{equation*}
\begin{split}
	\alpha_H =\ & C_{1} C_{2} {R_{1}}^{2} R_{2} - 2 C_{1} C_{2} R_{1} {R_{2}}^{2} \psi^{2} + 2 C_{1} C_{2} R_{1} {R_{2}}^{2} \psi + C_{1} C_{2} R_{1} R_{2} R_{3}
\end{split}
\end{equation*}
\begin{equation*}
\begin{split}
	\beta_H =\ & 2 C_{1} R_{1} R_{2} + C_{2} {R_{1}}^{2} + C_{2} R_{1} R_{2} + C_{2} R_{1} R_{3} -\\ & C_{2} {R_{2}}^{2} \psi^{2} + C_{2} {R_{2}}^{2} \psi - C_{2} R_{2} R_{3} \psi + C_{2} R_{2} R_{3}
\end{split}
\end{equation*}
\begin{equation*}
\begin{split}
	\gamma_H =\ & C_{1} C_{2} {R_{1}}^{2} R_{2} - 2 C_{1} C_{2} R_{1} {R_{2}}^{2} \psi^{2} + 2 C_{1} C_{2} R_{1} {R_{2}}^{2} \psi + C_{1} C_{2} R_{1} R_{2} R_{3}
\end{split}
\end{equation*}
\begin{equation*}
\begin{split}
 \delta_H =\ & 2 C_{1} R_{1} R_{2} + C_{2} {R_{1}}^{2} + C_{2} R_{1} R_{2} + C_{2} R_{1} R_{3} - C_{2} {R_{2}}^{2} \psi^{2} + C_{2} {R_{2}}^{2} \psi + C_{2} R_{2} R_{3} \psi
\end{split}
\end{equation*}
\\
\\
\\

\begin{center}
	\textcolor{red}{\textbf{PONER EN EL INFORME SIMPLIFICACIÓN FINAL.}}
\end{center}

Si se considera $C_1 = 10 \ C_2$, $R_3 \gg R_1$ y $R_3 = 10 \ R_2$ se obtiene

\begin{equation*}
\begin{split}
	\alpha_H =\ \gamma_H =\ & 10 {C_{2}}^{2} {R_{2}}^{2} R_{1} \left[ 10 + 2 \psi \left(1 - \psi \right) \right]
\end{split}
\end{equation*}
\begin{equation*}
\begin{split}
	\beta_H =\ & C_{2} R_{2} \left[ 10R_1 + R_2 \left(10 - \psi - 9 \psi^2 \right) \right]
\end{split}
\end{equation*}
\begin{equation*}
\begin{split}
	\gamma_H =\ & 10 {C_{2}}^{2} {R_{1}}^{2} R_{2} - 20 {C_{2}}^{2} R_{1} {R_{2}}^{2} \psi^{2} + 20 {C_{2}}^{2} R_{1} {R_{2}}^{2} \psi + 10 {C_{2}}^{2} R_{1} R_{2} R_{3}
\end{split}
\end{equation*}
\begin{equation*}
\begin{split}
 \delta_H =\ & 2 C_{2} R_{2} \left[ 31 R_1 + R_2 \psi \left(11 - \psi \right) \right]
\end{split}
\end{equation*}

\end{document}
