\documentclass[a4paper]{article}

\documentclass[a4paper]{article}
\usepackage[utf8]{inputenc}
\usepackage[spanish, es-tabla]{babel}

\usepackage{geometry}
 \geometry{includehead, footskip=7mm, headsep=6mm, headheight=4.8mm, top=25mm, bottom=5mm, left=10mm, right=10mm}

%\usepackage[a4paper, 					% Page Layout
%                     %showframe,				% This shows the frame
%                     includehead,
%                     footskip=7mm, headsep=6mm, headheight=4.8mm,
%                     top=25mm, bottom=5mm, left=5mm, right=5mm]{geometry}

\usepackage{amsmath}
\usepackage{amsfonts}
\usepackage{amssymb}

\usepackage{float}
\usepackage{graphicx}
\usepackage{caption}
\usepackage{subcaption}

\usepackage{multirow}
\setlength{\doublerulesep}{\arrayrulewidth}

\newcommand{\quotes}[1]{``#1''}

\usepackage{array}
\newcolumntype{C}[1]{>{\centering\let\newline\\\arraybackslash\hspace{0pt}}m{#1}}

\usepackage[american]{circuitikz}

\usepackage{fancyhdr}

\usepackage{units} 

\pagestyle{fancy}
\fancyhf{}
\lhead{22.01 Teoría de Circuitos}
\rfoot{Página \thepage}

\usepackage{float}
\usepackage{graphicx}

\usepackage[american voltage]{circuitikz}

\usepackage{amsmath}

\usepackage{xcolor}

\usepackage{caption}
\usepackage{subcaption}

\begin{document}

\subsection{Fisiología del oído}

%El sistema auditivo humano posee un órgano periférico conductor de los estímulos externos, como lo son el oído externo y medio, un órgano que transforma dichas señales, como lo es el oído interno, un nervio que transporta las excitaciones, en forma de impulsos eléctricos, al sistema nervioso central y una corteza cerebral que hace conscientes los estímulos para el humano.
La fisiología auditiva puede ser dividida en tres estructuras:
\begin{itemize}
	\item Oído externo:
	Este conforma parte del órgano periférico encargado de la conducción de estímulos externos, transmitiendolos hacia el conducto auditivo externo. Las paredes rígidas de este último evitan que el sonido sea absorbido por por los demás tejidos blandos.
	\item Oído medio:
	Compuesto por el sistema timpanoosicular (membrana timpánica y los tres huesillos: martillo, yunque y estribo), cumple con la función tanto de transmitir las señales del ambiente, como de proteger las estructuras neurosensoriales del oído interno y de la ventana redonda.
	\item Oído interno:
	
\end{itemize}


\subsection{Funcionamiento de un ecualizador}

\subsection{Desarrollo del circuito y selección de componentes}

El circuito brindado por la cátedra, mostrado en la Figura (\ref{fig:circuitoini}), es una instancia del equalizador. Se procede a analizar la transferencia de este.

\begin{figure}[H]
\begin{center}
\begin{circuitikz}
	\node [op amp](A){};
	\draw (A.+) to[short] ++(-0.5,0) to[short] ++(0,-1) node[ground]{};
	\draw (A.-) to[short] ++(-0.5,0) to[short] ++(0,1.5) node[](v1){} to[R, l = $R_3$] ++(-3.5,0) to[short] ++(0,3);
	\draw (v1) to[R, l_= $R_3$] ++(3.5,0) to[short] ++(0,3);
	\draw (v1) to[open] ++(0,3) node[potentiometershape, rotate=180, label=south:$R_2$](P){};
	\draw (v1) to[C, l = $C_2$] (P.wiper);
	\draw (P.right) to[R, l= $R_1$] ++(-4,0) node[ocirc,label=left:$V_{in}$]{};
	\draw (P.left) to[R, l_= $R_1$] ++(4,0) node[ocirc,label=right:$V_{out}$]{};
	\draw (P.right) to[open] ++(-0.5,0) to[short] ++(0,1.5) to[C, l = $C_1$] ++(2.2,0) to[short] ++(0,-1.5);
	\draw (A.out) to[short] ++(0.62,0) to[short] ++(0,2);
\end{circuitikz}
	\caption{Circuito a analizar.}
	\label{fig:circuitoini}
\end{center}
\end{figure}

Para facilitar el análisis, se reemplaza el potenciometro $R_2$ con dos resistencias fijas, tal que se cumpla $R_2 = R_{2}^{'} + R_{2}^{''} $, considerando que
\begin{equation}
\begin{split}
	R_{2}^{'} &= \psi R_2\\ R_{2}^{''} &= \left( 1 - \psi \right) R_2
\end{split}
\label{equ:r2p}
\end{equation}

\begin{figure}[H]
\begin{center}
\begin{circuitikz}
	\node [op amp](A){};
	\draw (A.+) to[short] ++(-0.5,0) to[short] ++(0,-1) node[ground]{};
	\draw (A.-) to[short] ++(-0.5,0) to[short] ++(0,1.5) node[](v1){} to[R, l = $R_3$] ++(-3.5,0) to[short] ++(0,3);
	\draw (v1) to[R, l_= $R_3$] ++(3.5,0) to[short] ++(0,3);
	\draw (v1) to[C, l = $C_2$] ++(0,1.5) node[](v2){};
	\draw[color=red] (v2) to[short] ++(-1,0) to[R, l= $R_{2}^{'}$, color = red] ++(0,1.5) node[](v3){};
	\draw[color=red] (v2) to[short] ++(1,0) to[R, l_= $R_{2}^{''}$, color = red] ++(0,1.5) node[](v4){};
	\draw[color=red] (v3) to[C, l = $C_1$, color = red] (v4);
	\draw (v3) to[R, l= $R_1$] ++(-3,0) node[ocirc,label=left:$V_{in}$]{};
	\draw (v4) to[R, l_= $R_1$] ++(3,0) node[ocirc,label=right:$V_{out}$]{};
	\draw (A.out) to[short] ++(0.62,0) to[short] ++(0,2);
\end{circuitikz}
	\caption{Reemplazando $R_2$ por resistencias fijas.}
	\label{fig:kennelly1}
\end{center}
\end{figure}

Luego se aplica el teorema de Kennelly para transformar la conexión tipo Pi, marcada en rojo en la Figura (\ref{fig:kennelly1}), a una tipo T.
\begin{figure}[H]
\begin{center}
\begin{circuitikz}
	\node [op amp](A){};
	\draw (A.+) to[short] ++(-0.5,0) to[short] ++(0,-1) node[ground]{};
	\draw (A.-) to[short] ++(-0.5,0) to[short] ++(0,1.5) node[](v1){} to[R, l = $R_3$] ++(-3.5,0) to[short] ++(0,3);
	\draw (v1) to[R, l_= $R_3$] ++(3.5,0) to[short] ++(0,3);
	\draw[color=red] (v1) to[C, l = $C_2$] ++(0,1.5) to[generic, l = $Z_C$] ++(0,1.5) node[](v2){};
	\draw[color=red] (v2) to[generic, l_= $Z_A$] ++(-2,0) to[R, l= $R_1$] ++(-1.5,0) node[](aux2){};
	\draw (aux2) to[short] ++(-1,0) node[ocirc,label=left:$V_{in}$]{};
	\draw[color=red] (v2) to[generic, l = $Z_B$] ++(2,0) to[R, l_= $R_1$] ++(1.5,0) node[](aux1){};
	\draw (aux1) to[short] ++(1,0) node[ocirc,label=right:$V_{out}$]{};
	\draw (A.out) to[short] ++(0.62,0) to[short] ++(0,2);
\end{circuitikz}
	\caption{Resultado de aplicar el teorema de Kennelly por primera vez.}
	\label{fig:kennelly2}
\end{center}
\end{figure}

Siendo:
\begin{equation*}
	Z_{A} = \frac{R_{2}^{'}}{C_{1} R_{2} S + 1}
\end{equation*}	
\begin{equation*}
	Z_{B} = \frac{R_{2}^{''}}{C_{1} R_{2} S + 1}
\end{equation*}	
\begin{equation*}
	Z_{C} = \frac{C_{1} R_{2}^{''} R_{2}^{'} S}{C_{1} R_{2} S + 1}
\end{equation*}

Nuevamente se aplica Kennelly, pero esta vez para transformar una configuración del tipo T al tipo Pi. Se utiliza la conexión marcada en rojo en la Figura (\ref{fig:kennelly2}).
\begin{figure}[H]
\begin{center}
\begin{circuitikz}
	\node [op amp](A){};
	\draw (A.+) to[short] ++(-0.5,0) to[short] ++(0,-1) node[ground]{};
	\draw (A.-) to[short] ++(-0.5,0) to[short] ++(0,1.5) node[](v1){} to[short] ++(-1.5,0) node[](aux1){};
	\draw[color=red] (aux1) to[short] ++(-2,0) to[R, l = $R_3$] ++(0,3);
	\draw[color=red] (aux1) to[generic, l = $Z_{AC}$] ++(0,3) node[](v2){};

	\draw (v1) to[short] ++(1.5,0) node[](aux2){};
	\draw[color=red] (aux2) to[short] ++(2,0) to[R, l_= $R_3$] ++(0,3);
	\draw[color=red] (aux2) to[generic, l_= $Z_{BC}$] ++(0,3) node[](v3){};
	
	\draw (v2) to[generic, l = $Z_{AB}$] (v3);
	\draw[color=red] (v3) to[short] ++(2,0) node[](aux3){};
	\draw (aux3) to[short] ++(1,0) node[ocirc,label=right:$V_{out}$]{};
	
	\draw[color=red] (v2) to[short] ++(-2,0) node[](aux4){};
	\draw (aux4) to[short] ++(-1,0)node[ocirc,label=left:$V_{in}$]{};
	\draw (A.out) to[short] ++(0.62,0) to[short] ++(0,2);
\end{circuitikz}
	\caption{Resultado de aplicar el teorema de Kennelly por segunda vez.}
	%\label{}
\end{center}
\end{figure}

Obteniendose así:
\begin{equation*}
	Z_{AC} =  \frac{\alpha_{AC} {C_{1}}^{2} C_2 R_{1} R_{2} S^{3} + \beta_{AC} C_1 S^{2}
		+ \gamma_{AC} S + 2 R_1 + R_{2}^{'} + R_{2}^{''}}{
		 C_2 S \left( C_1 R_{2} S + 1 \right)
		\left(C_1 R_1 R_{2} S + R_1 + R_{2}^{''}\right)}
\end{equation*}
con
\begin{equation*}
\begin{split}
	\alpha_{AC} = R_{1} R_{2} + 2 {C_{1}}^{2} C_2 R_{2}^{'} R_{2}^{''}
\end{split}
\end{equation*}
\begin{equation*}
\begin{split}
	\beta_{AC} =\ & 2 C_{1} R_1 R_{2}^{2} + 2 C_2 R_{1}^{2} R_{2} + C_2 R_1 R_{2} R_{2}^{'} +
		C_2 R_1 R_{2} R_{2}^{''} +\\
		&2 C_2 R_1 R_{2}^{'} R_{2}^{''} +
		C_2 R_{2}^{'2} R_{2}^{''} + C_2 R_{2}^{'} R_{2}^{''2}
\end{split}
\end{equation*}
\begin{equation*}
\begin{split}
	\gamma_{AC} =\ & 4 C_1 R_1 R_{2} + C_1 R_{2} R_{2}^{'} + C_1 R_{2} R_{2}^{''} +
		C_2 {R_{1}}^{2} +\\
		&C_2 R_1 R_{2}^{'} + C_2 R_1 R_{2}^{''} + C_2 R_{2}^{'} R_{2}^{''}
\end{split}
\end{equation*}

\begin{equation*}
	Z_{AB} = \frac{3 C_1 R_1 R_{2} S + 3 R_1 + 2 R_{2}^{'} + R_{2}^{''}}{C_1 R_{2} S + 1}
\end{equation*}

\begin{equation*}
	Z_{BC} =
	\frac{ \left( R_1 R_{2} + 2 R_{2}^{'} R_{2}^{''} \right ){C_{1}}^{2} C_2 R_1 R_{2} S^{3} +
	\beta_{BC} C_1 S^{2} + \gamma_{BC} S + 2 R_1 + R_{2}^{'} + R_{2}^{''}}
	{C_2 S \left(C_1 R_{2} S + 1\right) \left(C_1 R_1 R_{2} S + R_1 + R_{2}^{'} \right)}
\end{equation*}
con
\begin{equation*}
\begin{split}
	\beta_{BC} =\ & 2 C_1 R_1 {R_{2}}^{2} + 2C_2 {R_{1}}^{2} R_{2} + C_2 R_1 R_{2} R_{2}^{'} +\\
	&C_2 R_1 R_{2} R_{2}^{''} + 2 C_2 R_1 R_{2}^{'} R_{2}^{''} + 
	C_2 {R_{2}^{'}}^2 R_{2}^{''} + C_2 R_{2}^{'} R_{2}^{''2}
\end{split}
\end{equation*}
\begin{equation*}
\begin{split}
	\gamma_{BC} =  4 C_1 R_1 R_{2} + C_1 R_{2} R_{2}^{'} + C_1 R_{2} R_{2}^{''} +
	C_2 {R_{1}}^{2} + C_2 R_1 R_{2}^{'} + C_2 R_1 R_{2}^{''} +
	C_2 R_{2}^{'} R_{2}^{''} 
\end{split}
\end{equation*}

Luego, se toma el paralelo entre las impedancias marcadas en rojo en la Figura (\ref{fig:paralelo}).
\begin{figure}[H]
\begin{center}
\begin{circuitikz}
	\node [op amp](A){};
	\draw (A.-) to[short] ++(-1,0) to[generic, l = $Z_{AC}^{'}$] ++(-2,0) node[](v1){} to[short, -o] node[ocirc,label=left:$V_{in}$]{} ++(-1,0);
	\draw[color=red] (v1) to[short] ++(0,2.5) to[generic, l = $Z_{AB}$] ++(6.5,0) to[short] ++(0,-3);
	\draw (A.+) to[short] ++(-0.5,0) to[short] ++(0,-1) node[ground]{};
	\draw (A.-) to[short] ++(-0.5,0) to[short] ++(0,1.5) to[generic, l_= $Z_{BC}^{'}$] ++(3,0) to[short] ++(0,-2);
	\draw (A.out) to[short, -o] ++(2,0) node[ocirc,label=right:$V_{out}$]{};
\end{circuitikz}
	\caption{Resultado de tomar las impedancias en paralelo.}
	\label{fig:paralelo}
\end{center}
\end{figure}

Es así que, definiendo: $Z_{AC}^{'} = Z_{AC} // R_3 $, $Z_{BC}^{'} = Z_{AC} // R_3 $ y $k = \frac{V_{out}}{V_{in}}$, se aplica el teorema de Miller a la impedancia $Z_{AB}$, señalada en la Figura (\ref{fig:paralelo}).
\begin{figure}[H]
\begin{center}
\begin{circuitikz}
	\node [op amp](A){};
	\draw (A.-) to[short] ++(-1,0) to[generic, l_= $Z_{AC}^{'}$] ++(-2,0) node[](v1){} to[short, -o] node[ocirc,label=left:$V_{in}$]{} ++(-1,0);
	\draw (v1) to[generic, l = $\frac{Z_{AB}}{1 - k}$] ++(0,-2) node[ground]{};
	\draw (A.out) ++(0.5,0) to[generic, l = $\frac{Z_{AB}}{1 - \frac{1}{k}}$] ++(0,-1.5) node[ground]{};
	\draw (A.+) to[short] ++(-0.5,0) to[short] ++(0,-1) node[ground]{};
	\draw (A.-) to[short] ++(-0.5,0) to[short] ++(0,1.5) to[generic, l= $Z_{BC}^{'}$] ++(3,0) to[short] ++(0,-2);
	\draw (A.out) to[short, -o] ++(2,0) node[ocirc,label=right:$V_{out}$]{};
\end{circuitikz}
	\caption{Modelo final, resultado de aplicar el terorema de Miller.}
	\label{fig:final}
\end{center}
\end{figure}

La transferencia del circuito de la Figura (\ref{fig:final}) corresponde a la de un amplificador inversor, por lo tanto, considerando (\ref{equ:r2p}), se puede obtener:
\begin{equation}
	H(s) = -\frac{Z_{BC}^{'}}{Z_{AC}^{'}} = - \frac{\alpha_H S^{2} + \beta_H S + 2 R_{1} + R_{2}}
	{\gamma_H S^{2} + \delta_H S + 2 R_{1} + R_{2}}
	\label{equ:hs}
\end{equation}

Siendo: 
\begin{equation*}
\begin{split}
	\alpha_H =\ & C_{1} C_{2} {R_{1}}^{2} R_{2} - 2 C_{1} C_{2} R_{1} {R_{2}}^{2} \psi^{2} + 2 C_{1} C_{2} R_{1} {R_{2}}^{2} \psi + C_{1} C_{2} R_{1} R_{2} R_{3}
\end{split}
\end{equation*}
\begin{equation*}
\begin{split}
	\beta_H =\ & 2 C_{1} R_{1} R_{2} + C_{2} {R_{1}}^{2} + C_{2} R_{1} R_{2} + C_{2} R_{1} R_{3} -\\ & C_{2} {R_{2}}^{2} \psi^{2} + C_{2} {R_{2}}^{2} \psi - C_{2} R_{2} R_{3} \psi + C_{2} R_{2} R_{3}
\end{split}
\end{equation*}
\begin{equation*}
\begin{split}
	\gamma_H =\ & C_{1} C_{2} {R_{1}}^{2} R_{2} - 2 C_{1} C_{2} R_{1} {R_{2}}^{2} \psi^{2} + 2 C_{1} C_{2} R_{1} {R_{2}}^{2} \psi + C_{1} C_{2} R_{1} R_{2} R_{3}
\end{split}
\end{equation*}
\begin{equation*}
\begin{split}
 \delta_H =\ & 2 C_{1} R_{1} R_{2} + C_{2} {R_{1}}^{2} + C_{2} R_{1} R_{2} + C_{2} R_{1} R_{3} - C_{2} {R_{2}}^{2} \psi^{2} + C_{2} {R_{2}}^{2} \psi + C_{2} R_{2} R_{3} \psi
\end{split}
\end{equation*}

Si se tienen en cuenta las siguientes consideraciones: $C_1 = 10 \ C_2$, $R_3 \gg R_1$ y $R_3 = 10 \ R_2$, se pueden reescribir los coeficientes de (\ref{equ:hs}), obteniéndose: 
\begin{equation*}
\begin{split}
	\alpha_H =\ \gamma_H =\ & 10 {C_{2}}^{2} {R_{2}}^{2} R_{1} \left[ 10 + 2 \psi \left(1 - \psi \right) \right]
\end{split}
\end{equation*}
\begin{equation*}
\begin{split}
	\beta_H =\ & C_{2} R_{2} \left[ 31 R_1 + R_2 \left(10 - 9 \psi - \psi^2 \right) \right]
\end{split}
\end{equation*}
\begin{equation*}
\begin{split}
 \delta_H =\ & C_{2} R_{2} \left[ 31 R_1 + R_2 \psi \left(11 - \psi \right) \right]
\end{split}
\end{equation*}

De esta forma, con los nuevos coeficientes simplificados, se obtiene de (\ref{equ:hs}) la frecuencia de corte del filtro:
\begin{equation}
\begin{split}
	f_o &=\ \frac{1}{2 \pi} \sqrt{\frac{2R_1 + R_2}{\alpha_H}} = \sqrt{\frac{2R_1 + R_2}{10 {C_{2}}^{2} {R_{2}}^{2} R_{1} \left[ 10 + 2 \psi \left(1 - \psi \right) \right]}} = \\
	&=\ \sqrt{\frac{2 + \frac{R_2}{R_1}}{10 \left[ 10 + 2 \psi \left(1 - \psi \right) \right]}} \cdot \frac{1}{2 \pi C_2 R_2}
\end{split}
\label{equ:fogeneral}
\end{equation}

Tomando tanto $\psi = 1$, como $\psi = 0$ se obtiene la misma frecuencia de corte
\begin{equation*}
\begin{split}
	f_o &=\ \frac{\sqrt{2 + \frac{R_2}{R_1}}}{20 \pi C_2 R_2}
\end{split}
\end{equation*}

De esta forma, se busca la variación de amplitud máxima en dicha frecuencia. Esto se realiza evaluando la transferencia del sistema en $J\omega_o$, obteniendo de esta forma $A_{0Max}$ y $A_{0Min}$. Es así que, tomando $\psi = 0$:
\begin{equation*}
\begin{split}
|H\left(J\omega_o\right)| = A_{0Max} = \frac{31 R_{1} + 10 R_{2}}{31 R_{1}} \approx \frac{30 R_{1} + 10 R_{2}}{30 R_{1}} = \frac{3 R_{1} + R_{2}}{3 R_{1}}
\end{split}
\end{equation*}

Por otro lado, con $\psi = 1$:
\begin{equation*}
\begin{split}
|H\left(J\omega_o\right)| = A_{0Min} = \frac{31 R_{1}}{31 R_{1} + 10 R_{2}} \approx \frac{30 R_{1}}{30 R_{1} + 10 R_{2}} = \frac{3 R_{1}}{3 R_{1} + R_{2}}
\end{split}
\end{equation*}

Es así que se puede determinar que
\begin{equation*}
\begin{split}
\frac{3 R_{1}}{3 R_{1} + R_{2}} \leq A_0 \leq \frac{3 R_{1} + R_{2}}{3 R_{1}}
\end{split}
\end{equation*}

Por otro lado, mediante el uso de (\ref{equ:hs}), se realiza un diagrama de polos y ceros en el plano $S$, variando el factor $\psi$. De esta forma se obtiene la Figura (\ref{fig:zplanepsi}).

\begin{figure}
	\includegraphics[width=\textwidth]{Imagenes/Zplanepsi.png}
\caption{Diagrama de polos y ceros variando $\psi$ entre 0 y 1.}
	\label{fig:zplanepsi}
\end{figure}

Debido a que, tanto los polos como los ceros del sistema, sin importar el valor de $\psi$, se encuentran en el semiplano izquierdo, se obtiene el menor diagrama de fase posible, en consecuencia, este sistema es de fase mínima. 

\end{document}
