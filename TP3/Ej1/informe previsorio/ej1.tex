\documentclass[a4paper]{article}
\usepackage[utf8]{inputenc}
\usepackage[spanish, es-tabla]{babel}

\usepackage{geometry}
 \geometry{includehead, footskip=7mm, headsep=6mm, headheight=4.8mm, top=25mm, bottom=5mm, left=10mm, right=10mm}

%\usepackage[a4paper, 					% Page Layout
%                     %showframe,				% This shows the frame
%                     includehead,
%                     footskip=7mm, headsep=6mm, headheight=4.8mm,
%                     top=25mm, bottom=5mm, left=5mm, right=5mm]{geometry}

\usepackage{amsmath}
\usepackage{amsfonts}
\usepackage{amssymb}

\usepackage{float}
\usepackage{graphicx}
\usepackage{caption}
\usepackage{subcaption}

\usepackage{multirow}
\setlength{\doublerulesep}{\arrayrulewidth}

\newcommand{\quotes}[1]{``#1''}

\usepackage{array}
\newcolumntype{C}[1]{>{\centering\let\newline\\\arraybackslash\hspace{0pt}}m{#1}}

\usepackage[american]{circuitikz}

\usepackage{fancyhdr}

\usepackage{units} 

\pagestyle{fancy}
\fancyhf{}
\lhead{22.01 Teoría de Circuitos}
\rfoot{Página \thepage}
\begin{document}

\subsection{Introducción Teórica}

\subsubsection{Filtros Activos con GIC}

Los inductores suelen ser los componentes electrónicos menos ideales a la hora de querer realizar un filtro. Esto se debe a varias razones: su tamaño en frecuencias bajas y por ende su peso, su resistencia interna grande en comparación a los capacitores, su dificultad a la hora del armado y más. Con el surgimiento del concepto de la realimentación en los circuitos electrónicos, se pudo mediante el uso de amplificadores operacionales, obtener cualquier tipo de respuesta en los filtros. Además, como los operacionales son dispositivos activos, estos pueden proveer al circuito la energía que es disipada por los resistores y aún más por medio de la alimentación que estos dispositivos requieren.

Sin embargo, varias limitaciones surgen por el uso de operacionales. La más importante suele ser la dependencia de la ganancia a lazo abierto con la frecuencia y su inconveniente respuesta a altas frecuencias. Esto suele ocasionar que los filtros activos tengan un rango de correcta operación por debajo de hasta los $100Mhz$. Afortunadamente cuanto mayor la frecuencia menores son las dificultades presentadas por el uso de inductores y esto hace que para este rango de frecuencias el inductor vuelva a ser una buena opción a la hora de implementar filtros.

Varias implementaciones de filtros activos han sido utilizadas históricamente, siendo una de estas el método de la síntesis directa. Este método se basa en utilizar distintos convertidores de impedancia como giradores o GICs. Este informe se centra, en el análisis e implementación de filtros utilizando el método de síntesis directa, entre otras cosas, comenzando con una breve explicación del funcionamiento de un GIC.

\subsubsection{Convertidor Generalizado de Impedancia o GIC}
Los convertidores generalizados de impedancia o GICs son circuitos activos diseñados para simular impedancias que cambian con la frecuencia. Se puede observar en la Figura (\ref{fig:gic}) una implementación de un GIC.

\begin{figure}[H]
	\centering
	\includegraphics[width=0.6\textwidth]{Imagenes/gic.PNG}
	\caption[]{Convertidor generalizado de impedancia puesto a tierra\protect\footnote{}.}
	\label{fig:gic}
\end{figure}

\footnotetext{F. Sergio, Design with operational amplifiers and analog integrated circuits, 4th ed. New York [etc.]: McGraw-Hill, 1988, p. 185.}

Estos circuitos están compuestos solamente por resistores, capacitores y amplificadores operacionales, por lo que sobresalen en su uso como emuladores de inductores o capacitores de gran capacidad.

\begin{figure}[H]
	\centering
	\includegraphics[width=0.5\textwidth]{Imagenes/gic_zin.PNG}
	\caption[]{Obtención de la impedancia de entrada del convertidor generalizado de impedancia puesto a tierra\protect\footnote{}.}
	\label{fig:gic_zin}
\end{figure}

\footnotetext{F. Sergio, Design with operational amplifiers and analog integrated circuits, 4th ed. New York [etc.]: McGraw-Hill, 1988, p. 186.}

Para obtener la impedancia de entrada del GIC basta solamente calcular la corriente de entrada y utilizar la ley de nodos de Kirchhoff en los nodos de potencial V.

\begin{equation}
\frac{V-V_1}{Z_1}=I
\label{gic_zin_1}
\end{equation}

\begin{equation}
\frac{V_1-V}{Z_2} = \frac{V-V_2}{Z_3}
\label{gic_zin_2}
\end{equation}

\begin{equation}
\frac{V_2-V}{Z_4} = \frac{V}{Z_5}
\label{gic_zin_3}
\end{equation}

Utilizando (\ref{gic_zin_1}), (\ref{gic_zin_2}) y (\ref{gic_zin_3}) se llega finalmente a

\begin{equation}
Z_{in} = \frac{Z_1 Z_3 Z_5}{Z_2 Z_4}
\label{grounded_gic_zin}
\end{equation}

Esto demuestra que un GIC se puede utilizar para emular la impedancia que se desee, eligiendo convenientemente las impedancias $Z_1, \ Z_2, \ Z_3, \ Z_4 \ y \ Z_5$ dentro de las limitaciones que este posee.

\begin{figure}[H]
	\centering
	\includegraphics[width=1\textwidth]{Imagenes/gic_ind_fndr.PNG}
	\caption[]{Utilización de GICs para emular una inductancia a la izquierda y realizar un FNDR o elemento D a la derecha\protect\footnote{}.}
	\label{fig:gic_ind_fndr}
\end{figure}

\footnotetext{F. Sergio, Design with operational amplifiers and analog integrated circuits, 4th ed. New York [etc.]: McGraw-Hill, 1988, p. 187.}

Como se puede observar en la Figura (\ref{fig:gic_ind_fndr}), dos implementaciones muy utilizadas de los GICs puestos a tierra son la de un inductor o un FNDR. Sin embargo, estas implementaciones poseen varias limitaciones a la hora de realizar el diseño de un filtro ya que por la naturaleza interna de los amplificadores operacionales, se debe tener en cuenta que debe haber un camino para la continua que pueda ser utilizada para polarizar correctamente los transistores en el interior de los operacionales. En el siguiente circuito se analizará e implementará un filtro activo con un GIC que no está puesto a tierra.


\subsection{Análisis del Circuito}

En una primera instancia, se puede observar en la Figura (\ref{fig:circ_redibujado}) como el circuito a analizar consiste de un filtro de primer orden compuesto por un convertidor generalizado de impedancia cuya salida se encuentra realimentada a la entrada. Si bien la elección de las impedancias del GIC sugieren que este se comporta como un inductor y que el filtro será de segundo orden, la realimentación por la resistencia $R_8$ al potencial de entrada exige un análisis detallado de la función transferencia del circuito. Este será el primer paso a tomar en el análisis del filtro.

\subsubsection{Análisis de la Función Transferencia}
\label{sec:fun_trans}
Para realizar el análisis de la función transferencia se consideran los amplificadores operacionales como ideales.

\begin{figure}[H]
\centering
\scalebox{0.5}{
\begin{circuitikz}
\draw

(5, 0) node[label=north:$V_1$](V1){} %Punto inicial, mas o menos centrado horizontalmente
	to[short] ++ (-3,0)
	to[R, l=$R_6$, *-] ++ (0, -2)
	node[ground]{}
	to[open] ++ (0, 2)
	to[short] ++ (-1, 0)
	to[C, l_=$C_6$, *-] ++ (0, -2)
	node[ground]{}
	to[open] ++ (0, 2)
	to[short] ++ (-1, 0)
	node[](R7_right){}
	to[C, l=$C_7$, *-*] ++ (-2, 0)
	node[](R7_left){}
	to[short] ++ (-1,0)
	node[](gic_loop){}
	to[short] ++ (-1,0)
	to[V, l=$V_i$] ++ (0, -2)
	node[ground]{}

(R7_left) to[short] ++ (0, 1)
	to[R, l=$R_7$] ++ (2, 0)
	to[short] (R7_right)	

(V1) to[R, l=$R_1$] ++ (0, -2)
	node[label = east:$V_2$](V2){}
	
(V2) to[C, l=$C_2$] ++ (0, -2)
	node[](V3){}
	to[open] ++ (-0.5, 0.5)
	node[](){$V_3$}
	to[open] ++ (0.5, -0.5)

(V3) to[R, l=$R_3$] ++ (0, -2)
	node[label=west:$V_4$](V4){}
	
(V4) to[R, l=$R_4$] ++ (0, -2)
	node[label=east:$V_5$](V5){}

(V5) to[R, l=$R_8$] ++ (0,-2)
	to[short] ++ (-8, 0)
	to[short, -*] (gic_loop)

(V2) to[open] ++ (3, 0)
	node[op amp, yscale=-1](opamp){}

(opamp.+) to[short] ++ (-0.5, 0)
	to[short] ++ (0, 1.5)
	to[short, -*] (V1)
	
(opamp.-) to[short] ++ (-0.5, 0)
	to[short] ++ (0, -1.5)
	to[short, -*] (V3)

(opamp.out) to[short] ++ (1, 0)
	node[label=east:$V_{out}$](Vout){}
	to[short] ++ (0, -4)
	to[short, -*] (V4)
	
(V4) ++ (-3, 0) node[op amp, xscale=-1](opamp2){}

(opamp2.+) to[short] ++ (0.5, 0)
	to[short] ++ (0, -1.5)
	to[short, -*] (V5)
	
(opamp2.-) to[short] ++ (0.5, 0)
	to[short] ++ (0, 1.5)
	to[short, -*] (V3)
	
(opamp2.out) to[short] ++ (0, 2.5)
	to[short] ++ (2.5, 0)
	to[short] ++ (0, 1.5)
	to[short, -*] (V2)

;
\end{circuitikz}
}
\caption{Circuito a analizar redibujado para facilitar la obtención de la función transferencia.}
\label{fig:circ_redibujado}
\end{figure}

Redibujando adecuadamente el circuito, se puede observar como los operacionales mantienen a sus entradas el mismo potencial. Como paso siguiente, y para utilizar el método de nodos en aquellos que son mantenidos a un mismo potencial, se debe dibujar nuevamente al circuito aplicando el teorema de Thévenin entre los nodos $V_1$ y tierra. Para esto, se desconecta la realimentación por medio de la resistencia $R_8$ teniendo en cuenta los efectos que esta producen sobre el circuito.

\begin{figure}[H]
\centering
\begin{subfigure}[t]{0.49\textwidth}
\centering
\scalebox{0.4}{
\begin{circuitikz}

\draw

(5, 0) node[label=north:$V_1$](V1){} %Punto inicial, mas o menos centrado horizontalmente
	to[short] ++ (-3,0)
	to[R, l=$R_6$, *-] ++ (0, -2)
	node[ground]{}
	to[open] ++ (0, 2)
	to[short] ++ (-1, 0)
	to[C, l_=$C_6$, *-] ++ (0, -2)
	node[ground]{}
	to[open] ++ (0, 2)
	to[short] ++ (-1, 0)
	node[](R7_right){}
	to[C, l=$C_7$, *-*] ++ (-2, 0)
	node[](R7_left){}
	to[short] ++ (-1,0)
	node[](gic_loop){}
	to[short] ++ (-1,0)
	to[V, l=$V_i$] ++ (0, -2)
	node[ground]{}

(R7_left) to[short] ++ (0, 1)
	to[R, l=$R_7$] ++ (2, 0)
	to[short] (R7_right)	

(V1) to[R, l=$R_1$] ++ (0, -2)
	node[label = east:$V_2$](V2){}
	
(V2) to[C, l=$C_2$] ++ (0, -2)
	node[](V3){}
	to[open] ++ (-0.5, 0.5)
	node[](){$V_3$}
	to[open] ++ (0.5, -0.5)

(V3) to[R, l=$R_3$] ++ (0, -2)
	node[label=west:$V_4$](V4){}
	
(V4) to[R, l=$R_4$] ++ (0, -2)
	node[label=east:$V_5$](V5){}

(V5) to[R, l=$R_8$] ++ (0,-2)
	to[short] ++ (-7, 0)
	to[short, -*] ++ (0, 5)
	to[V, l=$V_i$] ++ (-2, 0)
	to[short] ++ (0, -1)
	node[ground](){}

(V2) to[open] ++ (3, 0)
	node[op amp, yscale=-1](opamp){}

(opamp.+) to[short] ++ (-0.5, 0)
	to[short] ++ (0, 1.5)
	to[short, -*] (V1)
	
(opamp.-) to[short] ++ (-0.5, 0)
	to[short] ++ (0, -1.5)
	to[short, -*] (V3)

(opamp.out) to[short] ++ (1, 0)
	node[label=east:$V_{out}$](Vout){}
	to[short] ++ (0, -4)
	to[short, -*] (V4)
	
(V4) ++ (-3, 0) node[op amp, xscale=-1](opamp2){}

(opamp2.+) to[short] ++ (0.5, 0)
	to[short] ++ (0, -1.5)
	to[short, -*] (V5)
	
(opamp2.-) to[short] ++ (0.5, 0)
	to[short] ++ (0, 1.5)
	to[short, -*] (V3)
	
(opamp2.out) to[short] ++ (0, 2.5)
	to[short] ++ (2.5, 0)
	to[short] ++ (0, 1.5)
	to[short, -*] (V2)

;

\end{circuitikz}
}
\caption{Circuito redibujado.}\label{fig:circ_redibujado_2}
\end{subfigure}
\begin{subfigure}[t]{0.49\textwidth}
\centering
\scalebox{0.6}{
\begin{circuitikz}
 
 \draw

(0,0) node[label=north:$V_1$](V1){}
	to[short, *-] ++ (-1, 0)
	to[R, l=$R_6$] ++ (0, -2)
	node[ground](){}
	to[open] ++ (0, 2)
	to[short, *-*] ++ (-1, 0)
	to[C, l_=$C_6$] ++ (0,-2)
	node[ground](){}
	to[open] ++ (0, 2)
	to[short] ++ (-1, 0)
	to[C, l=$C_7$, *-*] ++ (-2, 0)
	to[short] ++ (0, 1)
	to[R, l=$R_7$] ++ (2, 0)
	to[short] ++ (0, -1)
	to[open] ++ (-2, 0)
	to[short] ++ (-1, 0)
	to[V, l=$V_i$] ++ (0, -2)
	node[ground](){}
	to[open] ++ (0, -2)

;
 
\end{circuitikz}
}
\caption{Circuito para el análisis de Thévenin}
\label{fig:circ_thevenin}
\end{subfigure}
\end{figure}

Se puede observar en la Figura (\ref{fig:circ_thevenin}) como la tensión de Thévenin será el divisor resistivo entre las impedancias equivalentes $Z_6$ y $Z_7$, siendo estas el resultado del paralelo entre $R_6$ y $C_6$ y entre $R_7$ y $C_7$ respectivamente, mientras que la impedancia de Thévenin será el paralelo entre ambas impedancias equivalentes. Resulta entonces:

\begin{equation}
V_{Th} = V_i \cdot \frac{\left(R_6 // \frac{1}{SC_6}\right)}{\left(R_6 // \frac{1}{SC_6}\right) + \left(R_7 // \frac{1}{SC_7}\right)}
\end{equation}

\begin{equation}
R_{Th} = \left(R_6 // \frac{1}{SC_6}\right) // \left(R_7 // \frac{1}{SC_7}\right)
\end{equation}

Finalmente, con las ecuaciones obtenidas habiendo aplicado el teorema de Thévenin, se aplica el método de los nodos en aquellos cuyo potencial se mantiene igual por los operacionales, resultando en:

\begin{equation}
V_1=V_3=V_5=V_A
\label{eq:circ_trans1}
\end{equation}

\begin{equation}
\frac{V_{out} - V_A}{R_4} = \frac{V_A-V_i}{R8}
\label{eq:circ_trans2}
\end{equation}

\begin{equation}
\frac{V_2 - V_A}{\frac{1}{SC_2}} = \frac{V_A-V_{out}}{R_3}
\label{eq:circ_trans3}
\end{equation}

\begin{equation}
\frac{V_i \cdot \frac{\left(R_6 // \frac{1}{SC_6}\right)}{\left(R_6 // \frac{1}{SC_6}\right) + \left(R_7 // \frac{1}{SC_7}\right)} - V_A}{\left(R_6 // \frac{1}{SC_6}\right) // \left(R_7 // \frac{1}{SC_7}\right)} = \frac{V_A-V_2}{R_1}
\label{eq:circ_trans4}
\end{equation} \\

A partir de (\ref{eq:circ_trans1}), (\ref{eq:circ_trans2}), (\ref{eq:circ_trans3}) y (\ref{eq:circ_trans4}) se obtiene:

\begin{equation}
\hspace{-0.5cm}
\frac{V_{out}}{V_i} = \frac{S^2 C_2 R_1 R_3 R_6 R_7 (-C_6 R_4 + C_7 R_8) + S C_2 R_1 R_3 (R_6 R_8 - R_4 R_7) + R_4 R_6 R_7}{S^2 C_2 R_1 R_3 R_6 R_7 R_8 (C_6 + C_7)+S C_2 R_1 R_3 R_8 (R_7 + R_6) + R_4 R_6 R_7}
\label{circ_trans}
\end{equation}

\subsubsection{Análisis de Sensibilidades}
\label{sec:sens}

El análisis de sensibilidades es una herramienta fundamental a la hora de diseñar no solo filtros, sino cualquier circuito electrónico. Este análisis permite conocer cuánto cambia un parámetro fundamental del circuito, como por ejemplo la selectividad, dada una variación en otro de sus parámetros, como por ejemplo una resistencia.

La sensibilidad de un parámetro $y$ respecto a otro parámetro $x$ se calcula como

\begin{equation}
S^y_x = \frac{x}{y} \frac{\partial y}{\partial x}
\end{equation}

Para el análisis de sensibilidades se remite a la función transferencia obtenida en la Sección (\ref{sec:fun_trans}) reescrita a continuación.

\[
\hspace{-0.5cm}
\frac{V_{out}}{V_i} = \frac{\frac{S^2 C_2 R_1 R_3 (-C_6 R_4 + C_7 R_8)}{R_4} + S \frac{C_2 R_1 R_3 (R_6 R_8 - R_4 R_7)}{R_4 R_6 R_7} + 1}{S^2 \frac{C_2 R_1 R_3 R_8 (C_6 + C_7)}{R_4}+S \frac{C_2 R_1 R_3 R_8 (R_7 + R_6)}{R_4 R_6 R_7} + 1}
\]

Como primer paso, se realiza un análisis de la sensibilidad de la frecuencia de notch $\omega_z$ en función de los componentes resistivos y capacitivos del circuito.

Considerando que
\begin{equation}
\omega_z = \frac{1}{\sqrt{\frac{C_2 R_1 R_3 (C_7 R_8 - C_6 R_4)}{R_4}}}
\end{equation}

se halla la derivada parcial de $\omega_z$ respecto de $R_1$ como

\begin{equation}
\frac{\partial \omega_z}{\partial R_1} = -\frac{1}{2} \frac{C_2 R_3 (C_7 R_8 - C_6 R_4)}{(\frac{C_2 R_1 R_3 (C_7 R_8 - C_6 R_4)}{R_4})^{\frac{3}{2}} R_4}
\end{equation}

siendo finalmente la sensibilidad de la frecuencia de notch respecto de la resistencia $R_1$

\begin{equation}
S^{\omega_z}_{R_1} = -\frac{1}{2} \frac{C_2 R_3 (C_7 R_8 - C_6 R_4)}{(\frac{C_2 R_1 R_3 (C_7 R_8 - C_6 R_4)}{R_4})^{\frac{3}{2}} R_4} \frac{R_1}{\sqrt{\frac{C_2 R_1 R_3 (C_7 R_8 - C_6 R_4)}{R_4}}} = -\frac{1}{2}
\end{equation}

Luego, se realiza el mismo procedimiento con $\omega_p$, $Q_p$ y $Q_z$ siendo estas

\begin{equation}
\omega_p = \frac{1}{\sqrt{\frac{C_2 R_1 R_3 R_8  C_6 C_7}{R_4}}}
\end{equation}

\begin{equation}
Q_p = \frac{R_4 R_6 R_7 \sqrt{\frac{C_2 R_1 R_3 R_8 C_6 C_7}{R_4}}}{C_2 R_1 R_3 R_8 (R_7 + R_6)}
\end{equation}

\begin{equation}
Q_z = \frac{R_4 R_6 R_7 \sqrt{\frac{C_2 R_1 R_3 (C_7 R_8 - C_6 R_4)}{R_4}}}{C_2 R_1 R_3 (R_6 R_8 - R_4 R_7)}
\end{equation}

De forma análoga se calculó el resto de las sensibilidades de los parámetros más importantes del circuito respecto a los componentes resistivos y capacitivos presentados a continuación.

\begin{table}[H]
\centering
\begin{tabular}{@{}ccccccccc@{}}
\toprule
$S^{\omega_z}_{R_1}$ & $S^{\omega_z}_{C_2}$ & $S^{\omega_z}_{R_3}$ & $S^{\omega_z}_{R_4}$ & $S^{\omega_z}_{R_8}$ & $S^{\omega_z}_{R_6}$ & $S^{\omega_z}_{C_6}$ & $S^{\omega_z}_{R_7}$ & $S^{\omega_z}_{C_7}$  \\ \midrule
$-\frac{1}{2}$ & $-\frac{1}{2}$ & $-\frac{1}{2}$ & $-\frac{1}{2} \frac{C_7 R_8}{C_6 R_4 - C_7 R_8}$ & $-\frac{1}{2}\frac{C_7 R_8}{C_7 R_8 - C_6 R_4}$ & $0$ & $\frac{1}{2} \frac{C_6 R_4}{C_7 R_8 - C_6 R_4}$ & $0$ & $-\frac{1}{2} \frac{R_8 C_7}{C_7 R_8 - C_6 R_4}$ \\ \bottomrule
\end{tabular}
\caption{Sensibilidades de la frecuencia de notch respecto a los componentes resistivos y capacitivos del circuito.}
\label{tab:sens_wz}
\end{table}

\begin{table}[H]
\centering
\begin{tabular}{@{}ccccccccc@{}}
\toprule
$S^{\omega_p}_{R_1}$ & $S^{\omega_p}_{C_2}$ & $S^{\omega_p}_{R_3}$ & $S^{\omega_p}_{R_4}$ & $S^{\omega_p}_{R_8}$ & $S^{\omega_p}_{R_6}$ & $S^{\omega_p}_{C_6}$ & $S^{\omega_p}_{R_7}$ & $S^{\omega_p}_{C_7}$  \\ \midrule
$-\frac{1}{2}$ & $-\frac{1}{2}$ & $-\frac{1}{2}$ & $\frac{1}{2}$ & $-\frac{1}{2}$ & $0$ & $-\frac{1}{2}$ & $0$ & $-\frac{1}{2}$ \\ \bottomrule
\end{tabular}
\caption{Sensibilidades de la frecuencia de los polos respecto a los componentes resistivos y capacitivos del circuito.}
\label{tab:sens_wp}
\end{table}

\begin{table}[H]
\centering
\scalebox{0.9}{
\hspace*{-2cm}
\begin{tabular}{@{}ccccccccc@{}}
\toprule
$S^{Q_z}_{R_1}$ & $S^{Q_z}_{C_2}$ & $S^{Q_z}_{R_3}$ & $S^{Q_z}_{R_4}$ & $S^{Q_z}_{R_8}$ & $S^{Q_z}_{R_6}$ & $S^{Q_z}_{C_6}$ & $S^{Q_z}_{R_7}$ & $S^{Q_z}_{C_7}$  \\ \midrule
$-\frac{1}{2}$ & $-\frac{1}{2}$ & $-\frac{1}{2}$ & $-\frac{1}{2} \frac{R_8 (2 C_6 R_4 R_6 - C_7 R_4 R_7 - C_7 R_6 R_8)}{(R_4 R_7 - R_6 R_8)(C_6 R_4 - C_7 R_8)}$ & $\frac{1}{2} \frac{R_8 (2 C_6 R_4 R_6 - C_7 R_4 R_7 - C_7 R_6 R_8)}{(R_4 R_7 - R_6 R_8)(C_6 R_4 - C_7 R_8)}$ & $\frac{R_7 R_4}{R_4 R_7 - R_6 R_8}$ & $-\frac{1}{2}\frac{C_6 R_4}{C_7 R_8 - C_6 R_4}$ & $-\frac{R_8 R_6}{R_4 R_7 - R_6 R_8}$ & $-\frac{1}{2}\frac{C_7 R_8}{C_6 R_4 - C_7 R_8}$ \\ \bottomrule
\end{tabular}
}
\caption{Sensibilidades del factor de calidad de los ceros respecto a los componentes resistivos y capacitivos del circuito.}
\label{tab:sens_Qz}
\end{table}


\begin{table}[H]
\centering
\begin{tabular}{@{}ccccccccc@{}}
\toprule
$S^{Q_p}_{R_1}$ & $S^{Q_p}_{C_2}$ & $S^{Q_p}_{R_3}$ & $S^{Q_p}_{R_4}$ & $S^{Q_p}_{R_8}$ & $S^{Q_p}_{R_6}$ & $S^{Q_p}_{C_6}$ & $S^{Q_p}_{R_7}$ & $S^{Q_p}_{C_7}$  \\ \midrule
$-\frac{1}{2}$ & $-\frac{1}{2}$ & $-\frac{1}{2}$ & $\frac{1}{2}$ & $-\frac{1}{2}$ & $\frac{R_7}{R_7 + R_6}$ & $\frac{1}{2}$ & $\frac{R_6}{R_7+R_6}$ & $\frac{1}{2}$ \\ \bottomrule
\end{tabular}
\caption{Sensibilidades del factor de calidad de los polos respecto a los componentes resistivos y capacitivos del circuito.}
\label{tab:sens_Qp}
\end{table}

En una primera instancia se puede observar como las sensibilidades de $R_1$, $R_3$, $C_2$ resultan ser las mismas para los cuatro parámetros analizados. Esto indica que si se cometiese un error en las frecuencias de corte por la tolerancia de los componentes utilizados, este se podría mitigar calibrando el circuito con cualquiera de los tres susodichos componentes. Como los capacitores variables suelen ser muy caros y de muy poca capacidad, no se tomó en consideración esta opción.

Para continuar con el análisis de sensibilidades, se utilizaron las consideraciones de diseño para el circuito presentadas en la sección \ref{sec:eleccion_componentes}. A continuación se presenta una tabla de sensibilidades con estas consideraciones tomadas en cuenta.


\begin{table}[H]
\centering
\begin{tabular}{@{}cccccccccc@{}}
\toprule
$-$ & $R_1$ & $C_2$ & $R_3$ & $R_4$ & $R_8$ & $R_6$ & $C_6$ & $R_7$ & $C_7$  \\ \midrule


$S^\omega_z$ & $-\frac{1}{2}$ & $-\frac{1}{2}$ & $-\frac{1}{2}$ & $\frac{3}{4}$ & $-\frac{3}{4}$ & $0$ & $\frac{1}{4}$ & $0$ & $-\frac{3}{4}$ \\


$S^\omega_p$ & $-\frac{1}{2}$ & $-\frac{1}{2}$ & $-\frac{1}{2}$ & $\frac{1}{2}$ & $-\frac{1}{2}$ & $0$ & $-\frac{1}{2}$ & $0$ & $-\frac{1}{2}$ \\


$S^Q_z$ & $-\frac{1}{2}$ & $-\frac{1}{2}$  & $-\frac{1}{2}$ & $\uparrow \uparrow$ & $\uparrow \uparrow$ & $\uparrow \uparrow$ & $-\frac{1}{4}$ & $\uparrow \uparrow$ & $\frac{3}{4}$\\


$S^Q_p$ & $-\frac{1}{2}$ & $-\frac{1}{2}$ & $-\frac{1}{2}$ & $\frac{1}{2}$ & $-\frac{1}{2}$ & $\frac{1}{2}$ & $\frac{1}{2}$ & $\frac{1}{2}$ & $\frac{1}{2}$\\ \bottomrule


\end{tabular}
\caption{Sensibilidades del circuito teniendo en cuenta las consideraciones de diseño. Doble flecha hacia arriba significa sensibilidad tendiendo a infinito.}
\label{tab:sens}
\end{table}

Se puede observar en la Tabla (\ref{tab:sens}) como es crucial minimizar las tolerancias de las resistencias $R_4$, $R_8$, $R_6$ y $R_7$ para no afectar la profundidad del notch. 

\subsubsection{Elección de Componentes y Diseño}
\label{sec:eleccion_componentes}

A continuación se hace un breve análisis de la elección de los valores de los componentes y de la funcionalidad de algunos de estos.

Utilizando las consideraciones de diseño del circuito de low-pass notch:
$$R_1=R_3=R_4=R_8=R \ \ \ ; \ \ \ R_6=R_7=2QR$$
$$C_2=C \ \ \ ; \ \ \ C_6=(1-\frac{1}{k^2})\frac{C}{2}  \ \ \ ; \ \ \ C_7=(1+\frac{1}{k^2})\frac{C}{2}$$
$$k=\left(\frac{\omega_z}{\omega_p} \right) > 1$$

Se logra obtener la siguiente expresión para el circuito:

\begin{equation}
\frac{Vo}{Vi} = \frac{\left(\frac{S}{\frac{k}{C R}}\right)^2 + 1}{\left( \frac{S}{\frac{1}{CR}} \right)^2+\frac{SCR}{Q} + 1}
\label{circ_trans_simple}
\end{equation}

Finalmente utilizando las consideraciones de los valores de los componentes:
\begin{table}[H]
\centering
\begin{tabular}{@{}ccc@{}}
\toprule
$\omega_p$ & Q & $\omega_z$ \\ \midrule
13.000 $\frac{rad}{s}$ & 2 & $\sqrt{2}*\omega_p \frac{rad}{s}$ \\ \bottomrule
\end{tabular}
\end{table}
Se logra obtener el valor de k:

\begin{equation}
k = \sqrt{2}
\end{equation}

Considerando que hay una mayor facilidad de obtener resistencias de distintos valores que capacitores, se logra obtener la relación que vincula el valor de la resistencia R en función de la capacitancia C elegida:

\begin{equation}
R = \frac{\left(\frac{1}{13000}\right) \frac{s}{rad}}{C}
\end{equation}

Como se desean poseer resistencias pequeñas, ya que el ruido es proporcional a estas, se eligió un valor comercial para la capacitancia C de tal manera que la resistencia R sea del orden de los $10k\Omega$. Además, se utilizó un preset en la resistencia $R_1$ por razones argumentadas en la sección \ref{sec:sens}.

$$C=15nF \rightarrow R=5128.2\Omega$$

Es así que quedan fijados los valores de todos los componentes del circuito. A continuación se muestra una tabla con los valores que se deben utilizar, los nominales que se utilizaron en la implementación y los reales de dichos componentes.

\begin{table}[H]
\centering
\begin{tabular}{@{}ccccc@{}}
\toprule
Componente & Valor Teórico & Implementación & Valor Final & Error \\ \midrule
$C_2$ & $15nF$ & $15nF$ & $15nF$ & $0\%$ \\
$C_7$ & $11.25nF$ & $10nF//1.2nF$ & $11.2nF$ & $0.40\%$ \\
$C_6$ & $3.75nF$ & $3.3nF//470pF$ & $3.77nF$ & $0,53\%$\\
$R_{1,3,8}$ & $5128.2 \Omega$ & $12K\Omega//9K1\Omega$ & $5175.35 \Omega$ & $0,92\%$ \\
$R_4$ & $5128.2 \Omega$ & $Preset$ & $5128.2 \Omega$ & $\approx 0\%$\\
$R_{6,7}$ & $20512.82 \Omega$ & $330K\Omega//22K\Omega$ & $20625\Omega$ & $0,55\%$ \\ \bottomrule
\end{tabular}
\caption{Valores comerciales finales utilizados.}
\label{Tab:valores}
\end{table}

Con estos valores se graficó el Bode en amplitud y fase del filtro:

\begin{figure}[H]
	\centering
	\begin{subfigure}[t]{0.49\textwidth}
	\hspace*{-2cm}
	\centering
		\includegraphics[width=1.2\textwidth]{Imagenes/bodemag_calc.png}
	\end{subfigure}
	\begin{subfigure}[t]{0.49\textwidth}
	\centering
		\includegraphics[width=1.2\textwidth]{Imagenes/bodefase_calc.png}
	\end{subfigure}
	\label{fig:bode_calc}
	\caption{Gráficos de Bode para el filtro con los componentes de valor comercial.}
\end{figure}

A continuación se presenta una tabla con los valores significantes de la respuesta en frecuencia.

\begin{table}[H]
\centering
\begin{tabular}{@{}ccccccc@{}}
\toprule
- & $\omega_z$ & $\omega_p$ & Q & $G_{Banda Pasante}$  & $G_{Notch}$ & $G_{Banda Atenuante}$ \\ \midrule
Valor deseado & $2926Hz$ & $2069Hz$ & $2$ & - & - & - \\
Valor calculado & $2930Hz$ & $2066,3Hz$ & $2.01$ & $0$ & $-76.91dB$ & $-6.08dB$ \\ 
Error & $0.135\%$ & $0.131\%$ & $0.546\%$ & - & - & - \\ \bottomrule
\end{tabular}
\caption{Valores significantes teóricos de la respuesta en frecuencia calculada.}
\label{tab:rta_freq_calc}
\end{table}

\subsubsection{Análisis de Ceros y Polos}

Se realizó un análisis de los ceros y polos de la transferencia ideal calculada tanto con los valores usados en la Tabla (\ref{Tab:valores}) como variando las resistencias $R_6$ y $R_8$.
A continuación se muestran los ceros y polos con los valores de la Tabla (\ref{Tab:valores}).

\begin{figure} [H]
	\centering
	\includegraphics[width=0.8\textwidth]{Imagenes/cerosypolos_calc.PNG}
	\caption{Ceros y polos de la función transferencia ideal del circuito. Módulo y ángulo de cada cero y polo indicado.}
	\label{fig:cerosypolos_calc}
\end{figure}

Se puede observar como los polos y los ceros de la transferencia no se encuentran sobre una circunferencia de mismo radio, sino que los polos tienen una frecuencia de corte menor que la de los ceros. Esto genera que en el diagrama de Bode se encuentre primero el polo y luego el cero, por lo que la curva asintótica permanece por debajo de la línea del cero para las frecuencias mayores a la frecuencia de notch. Se puede contemplar además como los ceros se sitúan sobre el eje imaginario, por lo que el factor de calidad tiende a infinito y genera un gran sobrepico, atenuando casi totalmente las frecuencias cuyo valor sean igual al módulo de los ceros.

Luego, se procede a realizar un análisis del desplazamiento de los polos dadas variaciones en las resistencias $R_6$ y $R_8$. Cabe notar que el color más oscuro corresponde con la posición inicial de los ceros y polos.

\begin{figure}[H]
	\centering
	\begin{subfigure}[t]{0.49\textwidth}
	\hspace*{-2cm}
	\centering
		\includegraphics[width=1.1\textwidth]{Imagenes/polosr6a0.png}
	\end{subfigure}
	\begin{subfigure}[t]{0.49\textwidth}
	\centering
		\includegraphics[width=1.1\textwidth]{Imagenes/cerosr6a0.png}
	\end{subfigure}
	\caption{Posición de los ceros y polos de la transferencia cuando $R_6$ parte de su valor original y tiende a cero.}
	\label{fig:r6a0}
\end{figure}

Observando la transferencia cuando $R_6 \rightarrow 0$ resulta en
\begin{equation}
H(s)=-\frac{R_4}{R_8}
\label{ec:r6a0}
\end{equation}
lo cual se comporta como un amplificador operacional en configuración inversora.

\begin{figure}[H]
	\centering
	\begin{subfigure}[t]{0.49\textwidth}
	\hspace*{-2cm}
	\centering
		\includegraphics[width=1.1\textwidth]{Imagenes/polosr6ainf.png}
	\end{subfigure}
	\begin{subfigure}[t]{0.49\textwidth}
	\centering
		\includegraphics[width=1.1\textwidth]{Imagenes/cerosr6ainf.png}
	\end{subfigure}
	\caption{Posición de los ceros y polos de la transferencia cuando $R_6$ parte de su valor original y tiende a infinito.}
	\label{fig:r6ainf}
\end{figure}

Para $R_6 \rightarrow \infty$ se tiene que la transferencia resulta

\begin{equation}
H(s)=-\frac{S^2 (C_2 R_1 R_3 R_7 (C_6 R_4 - C_7 R_8)) - S (C_2 R_1 R_3 R_8) - R_4 R_7}{S^2 (C_2 R_1 R_3 R_7 R_8 (C_6 + C_7) ) + S (C_2 R_1 R_3 R_8) + R_4 R_7}
\end{equation}

Se puede observar en las Figuras (\ref{fig:r6a0}) y (\ref{fig:r6ainf}) que cambios en la resistencia $R_6$ genera que los ceros y polos no solo se desplacen horizontalmente sino también que estos transicionen de ser complejos conjugados, a estar ubicados en un mismo punto y eventualmente situarse solamente sobre el eje real, apartándose el uno del otro. Además, es observable como si $R_6$ aumenta de su valor original, disminuirá el sobrepico de los ceros y aumentará el sobrepico de los polos. Si $R_6$ tiende a cero, se contempla como baja el factor de calidad de tanto los ceros como los polos, por lo que el filtro pierde su naturaleza de notch y pasa a asemejarse más a la curva asintótica de la transferencia. Esto también puede verse en (\ref{ec:r6a0}). Con este análisis se puede llegar a la conclusión que la resistencia $R_6$ regula el factor de calidad del circuito.


A continuación se analiza el desplazamiento de los polos variando la resistencia $R_8$. 

\begin{figure}[H]
	\centering
	\begin{subfigure}[t]{0.49\textwidth}
	\hspace*{-2cm}
	\centering
		\includegraphics[width=1.1\textwidth]{Imagenes/polosr8a0.png}
	\end{subfigure}
	\begin{subfigure}[t]{0.49\textwidth}
	\centering
		\includegraphics[width=1.1\textwidth]{Imagenes/cerosr8a0.png}
	\end{subfigure}
	\caption{Posición de los ceros y polos de la transferencia cuando $R_8$ parte de su valor original y tiende a cero.}
	\label{fig:r8a0}
\end{figure}

Quedando la función transferencia como

\begin{equation}
H(S) = - \frac{S^2 (C_2 C_6 R_1 R_3 R_6) + S (C_2 R_1 R_3) - R_6}{R_6}
\label{eq:r8a0}
\end{equation}

\begin{figure}[H]
	\centering
	\begin{subfigure}[t]{0.49\textwidth}
	\hspace*{-2cm}
	\centering
		\includegraphics[width=1.1\textwidth]{Imagenes/polosr8ainf.png}
	\end{subfigure}
	\begin{subfigure}[t]{0.49\textwidth}
	\centering
		\includegraphics[width=1.1\textwidth]{Imagenes/cerosr8ainf.png}
	\end{subfigure}
	\caption{Posición de los ceros y polos de la transferencia cuando $R_8$ parte de su valor original y tiende a infinito.}
	\label{fig:r8ainf}
\end{figure}

Resultando en la función transferencia

\begin{equation}
H(S) = \frac{R_6 C_7 R_7 S + R_6 }{S (R_6 R_7 (C_6 + C_7)) + R_6 + R_7}
\end{equation}

Se contempla en las Figuras (\ref{fig:r8a0}) y (\ref{fig:r8ainf}) como pequeñas variaciones en la resistencia $R_8$ logra generar corrimientos en la frecuencia de notch del circuito, manteniendo aproximadamente su forma. Si las variaciones son muy extremas, se pierde la condición de notch ya que varía el factor de calidad del circuito. Además, se puede observar en (\ref{eq:r8a0}) como el sistema se vuelve inestable si $R_8$ tiende a cero ya que la tranferencia tendrá un polo doble en el infinito. Si la resistencia $R_8$ tiende a infinito, el sistema se convierte en un pasa altos o pasa bajos dependiendo los valores de los componentes.

Se puede concluir que la resistencia $R_8$ regula tanto la frecuencia de notch como la selectividad de los ceros. Si la resistencia es muy baja, el sistema se torna inestable.

\subsubsection{Elección de los Amplificadores Operacionales}

Al momento de elegir el operacional a utilizar, se tuvo un claro objetivo: mantener las especificaciones del filtro fieles a lo calculado en el mayor rango de frecuencias posible. Es por esto que uno de los principales parámetros a estudiar en la selección del opamp fue el ancho de banda. Cuanto mayor sea este, más fidedigna será la transferencia en las altas frecuencias, ya que los polos del dispositivo estarán ubicados en mayores frecuencias cuanto más alto sea el ancho de banda.

Otro parámetro importante fueron aquellos que permiten lograr obtener una señal de salida con una baja distorsión armónica. En otras palabras, se buscó no deformar alinealmente a la señal de entrada introduciendo armónicos en esta. Para ello, se consideró la utilización de un amplificador operacional con un alto slew rate.

Finalmente se optó por utilizar el integrado TL-082 el cual posee dos amplificadores operacionales dentro. Se decidieron utilizar estos operacionales ya que poseen un slew rate alto ($13\frac{V}{\mu s}$ valor típico) en comparación a otros operacionales, un ancho de banda grande ($3MHz$ valor típico para ganancia unitaria) y además posee transistores de tecnología J-FET a la entrada del dispositivo, por lo que tanto la corriente de bias como la tensión de offset son más bajos que otro operacional con tecnología BJT a la entrada.

\subsection{Simulación del Circuito}
\label{sec:simulacion}

Se realizó la simulación del circuito utilizando el software \textit{LTSpiceXVII}.

\subsubsection{Simulación de la Transferencia}

Se procedió a realizar la simulación del circuito con los valores obtenidos mediante los paralelos en la Tabla (\ref{Tab:valores}). La primera simulación que se observó fue la respuesta en frecuencia.

\begin{figure}[H]
	\centering
	\begin{subfigure}[t]{0.49\textwidth}
	\hspace*{-2cm}
	\centering
		\includegraphics[width=1.2\textwidth]{Imagenes/bodemag_calc_sim.png}
	\end{subfigure}
	\begin{subfigure}[t]{0.49\textwidth}
	\centering
		\includegraphics[width=1.2\textwidth]{Imagenes/bodefase_calc_sim.png}
	\end{subfigure}
	\caption{Comparación entre los gráficos de bode simulados y calculados teóricamente.}
	\label{fig:bode_calc_sim}
\end{figure}

Utilizando el operacional elegido en la sección \ref{sec:eleccion_componentes}, se observa una respuesta en frecuencia muy similar a la calculada anteriormente, exceptuando dos puntos de interes.

La ganancia en la frecuencia de notch es mucho menor a la calculada. Esto es algo esperado, ya que muchas cosas que no se tienen en consideración en los cálculos logran llegar a un resultado mucho más ideal que el real, como por ejemplo capacitancias e inductancias parásitas como las resistencias de los componentes, terminales y pistas de cobre.

Respecto a las altas frecuencias, se puede observar una discrepancia entre lo calculado y lo teórico. Esta diferencia es mucho más grande tendiendo a las frecuencias cercanas a los $10MHz$. A continuación se grafica el Bode en fase y magnitud entre las frecuencias de los $100KHz$ y los $100MHz$.

\begin{figure}[H]
	\centering
	\begin{subfigure}[t]{0.49\textwidth}
	\hspace*{-2cm}
	\centering
		\includegraphics[width=1.2\textwidth]{Imagenes/bodemag_calc_sim_highf.png}
	\end{subfigure}
	\begin{subfigure}[t]{0.49\textwidth}
	\centering
		\includegraphics[width=1.2\textwidth]{Imagenes/bodefase_calc_sim_highf.png}
	\end{subfigure}
	\label{fig:bode_calc_sim_highf}
	\caption{Comparación entre los gráficos de bode simulados y calculados teóricamente para las frecuencias mayores a $100KHz$.}
\end{figure}

Si bien el operacional elegido posee un gran ancho de banda para mitigar lo más posible estos problemas a las altas frecuencias, como justificado en la Sección (\ref{sec:eleccion_componentes}), se puede observar que los cálculos realizados con el modelo ideal del operacional poseen una gran diferencia respecto a la simulación, la cual tiene en cuenta los varios polos que posee este dispositivo. Debido a esto, se tendrá en cuenta en la Sección (\ref{sec:limitaciones}) las limitaciones del filtro.

\subsubsection{Simulación de la Impedancia de Entrada y Salida}

Se observó luego la impedancia de entrada y salida del circuito.

\begin{figure}[H]
	\centering
	\begin{subfigure}[t]{0.49\textwidth}
	\hspace*{-2cm}
	\centering
		\includegraphics[width=1.1\textwidth]{Imagenes/sim_zin.png}
		\caption{Módulo de la impedancia de entrada.}
	\end{subfigure}
	\begin{subfigure}[t]{0.49\textwidth}
	\centering
		\includegraphics[width=1.1\textwidth]{Imagenes/sim_zout.png}
		\caption{Módulo de la impedancia de salida.}
	\end{subfigure}
	\label{fig:zin_zout}
	\caption{Simulación de la impedancia de entrada y salida del circuito.}
\end{figure}

Se puede ver como la impedancia de entrada tiende a un valor nulo cuanto más alta la frecuencia. Esto era esperable ya que a la entrada se encuentran dos capacitores cuyo camino a través de ellos lleva a tierra. Esto puede llegar a traer varios problemas a la hora de realizar las mediciones del filtro una vez implementado, ya que a altas frecuencias la impedancia de entrada del circuito se hará comparable con la del generador, por lo que se debe tomar un especial cuidado en esta etapa del análisis.

La impedancia de salida puede observarse como se mantiene casi siempre por debajo de los $100\Omega$. Esto también era un resultado esperable ya que la salida del circuito se encuentra conectado a la salida del primer operacional el cual posee una impedancia de salida muy baja.

Cabe notar que si bien la impedancia de entrada parece anularse para frecuencias mayores a $1Khz$, la impedancia se mantiene en el orden de los $K\Omega$ hasta casi llegar a los $100KHz$. A los $10KHz$ su valor es de $5.027K\Omega$ mientras que su valor a los $100KHz$ es de $563\Omega$. A partir de los $\approx 700KHz$ su valor se mantiene por debajo de los $100\Omega$.

\subsubsection{Simulación Montecarlo}
\label{sec:mont}
Se utilizó la simulación Montecarlo para observar la dispersión de la respuesta en frecuencia causada por las tolerancias de los componentes. Se utilizó una tolerancia del 5\% para todos los resistores excepto 1\% para las resistencias de 9k1. Para los capacitores se utilizó una tolerancia del 10\%.

\begin{figure}[H]
	\centering
	\includegraphics[width=\textwidth]{Imagenes/Montecarlo1.PNG}
	\caption{Simulación de Montecarlo.}
	\label{fig:montecarlo}
\end{figure}

Se puede observar una gran dispersión en la frecuencia de notch de aproximadamente $300Hz$ sin tomar en cuenta el caso de más a la derecha.

\subsection{Implementación del Circuito}
Se implementó el circuito de dos maneras distintas para continuar con la investigación acerca de las sensibilidades calculadas y mejorar el aprendizaje de esta herramienta. Se siguieron las consideraciones y el análisis realizado en la Sección \ref{sec:eleccion_componentes} excepto por el preset utilizado en la resistencia $R_1$. En su lugar, se utilizó el mismo paralelo de dos resistencias que para las resistencias de mismo valor. Se decidió utilizar una placa 5x5, resistores al 1\% y 5\% y capacitores de film al 10\%. También se optó por utilizar dos capacitores de desacople de tecnología SMD con un valor de $100nF$ colocados lo más próximo al operacional como posible. Una vez que se calcule la respuesta en frecuencia del circuito, se colocará el preset en la resistencia $R_1$ y se contrastará el error cometido debido a las tolerancias con el error cometido luego de la calibración, el cual se espera que sea lo más cercano a cero.
\subsubsection{Mediciones del Circuito y Análisis de Error}
\label{sec:mediciones}

En una primera instancia se midió la respuesta en frecuencia del filtro sin el preset en $R_1$. Se presenta a continuación.

\begin{figure}[H]
	\centering
	\includegraphics[width=\textwidth]{Imagenes/bode_calc_sim_med.PNG}
	\caption{Diagrama de Bode en módulo medida, simulada y calculada del filtro sin preset.}
	\label{fig:bode_calc_sim_med}
\end{figure}

\begin{figure}[H]
	\centering
	\includegraphics[width=\textwidth]{Imagenes/bodefase_calc_sim_med.PNG}
	\caption{Diagrama de Bode en fase medida, simulada y calculada del filtro sin preset.}
	\label{fig:bodefase_calc_sim_med}
\end{figure}

Se presenta también una tabla con los valores más significantes de la medición contrastándola con los valores deseados del filtro.

\begin{table}[H]
\centering
\begin{tabular}{@{}ccccccc@{}}
\toprule
- & $\omega_z$ & $G_{Banda Pasante}$  & $G_{Notch}$ & $G_{Banda Atenuante}$ \\ \midrule
Valor deseado & $2926Hz$ & - & - & - \\
Valor medido & $2830Hz$ & $0dB$ & $-36.927dB$ & $-6.12dB$ \\
Error & $3.281\%$ & - & - & - \\ \bottomrule
\end{tabular}
\caption{Valores significantes teóricos de la respuesta en frecuencia medida sin preset.}
\label{tab:rta_freq_calc}
\end{table}

Se puede observar que la implementación del filtro low pass notch se apegó a los cálculos y simulaciones realizadas en casi todas las zonas. Las dos áreas donde la discrepancia es mayor es alrededor de la frecuencia de notch y para frecuencias mayores que $1MHz$. 

El error de la frecuencia de notch se encuentra dentro de la dispersión simulada en la sección \ref{sec:mont} y era esperable con la tecnología de los componentes utilizados y sin una opción de calibración.

Para las altas frecuencias mucho mayores a la frecuencia de notch, se observa como el sobrepico alrededor de $1Mhz$ es mucho mayor en la respuesta medida que en la simulada. Se presume que esta diferencia se debe al muy bajo valor al que tiende la impedancia de entrada del circuito para frecuencias elevadas. Esto genera que haya una pérdida de información provista por el generador de tensión conectado a la entrada, ya que la impedancia de este se vuelve considerable frente a la impedancia de entrada del filtro. Cabe notar que cuando se menciona a un generador de tensión se refiere a cualquier fuente de información que pueda estar conectado a la entrada del circuito y la impedancia de este sería su impedancia de salida, la cual es vista por el filtro. 

\begin{figure}[H]
	\centering
	\includegraphics[width=\textwidth]{Imagenes/bode_calc_sim_med_highf.PNG}
	\caption{Diagrama de Bode en fase medida, simulada y calculada del filtro sin preset para altas frecuencias y con puntas x1.}
	\label{fig:bodefase_calc_sim_med_highf}
\end{figure}

Se descartó la posibilidad de que esta anomalía sea a causa de las puntas del osciloscopio ya que cuando estas fueron agregadas a la simulación el sobrepico del cual se habla aumentó en casi $1dB$, mientras que entre la respuesta medida y simulada se encuentra una diferencia de $\approx 8dB$ como puede observarse en la Figura (\ref{fig:bode_calc_sim_med}).

A continuación se grafica nuevamente el bode en amplitud medido haciendo un enfoque en la frecuencia de notch.

\begin{figure}[H]
	\centering
	\includegraphics[width=\textwidth]{Imagenes/bode_calc_sim_med_notch.PNG}
	\caption{Diagrama de Bode en amplitud medido, simulado y calculado del filtro sin preset con un enfoque en la frecuencia de notch.}
	\label{fig:bodefase_calc_sim_med_notch}
\end{figure}

Posteriormente, se decidió colocar el preset en la resistencia $R_1$, calibrar el filtro y realizar nuevamente la medición de los gráficos de Bode.

%%%%%%%%%%%%%%%%%%%%%%%%%%%%%%%%%%%%%%%%%%%%%%%%%%%%%%%%%%%ACA PONER MEDICIONES CON PRESET.

A continuación se midió la impedancia de entrada y de salida del filtro.

\begin{figure}[H]
	\centering
	\begin{subfigure}[t]{0.49\textwidth}
	\centering
		\includegraphics[width=1.1\textwidth]{Imagenes/pend.jpg}
		\caption{Módulo de la impedancia de entrada medida.}
	\end{subfigure}
	\begin{subfigure}[t]{0.49\textwidth}
	\centering
		\includegraphics[width=1.1\textwidth]{Imagenes/pend.jpg}
		\caption{Módulo de la impedancia de salida medida.}
	\end{subfigure}
	\label{fig:zin_zout}
	\caption{Medición de la impedancia de entrada y salida del circuito.}
\end{figure}

\subsubsection{Limitaciones del Circuito}
\label{sec:limitaciones}

Se ha recopilado, de la hoja de datos del \href{http://www.ti.com/lit/ds/symlink/tl082.pdf}{TL-082}, los parámetros utilizados en el análisis de las limitacionnes del circuito.

\begin{table}[H]
\centering
\begin{tabular}{@{}ccccc@{}}
\toprule
$V_{CC_{max}}$ & $V_{in_{max}}$ & $BW_{unitgain}$ & $V_{in_{max}}$ dada $V_{CC}$ & Slew Rate (SR)\\ \midrule
$\pm 18V$ & $\pm 15V$ & $3MHz$ & $\approx V_{CC}-1.5V$ & $13\frac{V}{\mu s}$\\ \bottomrule
\end{tabular}
\caption{Datos recopilados de la hoja de datos del TL-082.}
\label{tab:datos_tl082}
\end{table}

Observando la tabla se puede corroborar que la tensión de alimentación no debe sobrepasar los $18V$, mientras que la de entrada no debe sobrepasar los $15V$. Para el análisis de las limitaciones del circuito, sin embargo, se utilizarán $\pm 15V$ de alimentación, valor recomendado por el fabricante.
Luego, teniendo en cuenta que la ganancia máxima del circuito es de $\approx 2dB$, se tiene que

\begin{equation}
	V_{in_{max}} = \frac{V_{CC} - 1.5V}{G_{max}} = \frac{13.5V}{1.26} = 10.72V
\end{equation}

Sin embargo, también se debe de tener en cuenta el slew rate en el análisis de la tensión de entrada máxima.

\begin{equation}
	SR= Max\left( \frac{\partial (G\cdot A\cdot \sin (\omega t))}{\partial t}\right) = V_{in} \cdot \omega \cdot G  
\end{equation}

Primero se analiza la tensión máxima de entrada en la frecuencia de sobrepico ya que esta posee la ganancia más alta.

\begin{equation}
	V_{in_{max}} = \frac{SR}{G_{max}\cdot \omega} = \frac{1.64\cdot 10^6\frac{V}{s}}{1686Hz} = 972.71
\end{equation}

Luego se analiza la tensión máxima de entrada para las frecuencias altas.

\begin{equation}
	V_{in_{max}} = \frac{SR}{G_{Banda Atenuante}\cdot \omega} = \frac{2.01\cdot 10^6\frac{V}{s}}{f}
\end{equation}

Por lo que se graficó la tensión máxima de entrada final en función de la frecuencia tomando en consideración los cálculos previos.

\begin{figure} [H]
	\centering
	\includegraphics[width=0.7\textwidth]{Imagenes/vin_max.PNG}
	\caption{Tensión de entrada máxima del circuito en función de la frecuencia teniendo en cuenta distorsiones alineales causadas por el operacional.}
	\label{fig:vin_max}
\end{figure}

Luego, el rango de frecuencias en que el circuito opera correctamente estará limitado en su mayoría por dos cosas. La primera es la deformación de la respuesta en frecuencia producida por los polos del amplificador operacional, la cual genera que la ganancia en la banda atenuante del filtro tenga un desvío de $3dB$ al rededor de los $800KHz$. La segunda limitación del rango de frecuencias de correcta operación se debe a la baja impedancia de entrada que ofrece el circuito cuanto más alta es la frecuencia. Esto genera, como ya discutido en la sección \ref{sec:mediciones} una pérdida de información por parte del generador de tensión a la entrada. Cuanto más alta es la frecuencia menor es la impedancia de salida que se requiere en el generador de tensión conectado a la entrada del circuito para que haya una completa transferencia de información. Esta anomalía se comenzaba a hacer más evidente a partir de los $700Khz$. Estas dos cuestiones limitan el rango de frecuencias del circuito por debajo de $\approx 700kHz$.


\subsection{Bibliografía Utilizada}
[1]F. Sergio, Design with operational amplifiers and analog integrated circuits, 4th ed. New York [etc.]: McGraw-Hill, 1988, p. 185.

[2]F. Sergio, Design with operational amplifiers and analog integrated circuits, 4th ed. New York [etc.]: McGraw-Hill, 1988, p. 186.

[3]F. Sergio, Design with operational amplifiers and analog integrated circuits, 4th ed. New York [etc.]: McGraw-Hill, 1988, p. 187.


\end{document}
