\documentclass[a4paper]{article}
\usepackage[utf8]{inputenc}
\usepackage[spanish, es-tabla]{babel}

\usepackage{amsmath}
\usepackage{amsfonts}
\usepackage{amssymb}

\usepackage{float}
\usepackage{graphicx}
\usepackage{subcaption}
\captionsetup{compatibility=false}

\usepackage{multirow}
\setlength{\doublerulesep}{\arrayrulewidth}

\usepackage{array}
\newcolumntype{C}[1]{>{\centering\let\newline\\\arraybackslash\hspace{0pt}}m{#1}}

\usepackage[american]{circuitikz}

\usepackage{fancyhdr}

\usepackage{units} 

\pagestyle{fancy}
\fancyhf{}
\lhead{22.01 Teoría de Circuitos}
\rhead{Mechoulam, Lambertucci, Rodriguez, Londero}
\rfoot{Página \thepage}
\begin{document}

\section{Introducción a diseño de filtros.}
%%%%%%%%%%%%%%%%%%%%%%%%%%%%%%%%%%%
\subsection{Introducción.}
El \textbf{Gyrator} es un componente electrónico inicialmente propuesto como el quinto componente lineal y pasivo luego del resistor, capacitor, inductor y transformador ideal, teninedo la propuesto en 1948 por Bernard D. H. Tellegen el componente podría ... siendo caracterizado por un cuadripolo con parametros impedancia:
$$ Z= \left(\begin{matrix}0&-R\\R&0\end{matrix}\right) $$
Donde R sería la resistencia interna del Gyrator.

\begin{figure}[H]	
	\centering
	\includegraphics[width=0.7\textwidth]{gyratorsimb.png}
	\caption{Simbolo eléctrico del Gyrator.}
	\label{fig:gyrsimb}
\end{figure}
Realmente es implementado utilizando opamps, y uno de los usos mas populares es para la inversión de impedancias, permitiendo simular un inductor utilizando otros componentes.La configuración generalmente utilizada es la siguiente:
\begin{figure}[H]	
	\centering
	\includegraphics[width=0.5\textwidth]{gyrop.png}
	\caption{Equivalente del circuito Gyrator.}
	\label{fig:gyrop}
\end{figure}

%%%%%%%%%%%%%%%%%%%%%%%%%%%%%%%%%%%%%%%%%
\subsection{Calculo analítico.}
Utilizando el circuito de la figura (\ref{fig:gyrop}) se plantearon las siguientes ecuaciones:
\begin{align}   V_{Out} = V^- \label{eq:1}\end{align}
\begin{align} V_{Out} = A(s) \cdot (V^+-V^-)\label{eq:2}\end{align}
\begin{align} A(s)= \frac{A_0}{1+\frac{s}{\omega'}}\end{align}
\begin{align} I_{In}=\frac{V_{In}-V^-}{R_L}+\frac{V^+}{R_g}\end{align}
\begin{align} V^+=V_{In}\cdot \frac{R_g}{R_g+\frac{1}{sC_g}} \end{align}
primero observaremos que en utilizando (\ref{eq:1}) y (\ref{eq:2}) se llega a que 
\begin{align}V^- =V^+ \cdot \frac{A(s)}{A(s)+1} = V^+ \cdot \frac{A_0}{A_0+1+\frac{s}{\omega'}} \approx V^+ \cdot \frac{1}{1+\frac{s}{GBP}} \label{eq:desp}   \end{align}
Donde GBP es un parámetro propio del operacional y suele ser del orden de los MHz. Por lo tanto este termino sería desprecianle siempre que el rango de frecuencias en los cuales se trabaje sea menor a... \center \textcolor{red}{Obtener algun valor significativo}
, teniendo en cuenta esa aproximacion se pueden llegar a las siguientes expresiones:
\begin{align}H(s)= \frac{R_g}{R_g+\frac{1}{sC_g}} \end{align}


\begin{align}Z_{In}=\frac{R_LR_gC_gs+R_L}{sC_gR_L+1}\underset{C_gR_Ls << 1}{\approx}R_gC_gR_L \cdot s + R_g \end{align}

Bajo estos supuestos se puede considerar la impedancia de entrada como una bobina con 
\begin{align}  L=R_gC_gR_L  \ \ \ \ \ \  R_L=r_{coil} \label{eq:basicL1}\end{align}
\flushleft
%%%%%%%%%%%%%%%%%%%%%%%%%%%%%%%%%%%%%%
\subsubsection{Gyrator vs Inductor}
Hablamos de la impedancia de entrada pero tambien es conveniente preguntarse, porque es necesario suplantar a los inductores con circuitos que se comporten de manera similar.
Los inductores ocupan un gran tamaño respecto a el resto de los componentes que pueden ser integrados para reducir aun mas su tamaño, una desventaja del uso del gyrator, es que su rango operativo en el cual se comporta como inductancia no es muy extenso.

%%%%%%%%%%%%%%%%%%%%%%%%%%%%%%%%%%%%%%%%%%
\subsubsection{Impedancia de entrada.}

Dado a que la caracteristica mas atractiva del gyrator es que su impedancia de entrada, bajo ciertas hipotesis, se comporta como un inductor se procedió a medirla y se contrasto con una simulación de \textbf{LTSpice} de la impedancia de entrada de un hyrator y la impedancia de entrada de una bobina, también simulada.

\begin{figure}[H]	
	\centering
	\includegraphics[width=\textwidth]{ZinG.png}
	\caption{Impedancia de entrada del Gyrator.}
	\label{fig:ZinG}
\end{figure}
Puede notarse que el pico el cual presenta el Gyrator simulado no se compara con el medido, esto se atribuye a una posible resonancia que se de en la simulación la cual no se da en el circuito real. Otra cosa a notar es que el gyrator tiene un comportamiento similar a el de una bobina en un rango de frecuencias de 0 ~40 kHz
También se midió la fase obteniendo el siguiente gráfico.
\begin{figure}[H]	
	\centering
	\includegraphics[width=\textwidth]{ZinGp.png}
	\caption{Impedancia de entrada del Gyrator fase.}
	\label{fig:ZinG}
\end{figure}
 Aquí tambien se puede notar la similtud entre las señales hasta aproximadamente 40 kHz, tambien que el cambio de fase en el gyrator medido no es tan brusco comparado  con el simulado.
Finalmente considerando las ultimas mediciones, el rango operativo en el cual se va  a trabajar con el Gyrator será desde 1Hz~40/50 kHz.
%%%%%%%%%%%%%%%%%%%%%%%%%%%%%%%%%%%%%%%
\subsubsection{Elección operacional.}
El operacional que se eligió fue un operacion \textbf{TL-084} dado a su GBP, sus pequeñas corrientes de Bias y que cuenta con 4 operacionales, la cual es una cualidad notable dado que solamente se podrá utilizar 1 integrado.

%%%%%%%%%%%%%%%%%%%%%%%%%%%%%%%%%%%%%%
 \flushleft
\subsection{Filtros de segundo orden.}
Los filtros a implemetar en este trabajo serán filtros de segundo orden siendo estos Low-Pass, High-Pass, Band-Pass y Band-Reject. Y tendran las siguientes especificaciones de diseño:
\begin{table}[H]
\begin{center}
\begin{tabular}{|c|c|c|c|}
\hline
\textbf{Tipo de Filtro} & \textbf{$f_p$ [Hz]} & \textbf{$f_a$ [Hz]} & \textbf{$f_c$ [Hz]} \\ \hline
\textbf{LP}             & 1000                & 3500                & \textbf{-}          \\ \hline
\textbf{HP}             & 10500               & 3000                & \textbf{-}          \\ \hline
\textbf{BP}             & -                   & -                   & 2000                \\ \hline
\textbf{BR}             & -                   & -                   & 3000                \\ \hline
\end{tabular}
\caption{Tabla de especificaciones.}
\label{tab:specs}
\end{center}
\end{table}
Para el caso de LP que tenga ganancia unitaria para la continua que tenga una ganancia mayor a -3dB para frecuencias menores a $f_p$ y menor a -10dB para frecuencais mayores a $f_a$. Para el HP se desea que tenga ganancia unitaria para valores de $f$ muy grandes, que tenga una ganancia mayor a -3dB para frecuencias mayores a $f_p$ y menor a -10dB para frecuencais menores a $f_a$
Algunos parametros que vale la pena recordar de filtros de segundo orden son:
\begin{table}[H]
\begin{center}
\begin{tabular}{c|cl}
$\omega_0$: & Frecuencia de corte del sistema en [$\frac{rad}{s}$] & =$\frac{1}{\sqrt{LC}}$         \\
$\xi$:      & Factor de Amortiguamiento del sistema.               & =$\frac{R\sqrt{C}}{2\sqrt{L}}$ \\
Q:          & Selectividad del sistema                             & =$\frac{1}{2\xi}$             
\end{tabular}
\end{center}
\caption{Parametros útiles.}
\label{tab:utils}
\end{table}


%%%%%%%%%%%%%%%%%%%%%%%%%%%%%%%%%%%%%%%%%%%%
\subsection{High Pass.}
\subsubsection{Circuito utilizando inductor.}
El circuito propuesto para realizar un filtro pasa altos clasicamente es el siguiente:

\begin{figure}[H]
\begin{center}
\begin{circuitikz}
	\node [](v2){};
	\draw (v2) node[label=left:$V_{In}$]{} to[short, o-] ++ (0.5,0) to[R, l = $R$]++ (1.5,0) to[C, l = $C$] ++ (2,0) to[open] ++(0,-2) node[ground](gnd){};
	\draw (gnd) to[L, l_= $L$] ++(0,2);
	\draw (v2) to[open] ++(4,0) to[short, -o] ++(1,0) node[label=right:$V_o$](){};
	\end{circuitikz}
	\caption{Filtro High-Pass básico.}
	\label{fig:basHP}
\end{center}
\end{figure}

Para este circuito es sumamente facil obtener la transferencia, basta con plantear un divisor resistivo, con lo cual se llega a:
\begin{align} 
H(s)=\frac{s^2LC+s\cdot r_{coil}C}{s^2LC+s\cdot(R+r_{coil})C+1}
\label{eq:HPL}
 \end{align}

\subsubsection{Circuito utilizando Gyrator.}
Para la implementación de este filtro utilizando un Gyrator dado a que este usualmente esta referenciado a tierra  se optó por el siguiente diseño:
\begin{figure}[H]	
	\centering
	\includegraphics[width=0.7\textwidth]{gyrHP.PNG}
	\caption{Filtro High-Pass implementado con Gyrator.}
	\label{fig:gyrHP}
\end{figure}



Utilizando las siguientes ecuaciones se puede llegar a la expresión de la transferencia:
\begin{align}V^- = V_{o'}\end{align}
\begin{align}V_{o'} = A(s)\cdot (V^+-V^-)\end{align}
\begin{align}\frac{V_o-V_i}{R+\frac{1}{sC}} = \frac{V^+-V_o}{R_L}+(V^+-V_o)\cdot sC_g\end{align}
\begin{align}\frac{-V^+}{R_g}=(V^+-V_o)\cdot sC_g\end{align}
%\begin{align}\end{align}
\begin{align} H(s) = \frac{C R_L s \left(C_g R_g s + 1\right)}{C C_g R R_L s^{2} + C C_g R_g R_L s^{2} + C R s + C R_L s + C_g R_L s + 1}\label{eq:transgyrHP}\end{align}\footnote{Teniendo en cuenta que se despreció el término (\ref{eq:desp}) de (\ref{eq:transgyrHP})}
Luego teniendo en cuenta (\ref{eq:basicL1}) y tomando que
\begin{align} C_gR_LRC<<LC \implies R<<R_g \end{align}
La transferencia  se puede expresar como:
 \begin{align} H(s) = \frac{LC\cdot s^2+Cr_{coil}\cdot s}{s^2 \cdot (LC)+s\cdot C(R+r_{coil})+1}\label{eq:transgyrHPfinal} 
\end{align}
Vale la pena destacar que (\ref{eq:HPL}) y (\ref{eq:transgyrHPfinal}) son iguales.
Es de utilidad recordar las aproximiaciones utilizadas hasta ahora:
\begin{align}  \frac{1}{1+\frac{s}{GBP}}\approx 1  \ \ \ R<<R_g \ \ \ C_gR_Ls << 1 \label{eq:basicL2}\end{align}
\subsubsection{Comparación de modelos.}
\subsubsection{Respuesta en frecuencia.}
\subsubsection{Analisis de resultados.}

\newpage
%%%%%%%%%%%%%%%%%%%%%%%%%%%%%%%%%%%%%%%%%%%%
\subsection{Low Pass.}
\subsubsection{Circuito utilizando inductor.}
El circuito propuesto para realizar un filtro pasa bajos clasicamente es el siguiente:

\begin{figure}[H]
\begin{center}
\begin{circuitikz}
	\node [](v2){};
	\draw (v2) node[label=left:$V_{In}$]{} to[short, o-] ++ (0.5,0) to[R, l = $R$]++ (1.5,0) to[L, l = $L$] ++ (2,0) to[open] ++(0,-2) node[ground](gnd){};
	\draw (gnd) to[C, l_= $C$] ++(0,2);
	\draw (v2) to[open] ++(4,0) to[short, -o] ++(1,0) node[label=right:$V_o$](){};
	\end{circuitikz}
	\caption{Filtro Low-Pass básico.}
	\label{fig:basLP}
\end{center}
\end{figure}
Obteniendo la transferencia del circuito se obtiene:
\begin{align}
H(s)=\frac{1}{s^2\cdot LC+s\cdot C(R+r_{coil})+1}
\label{eq:LPL}
\end{align}
La cual verifica ser un pasabajos.
\subsubsection{Circuito utilizando Gyrator.}
El gyrator tiene la particularidad de que debe estar referenciado a tierra para funcionar como tal, en este caso lo utilizaremos "flotando" para realizar el filtro pasa bajos, el circuito propuesto fue el siguiente.
\begin{figure}[H]	
	\centering
	\includegraphics[width=0.7\textwidth]{gyrLP.PNG}
	\caption{Filtro Low-Pass implementado con Gyrator.}
	\label{fig:gyrLP}
\end{figure}

Utilizando las siguientes ecuaciones se puede llegar a la expresión de la transferencia:
\begin{align}V^- = V_{o'}\end{align}
\begin{align}V_{o'} = A(s)\cdot (V^+-V^-)\end{align}
\begin{align}\frac{V_{In}-V^+}{R_g}=(V^+-V_n)\cdot s C_g\end{align}
\begin{align}V_{out}\cdot s C=\frac{V_{n}-V_{out}}{R}\end{align}
\begin{align}\frac{V_{n}-V_{out}}{R}=\frac{V^--V_n}{R_L} +(V^+-V_n)\cdot s C_g\end{align}
%\begin{align}\end{align}
Operando se obtiene:
\begin{equation} H(s)= \frac{\left(C_g R_L s + 1\right)}{C C_g R R_L s^{2} + C C_g R_g R_L s^{2} + C R s + C R_L s + C_g R_L s + 1}
\end{equation}
Utilizando (\ref{eq:basicL1}) y (\ref{eq:basicL2}) se llega a que:
\begin{equation} H(s)= \frac{1}{C L s^{2} + s\cdot C (R+r_{coil}) + 1}
\label{eq:LPG}
\end{equation}
Vale la pena destacar que (\ref{eq:LPL}) y (\ref{eq:LPG}) son iguales.
\subsubsection{Comparación de modelos.}
\subsubsection{Respuesta en frecuencia.}
\subsubsection{Analisis de resultados.}
\newpage
%%%%%%%%%%%%%%%%%%%%%%%%%%%%%%%%%%%%%%%%%%%%
\subsection{Band Pass.}
\subsubsection{Circuito utilizando inductor.}
El circuito propuesto para realizar un filtro pasa banda clasicamente es el siguiente:

\begin{figure}[H]
\begin{center}
\begin{circuitikz}
	\node [](v2){};
	\draw (v2) node[label=left:$V_{In}$]{} to[short, o-] ++ (0.5,0) to[R, l = $R$]++ (2.5,0)  to[open] ++(0,-2) node[ground](gnd){};
	\draw (gnd) to[C, l_= $C$] ++(0,2);
	\draw(5,-2) node[ground](gnd2){};
	\draw (gnd2) to[L, l_= $L$] ++(0,2);

	\draw (v2) to[open] ++(3,0) to[short, -o] ++(3,0) node[label=right:$V_o$](){};
	\end{circuitikz}
	\caption{Filtro Band-Pass básico.}
	\label{fig:basBP}
\end{center}
\end{figure}
Operando en el circuito se obtiene:
\begin{align}H(s)=\frac{s\cdot \frac{L}{R}+\frac{r_{coil}}{R}}{s^2 LC+s\cdot (Cr_{coil}+\frac{L}{R})+1+\frac{r_{coil}}{R}}
	\label{eq:BPL}
\end{align}
%\begin{align}\end{align}
\subsubsection{Circuito utilizando Gyrator.}
El circuito propuesto para el filtro Band-Pass es identico a la figura (\ref{fig:basBP}) con la singularidad que intercambiaremos el inductor por un gyrator de la siguiente manera.
\begin{figure}[H]	
	\centering
	\includegraphics[width=0.7\textwidth]{gyrBP.PNG}
	\caption{Filtro Band-Pass implementado con Gyrator.}
	\label{fig:gyrBP}
\end{figure}
Para resolver el circuito se plantearon la siguientes ecuaciones:
\begin{align}V^- = V_{o'}\end{align}
\begin{align}V_{o'} = A(s)\cdot (V^+-V^-)\end{align}
\begin{align} -\frac{V^+}{R_g}=(V^+-V_{out})\cdot sC_g \end{align}
\begin{align}  V_{out}\cdot sC= \frac{V^--V_{out}}{R_L}+\frac{V_{in}-V_{out}}{R}-\frac{V_{out}}{R_g+\frac{1}{sC_g}}\end{align}
Se llega a que:
\begin{align}H(s)=\frac{R_{L} \left(C_{g} R_{g} s + 1\right)}{C C_{g} R R_{L} R_{g} s^{2} + C R R_{L} s + C_{g} R R_{L} s + C_{g} R_{L} R_{g} s + R + R_{L}}
\end{align}
Teniendo en cuenta (\ref{eq:basicL1}) y (\ref{eq:basicL2}) se obtiene:
\begin{align}H(s)=\frac{s\frac{L}{R}+\frac{r_{coil}}{R}}{s^2\cdot LC +s \cdot (Cr_{coil}+\frac{L}{R})+1+\frac{r_{coil}}{R}}
\label{eq:BPG}
\end{align}
Vale la pena destacar que (\ref{eq:BPL}) y (\ref{eq:BPG}) son iguales.
\subsubsection{Comparación de modelos.}
\subsubsection{Respuesta en frecuencia.}
\subsubsection{Analisis de resultados.}
\newpage
%%%%%%%%%%%%%%%%%%%%%%%%%%%%%%%%%%%%%%%%%%%%
\subsection{Band Reject.}
\subsubsection{Circuito utilizando inductor.}
El circuito propuesto para realizar un filtro rechaza banda clasicamente es el siguiente:
\begin{figure}[H]
\begin{center}
\begin{circuitikz}
	\node [](v2){};
	\draw (v2) node[label=left:$V_{In}$]{} to[short, o-] ++ (0.5,0) to[R, l = $R$]++ (1.5,0) to[L, l = $L$] ++ (0,-2) to[open](2,-4) node[ground](gnd){};
	\draw (gnd) to[C, l_= $C$] ++(0,2);
	\draw (v2) to[open] ++(2,0) to[short, -o] ++(1,0) node[label=right:$V_o$](){};
	\end{circuitikz}
	\caption{Filtro Band-Reject básico.}
	\label{fig:basBR}
\end{center}
\end{figure}
Al cual le corresponde una transferencia de la forma:
\begin{align} H(s)=\frac{s^2\cdot LC+s\cdot r_{coil}C+1}{s^2\cdot LC+s\cdot (RC+r_{coil}C)+1}
\label{eq:BRL}
\end{align}
\subsubsection{Circuito utilizando Gyrator.}
\begin{figure}[H]	
	\centering
	\includegraphics[width=0.7\textwidth]{gyrBR.PNG}
	\caption{Filtro Band-Reject implementado con Gyrator.}
	\label{fig:gyrBR}
\end{figure}
Luego se procedió a analizar el circuito para sacar la función transferencia:
\begin{align}V^- = V_{o'}\end{align}
\begin{align}V_{o'} = A(s)\cdot (V^+-V^-)\end{align}
\begin{align}\frac{V_{out}-V_{in}}{R}=(V_n-V_{out})\cdot sC\end{align}
\begin{align}(V_n-V_{out})\cdot sC = \frac{-V_n}{R_g+\frac{1}{sC_g}}+\frac{V^--V_n}{R_L}\end{align}
\begin{align}\frac{-V^+}{R_g}=(V^+-V_n)\cdot sC_g\end{align}
Se llega a :
\begin{align}  H(s)=\frac{C C_{g} R_{L} R_{g} s^{2} + C R_{L} s + C_{g} R_{L} s + 1}{C C_{g} R R_{L} s^{2} + C C_{g} R_{L} R_{g} s^{2} + C R s + C R_{L} s + C_{g} R_{L} s + 1} \end{align}
Teniendo en cuenta (\ref{eq:basicL1}) y (\ref{eq:basicL2}) se obtiene:
\begin{align}  H(s)=\frac{ s^{2}\cdot  LC + s \cdot  C r_{coil} + 1}{ s^{2}\cdot LC + s\cdot (C R+Cr_{coil}) + 1} 
\label{eq:BRG}
\end{align}
\subsubsection{Comparación de modelos.}
\subsubsection{Respuesta en frecuencia.}
\subsubsection{Analisis de resultados.}
%%%%%%%%%%%%%%%%%%%%%%%%%%%%%%%%%%%%%%%%%%%%
\subsection{Diseño en PCB.}
El circuito se diseño en el software \textbf{Altium Designer} dado a las condiciones de diseño requeridas, se optó por utilizar componentes \textbf{SMD} dado a que tienen un menor error y puede mantener mas acotado el tamaño final de la placa.
Se eligio utilizar un integrado TL084 utilizando cada opamp del mismo  para cada filtro y se pusieron capacitores de desacople para el mismo. Se obtuvo el siguiente esquemático:
\begin{figure}[H]	
	\centering
	\includegraphics[width=\textwidth]{esquematico.PNG}
	\caption{Esquemático del circuito.}
	\label{fig:esq}
\end{figure}
Luego se diseño el circuito en el PCB, se eligió utilizar una placa doble faz, dado a que al elegir utilizar componentes SMD y que tambien se deseaba que todos los componentes estén de un lado único no quedó otra opción.
\begin{figure}[H]	
	\centering
	\includegraphics[width=\textwidth]{pcb.PNG}
	\caption{PCB del circuito.}
	\label{fig:pcb}
\end{figure}
\end{document}
%\begin{align}\end{align}