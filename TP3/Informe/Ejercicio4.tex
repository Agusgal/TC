
\subsection{Introducción teórica}
Los amplificadores diferenciales son dispositivos cuya salida es linealmente proporcional a la diferencia entre sus entradas y que idealmente suprimen cualquier tension común a dichas entradas. Teóricamente, de ser iguales las entradas, la salida debería ser igual a cero, no obstante, dicha afirmación es muy difícil de cumplir en la práctica. Debido a lo anterior, se definen dos modos de operación para un amplificador diferencial, modo común y modo diferencial:



\begin{figure}[H]
\begin{center}
\begin{circuitikz}

	\node [op amp](U1){};
	\draw (U1.-) to[short] ++(0, 1) node[](vn){};
	\draw (vn) to[R, l_ = $R_1$] ++(-3, 0) node[label={[font=\footnotesize]above:$v_1$}](v1){} to[american voltage source, l = $\frac{v_{DM}}{2}$, invert] ++(0, -1.5) node[](comun){};
	\draw (comun) to[american voltage source, l = $v_{CM}$] ++(-2.5, 0) to[short] ++(0, -0.5) node[ground]{};
	\draw (U1.+) to[short] ++(0, -1) node[](vp){};
	\draw (vp) to[R, l = $R_3$] ++(-3, 0) node[label={[font=\footnotesize]below:$v_2$}](v2){} to [american voltage source, l_ = $\frac{v_{DM}}{2}$] (comun);
	
	\draw (vn) coordinate(leftR2) to[R, l = $R_2$] (leftR2 -| U1.out) to[short] (U1.out);
	\draw (vp) coordinate(leftR4) to[R, l_ = $R_4$] (leftR4 -| U1.out) to[short] ++(0, -0.5) node[ground]{};
	
	\draw (U1.out) to[short]	++(0.5, 0) node[ocirc, label=right:$v_{out}$]{};

\end{circuitikz}
	\caption{Entradas en términos de los componentes en modo común y modo diferencial.}
	\label{fig:com_dif}
\end{center}
\end{figure}

De la figura anterior pueden deducirse las siguientes relaciones:

\begin{equation}\label{eq:v1}
v_1 = v_{CM} - \frac{-v_{DM}}{2}
\end{equation}

\begin{equation}\label{eq:v2}
v_2 = v_{CM} + \frac{v_{DM}}{2}
\end{equation}

El amplificador operacional en modo diferencial es un caso más general del amplificador de tensión, el cuál por lo general posee uno de sus terminales referenciado a Tierra o $0 Volt$. La ventaja fundamental de la operación en modo diferencial es que de existir alguna señal de ruido común a ambas entradas la misma será cancelada a la salida, por ende los amplificadores diferenciales son útiles en el análisis de pequeñas señales propensas a recibir interferencias, como es el caso de señales biomédicas, circuitos de medición, sensores, etc.

Un caso particular de los amplificadores diferenciales son los amplificadores de instrumentación. Los útlimos cumplen con diferentes características, las cuales los hacen aptos para la instrumentación de pruebas y mediciones; De ahí su nombre. Los AI deben poseer una impedancia de entrada alta en extremo (infinita en caso ideal), una impedancia de salida lo más cercana a cero posible (nula en el caso ideal), una ganancia exacta y estable, y por último un CMRR por lo general, extremadamente elevado(la definición de este parámetro será dada más adelante).






%%Intro propuesta:
En el siguiente ejercicio se diseñará un circuito para posibilitar la medición de presion. Para eso se utilizará el sensor piezoeléctrico MPX2010DP y un amplificador de instrumentación capaz de amplificar la señal proveniente del mismo. El circuito podrá medir en el rango de presiones comprendido entre los $0$ y los $10kP$, para cuyo caso tendrá una salida de entre $0$ y $3,3 V$. Al medir la máxima diferencia de presion permitida por el sensor la señal de salida tendrá una amplitud de al menos $3,1 V$ y menor a $3,3 V$. En caso de medirse una difeencia de presion nula la salida también lo será. El circuito propuesto es el siguiente:


\begin{figure}[H]
\begin{center}
\begin{circuitikz}
	
	\node [op amp](U1){};
	\draw (U1.+) to[short] ++(-0.5, 0) to[sinusoidal voltage source, l = $V_1$] ++(0,-2) node[ground]{};
	\draw (U1.-) to[short] ++(-0.5, 0) node[](vn1){} to[R, l = $R_1$] ++(-2.5, 0) to[short] ++(0, -1.5) node[ground]{};
	\draw (vn1) to[short] ++(0, 1) coordinate(leftR2) to[R, l = $R_2$] (leftR2 -| U1.out) to[short] (U1.out);
	\draw (U1.out) to[open] (4.5, -0.5) node[op amp](U2){};
	\draw (U2.-) to[R, l = $R_3$] (U1.out);
	\draw (U2.+) to[short] ++(-0.5, 0) to[sinusoidal voltage source, l = $V_2$] ++(0,-2) node[ground]{};
	\draw (U2.-) to[short] ++(0, 1) coordinate(leftR4) to[R, l = $R_4$] (leftR4 -| U2.out) to[short] (U2.out);
	\draw (U2.out) to[short] ++(0.5, 0) node[ocirc, label=right:$V_{out}$]{};
	
\end{circuitikz}
	\caption{Circuito de medición propuesto.}
	\label{fig:circuito_inicial}
\end{center}
\end{figure}



Como bien puede apreciarse en la figura anterior, el amplificador de instrumentación cuenta con una configuración dual con solo dos amplificadores operacionales. Ambos están siendo utilizados en su modo no inversor. Para el análisis siguiente se trabajarán ambos op-amps como ideales.

Utilizando superposición se pasiva $V_2$ en primer lugar quedando así elpriemr op-amp en su configuración no inversora, mientras que el segundo en una configuración inversora:

\begin{figure}[H]
\begin{center}
\begin{circuitikz}
	
	\node [op amp](U1){};
	\draw (U1.+) to[short] ++(-0.5, 0) to[sinusoidal voltage source, l = $V_1$] ++(0,-2) node[ground]{};
	\draw (U1.-) to[short] ++(-0.5, 0) node[](vn1){} to[R, l = $R_1$] ++(-2.5, 0) to[short] ++(0, -1.5) node[ground]{};
	\draw (vn1) to[short] ++(0, 1) coordinate(leftR2) to[R, l = $R_2$] (leftR2 -| U1.out) to[short] (U1.out) node[label={[font=\footnotesize]below:$V_{o1}$}](v2){};
	\draw (U1.out) to[open] (4.5, -0.5) node[op amp](U2){};
	\draw (U2.-) to[R, l = $R_3$] (U1.out);
	\draw (U2.+) to[short] ++(-0.5, 0) to[short] ++(0,-2) node[ground]{};
	\draw (U2.-) to[short] ++(0, 1) coordinate(leftR4) to[R, l = $R_4$] (leftR4 -| U2.out) to[short] (U2.out);
	\draw (U2.out) to[short] ++(0.5, 0) node[ocirc, label=right:$V_{out}$]{};
	
\end{circuitikz}
	\caption{Circuito con $V_2$ pasivada.}
	\label{fig:circuito_V2_pasivada}
\end{center}
\end{figure}

Utilizando las expresiones para las ganacias ya conocidas se obtiene la tensión a la salida del primer amplificador y la tensión de salida del circuito:

\begin{equation}\label{eq:vo1}
V_{o1} = V_1\frac{R_1 + R_2}{R_1}
\end{equation}

\begin{equation}\label{eq:V_out_1}
V_{out} = \frac{-R_4}{R_3}V_{o1} 
\end{equation}

Reemplazando \ref{eq:vo1} en \ref{eq:V_out_1}:

\begin{equation}\label{eq:transferencia_1}
V_{out} = (\frac{-R_4}{R_3})(1 + \frac{R_2}{R_1})V_2
\end{equation}

Asimismo, si se pasiva la fuente $V_1$ la tensión despues del amplificador número 1 es cero, ya que ambas entradas están conecatdas a Tierra. Entonces el segundo op-amp esta en su configuración no inversora:

\begin{figure}[H]
\begin{center}
\begin{circuitikz}
	
	\node [op amp](U1){};
	\draw (U1.+) to[short] ++(-0.5, 0) to[short] ++(0,-2) node[ground]{};
	\draw (U1.-) to[short] ++(-0.5, 0) node[](vn1){} to[R, l = $R_1$] ++(-2.5, 0) to[short] ++(0, -1.5) node[ground]{};
	\draw (vn1) to[short] ++(0, 1) coordinate(leftR2) to[R, l = $R_2$] (leftR2 -| U1.out) to[short] (U1.out) node[label={[font=\footnotesize]below:$V_{o1}$}](v2){};;
	\draw (U1.out) to[open] (4.5, -0.5) node[op amp](U2){};
	\draw (U2.-) to[R, l = $R_3$] (U1.out);
	\draw (U2.+) to[short] ++(-0.5, 0) to[sinusoidal voltage source, l = $V_2$] ++(0,-2) node[ground]{};
	\draw (U2.-) to[short] ++(0, 1) coordinate(leftR4) to[R, l = $R_4$] (leftR4 -| U2.out) to[short] (U2.out);
	\draw (U2.out) to[short] ++(0.5, 0) node[ocirc, label=right:$V_{out}$]{};
	
\end{circuitikz}
	\caption{Circuito con $V_1$ pasivada.}
	\label{fig:circuito_V1_pasivada}
\end{center}
\end{figure}

\begin{equation}\label{eq:transferencia_2}
V_{out} = (1 + \frac{R_4}{R_3})V_2
\end{equation}

Por principio de superposición la salida total será la suma entre las dos transferencias anteriores:

\begin{equation}
V_{out} = (1 + \frac{R_4}{R_3})V_2 + (\frac{-R_4}{R_3})(1 + \frac{R_2}{R_1})V_1
\end{equation}

Factorizando la ecuación anterior se llega a:

\begin{equation}\label{eq:transferencia_final}
V_{out} = (1 + 	\frac{R_4}{R_3})(V_2 - \frac{(1 + \frac{R_2}{R_1})}{(1 + 	\frac{R_3}{R_4})}V_1)
\end{equation}


Sin embargo, no es posible factorizar la expresión anterior con tal de llegar a una ecuación de la forma $Ganancia = \frac{V_{out}}{V_2 - V_1}$, a menos que se cumpla la siguiente condición:

\begin{equation}\label{eq:condicion_diferencial}
1 + \frac{R_2}{R_1} = 1 + \frac{R_3}{R_4} \Rightarrow \boxed{\frac{R_2}{R_1} = \frac{R_3}{R_4}}
\end{equation}

Aunque bajo esta condición la ganancia del circuito dependa solamente de ña diferencia de tensiones entre ambas entradas, en la práctica lograr que las resistencias cumplan dicha condición es más complicado de lo que parece. Cualquier desviación generará ruido en la salida, el cual interferirá con la señal que realmente se quiere amplificar. Por ende, ésta variable es de suma importancia a la hora de diseñar e implementar un circuito de esta indole. 

Sabiendo entonces que las resistencias estarán desbalanceadas, pueden plantearse una ganacia en modo diferencial y una ganancia en modo común si se reemplazan \ref{eq:v1} y \ref{eq:v2} en \ref{eq:transferencia_final} y se reordena para lograr uan expresión más limpia:

\begin{equation}\label{eq:ganancias_no_ideales}
V_{out} = V_{CM}((1 + \frac{R_4}{R_3})(1 - \frac{1 + \frac{R_2}{R_1}}{1 + \frac{R_3}{R_4}})) + \frac{V_D}{2}((1 + \frac{R_4}{R_3})(1 + \frac{1 + \frac{R_2}{R_1}}{1 + \frac{R_3}{R_4}}))
\end{equation}

De la ecuación anterior se ven dos cosas, en primer lugar mientras más cerca se esté de la condición de funcionamiento diferencial descrita en \ref{eq:condicion_diferencial} $V_{CM}$ tiende a 0 y la ganancia tiende a:

\begin{equation}\label{ganancia_diferencial_ideal}
\boxed{V_{out} = V_D(1 + \frac{R_4}{R_3}) = (V_2 - V_1)(1 + \frac{R_4}{R_3})}
\end{equation}


En segundo lugar, pueden definirse tanto la ganancia diferencial como la ganancia a modo común:

\begin{equation}\label{eq:ganancia_diferencial}
A_D = \frac{V_{out}}{V_D} = \frac{1}{2}((1 + \frac{R_4}{R_3})(1 + \frac{1 + \frac{R_2}{R_1}}{1 + \frac{R_3}{R_4}}))
\end{equation}

\begin{equation}\label{eq:ganancia_comun}
A_{CM} = \frac{V_{out}}{V_{CM}} = (1 + \frac{R_4}{R_3})(1 - \frac{1 + \frac{R_2}{R_1}}{1 + \frac{R_3}{R_4}})
\end{equation}

Nuevamente se ve en \ref{eq:ganancia_comun} que la misma tiende a 0 a medida que la condiciónd e balance se cumple. Otra posibilidad para eliminar la componente común en la señal de salida es utilizar dos señales cuyo promedio sea igual a 0, en otras palabras, $V_2 = - V_1$ o viceversa. En la práctica, el sensor utilizado devuelve dos señales de es manera para lograr el comportamiento deseado. Que tan exactas sean esa señales será otro factor para tener en cuenta a la hora de determinar la precisión de las mediciones. 


Debido a las dos imperfeccciones que ya se describieron, el balance de las resistencias y la igualdad entre las señales de entrada, se define un factor de mérito con el objetivo de describir el funcionamiento de un amplificador de instrumentación, el factor de rechazo a modo común o CMRR por sus siglas en inglés. El CMRR es una medida del rechazo que ofrece la configuración a la entrada de  voltaje común. El mismo es positivos y se mide en decibelios, e idealmente es igual a infinito.

\begin{equation}\label{eq:CMRR}
CMRR   =   20\log(\frac{A_D}{A_{CM}})   = 20\log(\frac{\frac{1}{2}(1 + \frac{1 + \frac{R_2}{R_1}}{1 + \frac{R_3}{R_4}})}{1 - \frac{1 + \frac{R_2}{R_1}}{1 + \frac{R_3}{R_4}}})
\end{equation}

%%Montecarlo de las resistencias hay que agregar
\subsubsection{Análiss teórico de error}
Cualquier amplificador de instrumentación será insensible a $v_{CM}$ mientras se cumpla que los amplificadores sean ideales y la relación \ref{eq:condicion_diferencial} sea cierta. Si se considera que al menos el comportamiento de loas amplificadores se asemeja al ideal se puede analizar como las discordancias entre los valores nominales y reales de las resistencias afectan al circuito. En general puede decirse que si la condición de puente esta desbalanceada el circuito responderá tanto a $v_D$ como a $v_{CM}$.

Si se introduce un factor de desbalance $\epsilon$ y se asume que solo uno de los cuatro resistores posee una desviación de su valor nominal expresamos a $R_2$ como:

\begin{equation}\label{R_2_epsilon}
R_2(1 - \epsilon)
\end{equation} 

Si se propone que los resistores tienen una tolerancia igual a $p$ el peor caso posible es cuando, por ejemplo, la relación $\frac{R_3}{R_4}$ es maximizada y $\frac{R_2}{R_1}$ es minimizada. La anterior se cumple cuando $R_3$ y $R_1$ son maximizadas y $R_4$ y $R_2$ son minimizadas.Pero para que la condición de balance siga cumpliendose las resistencias minimizadas deben ser multiplicadas por $(1 + p)$ y las maximizadas por $(1 - p)$:

\begin{equation}
\frac{R_3(1 - p)}{R_4(1 + p)} = \frac{R_2(1 + p)}{R_1(1 - p)}
\end{equation}

Reagrupando:

\begin{equation}
\frac{R_3}{R_4} = \frac{R_2(1 + p)^2}{R_1(1 - p)^2}
\end{equation}

Para $p \gg 1$ se cumple que  $\frac{1}{(1 + p)} \cong (1 - p)$, entonces:

\begin{equation}
\frac{R_3}{R_4} = \frac{R_2(1 + p)^2}{R_1(1 - p)^2} \cong	 \frac{R_2(1 - p)^2}{R_4(1 - p)^2} \cong \frac{R_2(1 - 4p)}{R_4}
\end{equation}

Lo anterior si se consideran los terminos de $p^n$ con $n > 2$ como cercanos a $0$. Ahora bien, si se compara la ecuación anterior con \ref{R_2_epsilon} se llega a la siguiente conclusión:

\begin{equation}
|\epsilon|_{max} \cong 4p
\end{equation}



Si se reemplaza $\epsilon$ en \ref{eq:ganancias_no_ideales}:

\begin{equation}
V_{out} = V_{CM}((1 + \frac{R_4}{R_3})(1 - \frac{1 + \frac{R_2(1 - \epsilon)}{R_1}}{1 + \frac{R_3}{R_4}})) + \frac{V_D}{2}((1 + \frac{R_4}{R_3})(1 + \frac{1 + \frac{R_2(1 - \epsilon)}{R_1}}{1 + \frac{R_3}{R_4}}))
\end{equation}

Si nuevamente se separa $V_{out}$ en:

\begin{equation}
V_{out} = A_DV_D + D_{CM}V_{CM}
\end{equation}

\begin{equation}
A_D = \frac{1}{2}((1 + \frac{R_4}{R_3})(1 + \frac{1 + \frac{R_2(1 - \epsilon)}{R_1}}{1 + \frac{R_3}{R_4}}))
\end{equation}

\begin{equation}
A_{CM} = \frac{V_{out}}{V_{CM}} = (1 + \frac{R_4}{R_3})(1 - \frac{1 + \frac{R_2(1 - \epsilon)}{R_1}}{1 + \frac{R_3}{R_4}})
\end{equation}

\subsubsection{Comentarios respecto al funcionamiento del sensor}

El dispositivo utilizado para medir la diferencia de presion es un sensor piezoresistivo de silicio. Su principio de funcionamiento es simple: se aprovecha de cambios en la resistencia eléctrica del material que lo constituye generados por la diferencia de presion entre dos entradas. El sensor proporciona una salida de tensión lineal, precisa y directamente proporcional a la presión aplicada. El sensor posee cuatro pines: uno de alimentación, Tierra, y las dos salidas con signos opuestos antes mencionadas. El mismo opera en diferencias de presión entre los $0$ y $10kP$. Según las curvas del fabricante la tensión típica de salida va de $0$ a aproximadamente $25mV$, mientras que para el caso mínimo la misma oscila entre los $-2$ y $22,5mV$ y para el máximo entre los $2$ y los $27,5mV$. Hay que tener en cuenta los valores anteriores a la hora de elegir las resistencias a utilizar, para configurar la ganancia deseada. 

Por último se mencionan las ecuaciones manjeadas para el cálculo de presion. La presión hidrostática en un contenedor líquido depende solamente de tres factores y viene dada por la siguiente expresion:

\begin{equation}
P = \rho gh
\end{equation}

Donde $\rho$ es la densidad del líquido (en caso de ser agua $997\frac{kg}{m^3}$), $g$ es la aceleración de la gravedad ($9,7968\frac{m}{s^2}$ en la ciudad de Buenos Aires) y $h$ es la profundidad a la que está sumergido el objeto en metros. Con estos datos e debería sumergir un objeto hasta los $1,02$ metros de profundidad para llegar a los $10kPa$, la presión máxima que es capaz de medir el sensor utilizado. 


\subsection{Implementación práctica}

En primer lugar es necesario fijar la ganancia del circuito mediante la elección de las resistencias. Si se toman los valores típicos de la tensión de salida del sensor y se considera el funcionamiento en modo diferencial ideal dado por \ref{ganancia_diferencial_ideal}, para lograr una tension de salida de como mínimo $3,1V$ se necesita una ganancia como mínimo igual a $124$, por lo que $\frac{R_4}{R_3} = 123$. Es importante utilizar el mínimo número de resistencias, en lo posible solamente 4, con tal de minimizar el error producido por las tolerancias de las mismas. Con este fin se decidió utilizar resistencias SMD con 1 porciento de tolerancia, de $150k\Omega$ y $1,2k\Omega$, que dan una ganancia igual a $126$, o sea, una tension máxima teórica igual a $3,15V$.

En segundo lugar, se eligió el integrado $LM833$ de Texas Instruments debido a su gran impedancia de entrada, la cual permite tener flexibilidad en la elección de las resitencias, y evita cargar la fuente de la señal de entrada lo que podría hacer que la amplitud de la misma disminuyese. 
%%Comentar sobre CMRR 
Además de lo anterior se propone agregar un potenciómetro de la siguiente manera con el fin de poder regular la ganancia del circuito facilmente:

 
\begin{figure}[H]
\begin{center}
\begin{circuitikz}
	
	\node [op amp](U1){};
	\draw (U1.+) to[short] ++(-0.5, 0) to[sinusoidal voltage source, l = $V_1$] ++(0,-2) node[ground]{};
	\draw (U1.-) to[short] ++(-0.5, 0) node[](vn1){} to[R, l = $R_1$] ++(-2.5, 0) to[short] ++(0, -1.5) node[ground]{};
	\draw (vn1) to[short] ++(0, 1) coordinate(leftR2) to[R, l = $R_2$] (leftR2 -| U1.out) to[short] (U1.out);
	\draw (U1.out) to[open] (4.5, -0.5) node[op amp](U2){};
	\draw (U2.-) to[R, l = $R_3$] (U1.out);
	\draw (U2.+) to[short] ++(-0.5, 0) to[sinusoidal voltage source, l = $V_2$] ++(0,-2) node[ground]{};
	\draw (U2.-) to[short] ++(0, 1) coordinate(leftR4) to[R, l = $R_4$] (leftR4 -| U2.out) to[short] (U2.out);
	\draw (U2.out) to[short] ++(0.5, 0) node[ocirc, label=right:$V_{out}$]{};
	
	\draw (leftR2) to[short] ++(0, 1.5) coordinate(leftRg) to[vR, l = $R_G$] (leftRg -| U2.-) to[short] (U2.-);
\end{circuitikz}
	\caption{Circuito de medición con potenciómetro.}
	\label{fig:circuito_con_potenciometro}
\end{center}
\end{figure}

Puede demostrarse que la nueva ganancia del circuito al incluir el potenciómetros es igual a:

%%Agregar demostración..

\begin{equation}
\boxed{A = 1 + \frac{R_4}{R_3} + \frac{2R_2}{R_g}}
\end{equation}

%%Agregar justificacion de porque estos valores, tiene qeu ver con impedancia de entrada, etc.
%%Desvenatja del circuito esta en CMMR mas bajo comparado al circuito de 3 ver analog devices link en carpeta


\subsubsection{Mediciones}
La primera medicion que se realizo fue la de la ganancia a modo comun. Para poder medirla se conectaron ambas entradas del circuito al mismo generador y se estimul\'o el circuito con una entrada continua. Las tensiones fueron medidas con un mult\'imetro de banco debido a que se trabaj con valores demasiado pequenos.


\begin{table}[H]
\centering
\begin{tabular}{cccc}\hline
$V_{in}(V_{DC})$ & $V_{out}(mV_{DC})$ & $A_{CM}$ & $A_{CM} (dBs)$ \\
\hline
$0,25$ & $4$ & $0,008$ & $-41,9$ \\
$0,5$   & $4,2$  & $0,0084$ & $-41,51$ \\
$1$  & $4,5$ & $0,0045$ & $-46,9$\\
$1,5$  & $5$ & $0,0033$ & $-49,54$\\
$2$  & $5,5$ & $0,00275$ & $-57,23$\\
$2,5$  & $5,9$ & $0,00236$ & $-52,54$\\
$3$  & $7,1$ & $0,00366$ & $-52,51$\\
$3,3$  & $8,4$ & $0,00254$ & $-51,88$\\
$5$  & $10,3$ & $0,00206$ & $-53,72$\\
$7,5$  & $14$ & $0,00187$ & $-54,57$\\ \hline
\end{tabular}
\caption{Ganancia modo com\'un}
\label{table:ganancia_comun}
\end{table}


De la tabla anterior puede verse que al nivel de tension m\'aximo que se est\'a manejando ($3,3V$), la tension en modo com\'un a la salida del circuito es menor al umbral de ruido, por ende no afecta al funcionamiento del mismo. 


Por otro lado se hicieron las mediciones de la ganancia a modo diferencial. El primer inconveniente que se tuvo con la metodolog\'ia empleada fue que al conectar n generador distinto a cada entrada, debido a que no estaban en fase, la salida oscilaba entre dos valores. Por consiguiente, se decidi\'o medir la tension de salida conectando una entrada a Tierra y otra al generador. 

\begin{table}[H]
\centering
\begin{tabular}{cccc}\hline
$V_{in}(mV)$ & $V_{out}(V)$ & $A_{CM}$ & $A_{CM} (dBs)$ \\
\hline
$7$   & $0,8$  & $114,28$ & $41,16$ \\
$9$  & $1,07$ & $118,9$ & $41,5$\\
$11$  & $1,33$ & $121$ & $41,65$\\
$13$  & $1,61$ & $123,84$ & $41,85$\\
$15$  & $1,87$ & $124,7$ & $41,91$\\
$18$  & $2,26$ & $125,55$ & $41,97$\\
$20$  & $2,54$ & $127$ & $42$\\
$22$  & $2,8$ & $127,27$ & $42,1$\\
$25$  & $3,2$ & $128$ & $42,14$\\
$26$  & $3,34$ & $128,46$ & $42,17$\\
$30$  & $3,8$ & $126,7$ & $42$\\
$40$  & $5,22$ & $130,5$ & $42,31$\\
$50$  & $6,56$ & $131,2$ & $42,35$\\
$70$  & $9,18$ & $131,14$ & $42,35$\\ \hline
\end{tabular}
\caption{Ganancia modo diferencial}
\label{table:ganancia_diferencial}
\end{table}

Si se toma como referencia $25mV$, es decir el valor de tension m\'axima te\'orica que el sensor puede otorgar, el rechazo a modo comun es igual a:

\begin{equation}
CMRR = 20\log(\frac{128}{0,008}) = \boxed{84 dBS}
\end{equation}

El valor del $CMRR$ es menor al t\'ipico otorgado por el fabricante ($100dBs$) y en el rango de potenciales manejado es aceptable considerado la funcion del circuito. Una pregunta que surge al analizar dicho valor es porque no se acerca mas al ideal. En el mejor de los casos el rechazo deberia dar en el orden de los cientos de $dBS$(idealmente infinito). No obstante lo anterior, las resistencias deben estr perfectamente balanceadas para cuplir dicha condici\'on, hecho que en la pr\'actica no se da. Adem\'as, una caracter\'istica que diferencia a la configuraci\'on usada de otras mas usuales como por ejemplo el circuito del ejercicio 3 de este mismo informe es su falta de simetria. Si se sigue el recorrido de la primera senal, la misma pasa por un no inversor para luego ser invertida, mientras que la segunda senal pasa solamente por un no inversor. Esta diferencia de fase, genera que cuando se tienen que restar las senales, la operacion no da excatamente cero. En el caso del circuito desarrollado en el inciso anterior, debido a su alto grado de simetria ambas senales tienen la misma fase a la salida y el rechazo es aun mayor. Por ende no solo los valores reales de los componentes tienen que ser consideados cuando se disena un amplificador de instrumentacion, sino que tambien la simetria y distrubucion del mismo. 


Como último comentario respecto a la medicion de la ganancia diferencial, anteriormente se comentó que existe un problema para medir con dos generados dessincronizados. Una manera arreglar el desfase entre generadores es realizar un circuito no inversor simple con ganancia igual a $-1$ para obtener la misma senal que la del generador pero invertida. Luego la tension diferencial ser\'a igual al doble de la tensio del generador. Se prob\'o esta soluci\'on y se obtuvieron resultados identicos a los de la tabla \ref{table:ganancia_diferencial}.

En ultimo lugar se presentan los resultados de las mediciones con el tubo de agua:

\begin{table}[H]
\centering
\begin{tabular}{ccc}\hline
$Altura columna(m)$ & $V_{out}(V)$ & $Presion(kP)$ \\
\hline
$0$   & $0,005$ &   \\
$0,1$  & $0,323$ & 1 \\
$0,15$  & $0,416$ & 1,5 \\
$0,34$  & $1,031$ & 5,1 \\
$0,51$  & $1,588$ & 5,1\\
$0,59$  & $1,815$ & 5,9\\
$0,74$  & $2,337$ & 7,4\\
$0,79$  & $2,508$ & 7,9\\
$0,88$  & $2,795$ & 8,8\\
$0,9$  & $2,858$ & 9\\
$0,96$  & $3,083$ & 9,6\\
$1,02$  & $3,255$ & 10,2\\ \hline
\end{tabular}
\caption{Mediciones presion}
\label{table:presion}
\end{table}

Cuando se deja el sensor con las dos bocas al aire se mide un offset de $5mV$, el cual entra en el rango de tensiones medidas con ambas entradas en modo comun anteriormente. Por los resultados obtendos y debido a la calibracion previa del circuito se obtiene una tension igual a $3,255V$ pasados la diferencia de presion maxima del sensor. 


\subsection{Medicion de volumen}

Se sabe que la presion esta dada por:

\begin{equation}
P = \rho gh 
\end{equation}

Y el volumen en un contenedr cil\'indrico es:
 
\begin{equation}
V_l = \pi R^2h 
\end{equation}

Se despeja $h$ de la primera ecuaci\'on y se tiene:

\begin{equation}\label{eq:volumen}
V_l = \frac{\pi R^2P}{g\rho}
\end{equation} 

Adem\'as la presion medida tiene una relacion lineal con la salida del sensor, la cual puede encontrarse en la datasheet del mismo. Basicamente se tiene una recta con pendiente $2,5\frac{mV}{kP}$:

\begin{equation}
V_{OutSensor} = 2,5P
\end{equation}

Si tenemos en cuenta que amplificamos la senal $126$ veces:

\begin{equation}
V_{OutAmplificador} = V_{OutSensor}*2,5*126 \Longrightarrow P = \frac{V_{OutAmplificador}}{315}
\end{equation}

La ecuacion anterior nos da la relacion entre la tension de salida del amplificador (en $mV$) y la presion medida (en $kP$). Si se reemplaza $P$ en \ref{eq:volumen}:

\begin{equation}
V_l = \frac{\pi R^2V_{OutAmplificador}}{g\rho*315} \Longleftrightarrow V_l = k*V_{OutAmp}
\end{equation} 

Por ende se obtuvo una expresion que relaciona la tension de salida de circuito con el volumen del contenedor para una ganancia dada. Si se chequean las unidades correspondientes se ve que las mismas concuerdan. En un principio lo anterior significar\'ia que no es necesario modificar el circuito para calcular el volumen del contenedor, solamente se cambiar\'ian los valores de la ganancia y de la densidad en caso de usarse otro l'iquido que no sea agua. A pesar de eso, es necesario teer en cuanta la limitaci\'on del volumen m\'aximo y la presion m\'axima medible con el sensor.


\subsection{Conclusiones}

En esta experiencia de laboratorio se arm\'o un amplificador amplificador que tendr\'ia que cumplir con ciertos criterios para poder ser aplicado en la medici\'on de senales de pequena amplitud. LA primera de dichas condiciones era la de una gran imedancia de entrada la cual queda cumplida con la eleecion del amplificador operacional y la configuracion misma del circuito. La segunda y m\'as importante, es un rechazo de modo com\'un alto, considerando que se apunta a un rechazo en el orden de los $100dBs$ para poder eliminar el ruido de la medicion, los resultados obtenidos son satisfacotrios. M\'as a\'un si se tiene en consideracion la falta de simetria del circuito desarrollado. Por ultimo, los valores obtenidos en la medicion de la columna de agua corresponden a las tensiones teoricas propuestas. Gracias al valor del $CMRR$, se asegura eliminar casi todo el ruido que pueda interponerse en la medicion logrando asi un funcionamiento adecuado como amplificador de instrumentacion.
